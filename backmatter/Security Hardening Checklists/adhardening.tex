[] 1. \textbf{Implement account lockout policies} to slow down enumeration attempts.
[] 2. Limit LDAP access to reduce exposure with network segmentation and access controls.
[] 3. Enforce strong password policies. Include minimum length, character diversity, and password age. 
[] 4. Use MFA to make it challenging for attackers to gain unauthorized access with an extra layer of security
[] 5. Enable LDAP signing and channel binding to help protect against Man-in-the-Middle (MiTM) attacks and reduce the risk of unauthorized access to service accounts.
[] 6. Use Group Managed Service Accounts (gMSAs) to provide automatic password management and tpo help mitigate the risk of password hash exposure.
[] 7. Enabled Privileged Access Management (PAM) to limit the exposure of privileged credentials and reduces the attack surface for Kerberoasting.
[] 8. Secure AD CS configurations by restricting the use of certificate templates to necessary permissions and limiting the certificates users can request and assign.
[] 9. Adopt and implement the Principle of Least Privilege to reduce the risk of unauthorized certificate requests by granting users, computers, and services to only the most minimum and least privileges required as necessary to complete job duties.
[] 10. Audit AD CS setup and periodically check for misconfigurations or potential vulnerabilities that could potentially be leveraged for exploitation.
[] 11. Monitor issued certificates and keep track of issued certificates and their associated permissions.
[] 12 Implement security monitoring and alerting. Set up alerts for suspicious activities, such as: Unauthorized replication requests | The use of known DCSync-related tools | Unusal Kerberos ticket requests | The use of known Golden Ticket-related tools like Mimikatz.













