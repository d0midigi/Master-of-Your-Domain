%%%%%%%%%%%%%%%%%%%%%%acronym.tex%%%%%%%%%%%%%%%%%%%%%%%%%%%%%%%%%%%%%%%%%
% sample list of acronyms
%
% Use this file as a template for your own input.
%
%%%%%%%%%%%%%%%%%%%%%%%% Springer %%%%%%%%%%%%%%%%%%%%%%%%%%

\Extrachap{Glossary}


Use template \emph{glossary.tex} together with the Springer document class SVMono (monograph-type books) or SVMult (edited books) to style your glossary\index{glossary} in the Springer layout.

\section*{A}
\runinhead{Access Control} A foundational security concept that ensures that only authorized individuals or systems can access specific resources within a computing environment. Access control mechanisms include physical controls (e.g. biometric scanners), logical controls (e.g. passwords), and administrative controls (e.g., policies). It is commonly enforced in networks and applications using models such as DAC, MAC, or RBAC.

\runinhead{Access Control Entry (ACE)} An Access Control Entry (ACE) is an individual permission record within an Access Control List (ACL). Each ACE defines which users, groups, or processes have rights, such as reading, writing, or modifying, on an object. In Active Directory, ACEs regulate access to directory objects such as users, groups, organizational units (OUs), and domain controllers. Misconfigured or maliciously altered ACEs can grant attackers hidden or excessive permissions, allowing persistence, privilege escalation, or covert control of critical AD assets. Regular audit of ACE configurations is essential to maintain security.
\textit{See also: access control list (ACL)}

\runinhead{Access Control List (ACL)} An access control list (ACL) is a list of access control entries (ACEs) that apply to an AD object (e.g., a user, group, or computer object). Each ACE in an ACL identifies a trustee and specifies the access rights allowed, denied, or audited for that trustee. Incorrectly configured ALCs can lead to unauthorized access or data exposure. Unexpected or unusual modifications to ACLs can indicate potential security incidents or unauthorized attempts to gain elevated permissions within the AD infrastructure. Monitoring anomalous ACL changes, detecting and investigating them is crucial for maintaining a secure AD environment.
\textit{See also: Access Control Entry (ACE)}

\runinhead{Access Control List (ACL) Persistence Attack} An Access Control List (ACL) Persistence Attack is a technique by which attackers modify the permissions on Active Directory objects to maintain long-term stealthy access. ACLs define which users or groups can perform actions, such as read, write, or modify, on AD objects like users, groups, or organizational units. The thought process behind this attack is that, by granting themselves hidden or excessive permissions (e.g., through the \verb|WriteDacl| or \verb|GenericAll| rights), attackers can ensure continued control of AD resources even after passwords are reset or accounts are rotated. ACL persistence is difficult to detect because it exploits legitimate AD functionality rather than introducing malware, making regular auditing of AD permissions and monitoring for unusual changes essential defenses.

\runinhead{Access Token} An access token is a security object generated by Windows after a successful log-in that defines the user's identity and permissions. It contains Security Identifiers (SIDs) for the user, their groups, and assigned privileges, and it is attached to every process the user runs. The thought process behind access tokens is to ensure that Windows and Active Directory enforce access control consistently across all resources. In an Active Directory environment, attackers often target access tokens to impersonate users, escalate privileges, or move laterally through techniques like Pass-the-Token or token manipulation. Monitoring token usage and restricting privileged account access are key defenses against these attacks.

\runinhead{Access Token Manipulation} Access token manipulation is a post-exploitation technique in which attackers alter or forge access tokens to impersonate legitimate users or elevate privileges on a Windows system. Access tokens contain the Security Identifiers (SIDs) and permissions that determine what actions a user or process is allowed and authorized to perform. In an Active Directory environment, the manipulation of access tokens is to abuse legitimate AD credentials or processes to gain unauthorized access. Common methods include token impersonation, token duplication, and exploiting privileged services that can spawn processes with higher-level tokens.

\runinhead{Account Lockout Policy} The account lockout policy defines the conditions under which a user account is locked after multiple failed attempts to log in. It includes settings such as the lockout threshold, lockout duration, and reset counter. This policy helps protect against brute-force and password-guessing attacks but, if too strict, can be abused by attackers to trigger denial-of-service conditions against users. In Active Directory environments, account lockout policies are often enforced via Group Policy.

\runinhead{Active Directory (AD)} Active Directory (AD) is Microsoft’s directory service that provides centralized authentication, authorization, and management of users, groups, devices, and other resources in a Windows domain network. AD is a Tier 0 asset because it controls access to nearly all systems and data throughout the enterprise. Its compromise often results in full organizational control for attackers.

\runinhead{Active Directory Administrative Center (ADAC)} Active Directory Administrative Center (ADAC) is a Microsoft Management Console (MMC) tool that provides administrators with a modern interface to manage Active Directory objects. Built on PowerShell, ADAC allows tasks such as user account creation, group management, and fine-grained password policy administration. Because ADAC exposes powerful AD operations, compromised admin accounts using it can be leveraged for large-scale changes or privilege escalation.

\runinhead{Active Directory Application Mode (ADAM)} Active Directory Application Mode (ADAM) is a lightweight version of Active Directory designed for directory-enabled applications without requiring full AD domain services. Provides LDAP-based directory services for applications while remaining separate from the core AD domain. Today, ADAM has been largely replaced by Active Directory Lightweight Directory Services (AD LDS).
\textit{Succeeded by Active Directory Lightweight Directory Services (AD LDS)}

\runinhead{Active Directory Attack Lifecycle (AD Attack Lifecycle)} The AD attack lifecycle describes the typical stages that attackers follow when targeting Active Directory environments. It includes reconnaissance (mapping AD objects and trust relationships), initial access, credential theft (e.g., Kerberoasting or LSASS dumping), privilege escalation, lateral movement, persistence (such as ACL or DCShadow abuse) and finally domain dominance. Understanding this lifecycle helps defenders detect and disrupt attacks before the full compromise of AD.

\runinhead{Active Directory Backup (AD Backup)} An Active Directory backup is the process of creating and storing a copy of the AD database and the state of the system to ensure recovery in the event of corruption, ransomware, or accidental deletion of critical objects. AD backups typically include the NTDS.dit database, the SYSVOL folder, and the registry settings, all of which are necessary to restore domain controllers and directory services. In large enterprises, AD backups are essential for business continuity, as Active Directory underpins authentication and access for nearly every resource. Without a working backup, recovering from a compromise or failure can be impossible, which is why AD backups are considered a Tier 0 defense mechanism.
\textit{See also: System-State Restore}

\runinhead{Active Directory Canary (AD Canary) / Canaries)} An Active Directory canary is a decoy object, such as a fake user, group, or service principal, intentionally placed in the directory to detect suspicious activity. If an attacker queries, attempts authentication, or attempts to use the canary account, an alert is triggered, signaling that AD reconnaissance or exploitation is underway. Canaries are valuable for early detection because attackers often enumerate AD for accounts and permissions as one of their first steps. Since these accounts do not serve a legitimate function, any interaction with them is a strong indicator of malicious activity.

\runinhead{Active Directory Certificate Services (AD CS)} Active Directory Certificate Services (AD CS) is a Windows Server role that provides a Public Key Infrastructure (PKI) to issuing, managing, and re-voting digital certificates. These certificates are used for secure authentication, encryption, and digital signing in AD environments; however, when misconfigured, AD CS can be exploited through attacks such as ESC1–ESC8 (Enterprise Security Controls) to forge certificates and impersonate domain administrators. Because certificates can bypass password expiration policies and provide long-lived access, AD CS security is a major focus in modern AD hardening strategies.
\textit{See also: Public Key Infrastructure (PKI)}

\runinhead{Active Directory Disaster Recovery (AD DR)} Active Directory disaster recovery refers to strategies and processes used to restore AD services after catastrophic events such as hardware failures, ransomware attacks, or widespread corruption. Unlike object-level recovery, disaster recovery typically involves restoring entire domain controllers, forests, or replication topologies from backup. Since AD is a Tier 0 service that governs authentication and access throughout the enterprise, disaster recovery planning is critical. Enterprises must test recovery procedures regularly to ensure that forests and domains can be rebuilt quickly without leaving lingering backdoors.

\runinhead{Active Directory Domain Services (AD DS)} AD DS is the core directory service within Active Directory, responsible for storing and organizing information about users, groups, devices, and policies in a domain. Provides essential authentication and authorization services using Kerberos and LDAP, making it the foundation of access control in a Windows network. AD DS operates through a hierarchical structure of forests, domains, and organizational units (OUs). Compromise of AD DS means compromise of the entire identity infrastructure, making it one of the most valuable targets for attackers.

\runinhead{Active Directory Domain Services (AD DS) PowerShell Module}
The Active Directory Domain Services (ADDS) PowerShell module provides a collection of cmdlets that allow you to use PowerShell to manage and administer various aspects of AD, such as users, groups, computers, and organizational units.

\runinhead{Active Directory Federation Services (AD FS)} Active Directory Federation Services (AD FS) is a feature that extends authentication beyond a single domain or organization, enabling Single Sign-On (SSO) with external applications, cloud services, and business partners. Using AD credentials, AD FS provides seamless access across environments without requiring users to repeatedly authenticate. Although powerful for integration, AD FS can be abused by attackers who compromise it to forge authentication tokens (such as Golden SAML attacks), effectively granting domain- or enterprise-wide access. Securing AD FS is therefore critical for hybrid and federated environments.

\runinhead{Active Directory Forest Recovery (ADFR)} An Active Directory forest recovery is the most comprehensive form of AD restoration, involving rebuilding of an entire forest when it has been compromised or corrupted beyond repair. This process requires careful use of backups, authoritative restores, and reestablishing trust relationships between domains. Since the forest is the ultimate security boundary in AD, its compromise often equals a complete loss of organizational control. Forest recovery is considered one of the most challenging IT operations and should be regularly tested as part of AD disaster recovery planning.

\runinhead{Active Directory Hardening} Active Directory hardening is the process of applying best practices and security controls to reduce the attack surface and protect against AD exploitation. This includes enforcing the principle of least privilege, removing legacy authentication protocols, securing administrative accounts, implementing tiered administration, and allowing for monitoring for abnormal activity. Hardening AD is a continuous effort, as attackers frequently target misconfigurations, excessive permissions, and unmonitored objects to gain persistence and escalate privileges.

\runinhead{Active Directory Health Check} An AD health check is an assessment of the operational and security state of an Active Directory environment. It typically involves analyzing replication, DNS configuration, Group Policy application, privileged account usage, and security baselines. Health checks are valuable in identifying vulnerabilities, such as outdated protocols, excessive permissions, or replication errors that could be exploited by attackers. Regular health checks help organizations maintain AD reliability while closing security gaps.

\rininhead{Active Directory Lightweight Directory Services (AD LDS)} AD LDS is a lightweight, standalone directory service that provides LDAP capabilities without requiring a full AD DS deployment. It is often used for directory-enabled applications that need a flexible and isolated directory service separate from the core AD forest. Although AD LDS lacks features like Kerberos-based authentication and Group Policy, it still relies on secure design and configuration to prevent abuse, especially if connected to production AD environments.

\runinhead{Active Directory Migration Tool (ADMT)} The Active Directory Migration Tool (ADMT) is a Microsoft utility to migrate users, groups, and computers between AD domains or forests. It is often used during mergers, acquisitions, or consolidations to unify identity systems. While ADMT simplifies migration, poor planning or weak security during the process can create risks, such as duplicated permissions or lingering access paths that attackers might exploit.
\textit{See also: Active Directory Domain Services (AD DS)}

\runinhead{Active Directory Privilege Escalation (AD Privilege Escalation)} AD privilege escalation refers to the set of techniques that attackers use to gain higher levels of access in an AD environment, often moving from standard user accounts to domain administrator. Methods include Kerberoasting, abusing misconfigured ACLs, leveraging delegation, or exploiting vulnerable services such as AD CS. Privilege escalation is a central stage in the lifecycle of AD attacks, making its prevention and detection a key focus of security teams.

\runinhead{Active Directory Recovery (AD Recovery)} Active Directory recovery is the process of restoring AD services, objects, or configurations after corruption, deletion, or compromise. Unlike forest recovery, this can involve restoring individual objects from the AD Recycle Bin, performing an authoritative restore, or rebuilding specific domain controllers. Effective AD recovery strategies ensure that attackers cannot permanently cripple identity services.

 \runinhead{Active Directory Recycle Bin} The AD Recycle Bin is a feature that allows administrators to recover accidentally deleted objects, such as users, groups, and OUs, without restoring from a backup. Enabled in newer versions of Windows Server, it improves resilience by reducing downtime and administrative overhead in recovery scenarios.

 \runinhead{Active Directory Replication} Replication in AD is the process by which domain controllers synchronize directory data across a forest or a domain. Replication ensures the consistency, accuracy, and completeness of objects, such as users, groups, and policies. Attackers can exploit replication protocols in attacks such as DCSync, which allows them to request password hashes directly from domain controllers. Securing replication traffic and monitoring for anomalous replication requests are critical, yet proactive defenses.

 \runinhead{Active Directory Replication Status Tool (ADREPLSTATUS)} ADREPLSTATUS is a Microsoft tool that helps administrators monitor and troubleshoot replication between domain controllers. Highlights failures, inconsistencies, or delays in AD replication, which, if ignored, can lead to authentication problems and even security gaps. Attackers may take advantage of weak or broken replication to maintain persistence, making ADREPLSTATUS monitoring an important defensive practice.

\runinhead{Active Directory Rights Management Services (AD RMS)} AD RMS is a Windows Server role that provides information protection through encryption and usage policies applied to files, emails, and documents. It integrates with AD identities to enforce who can read, modify, print, or forward protected content. Although largely superseded by Azure Information Protection, AD RMS demonstrates how AD can extend beyond access control to data protection.

\runinhead{Active Directory Risk Assessment} An AD risk assessment evaluates the security posture of an Active Directory environment by analyzing configurations, permissions, account usage, and attack paths. Identifies weaknesses such as stale accounts, weak password policies, excessive group memberships, and misconfigurations that adversaries could exploit. Risk assessments are often the first step to harden AD.
\textit{See also: Active Directory Health Check, Compromise Indicators (IoC), Exposure Indicators (IoE)}

\runinhead{Active Directory Security} Because Active Directory is used to configure network access and permissions, it is a prime target for cyberattackers. Years of growth, mergers, etc. often result in sprawling 'configuration creep' and misconfigurations that leave AD open to attack. Closing security gaps in AD is, therefore, an important part of an organization’s overall cybersecurity strategy. 

\runinhead{Active Directory Security Assessment} An evaluation of the AD environment of an organization to help your organization identify, quantify, and reduce the risks affecting your AD. This analysis generates a list of issues to address and might also offer remediation guidance and best practices to improve the performance or security of the AD infrastructure.
\textit{See also: Active Directory Security Auditing}

\runinhead{Active Directory Security Auditing} The process of collecting data about AD objects and attributes and analyzing and reporting on those data to determine the general health of the directory, the adequacy of system controls, compliance with established security policy and procedures, any breaches in security services, and any changes indicated for countermeasures.

AD security auditing helps you detect and respond to insider threats, privilege misuse, and other indicators of exposure (IOEs) or indicators of compromise (IOCs), thereby strengthening your security posture.
\textit{See also: Active Directory Security Assessment}

\runinhead{Active Directory Service Interface (ADSI)} Active Directory Service Interfaces (ADSI) is a set of COM interfaces that is used to access the features of directory services from different network providers. ADSI is a programmatic interface to AD that enables developers to perform common tasks such as adding new users. Cybercriminals can use ADSI to manipulate directory entries.

\runinhead{Active Directory Service Interfaces Editor (ADSIEdit)} The Active Directory Service Interfaces Editor (ADSIEdit) Microsoft Management Console (MMC) snap-in acts as a low-level editor for Active Directory. ADSIEdit provides access to objects’ properties that are not exposed in other AD interfaces, offering a detailed view of every object and attribute in an AD forest.

\runinhead{Active Directory Sites and Services (AD SS)} Active Directory Sites and Services (ADSS) is a Microsoft Management Console (MMC) plug-in that is used to administer replication of directory data between all sites in an AD forest. Misconfigurations can affect AD performance and also cause replication of faulty security data.

\runinhead{Active Directory Trust (AD Trust)} Active Directory trusts (AD trusts) enable users in an AD domain to access resources in another AD domain. Carefully manage trust relationships to prevent unintended escalation of privileges or exposure of resources.

\runinhead{Active Directory Users and Computers (ADUC)} Active Directory Users and Computers (ADUC) is a Microsoft Management Console (MMC) plug-in that enables administrators to manage user accounts and various other objects in AD. Incorrect usage can lead to unintentional privilege assignments or data exposure.

\runinhead{Active Directory Vulnerability Assessment}
An evaluation of vulnerabilities in the Active Directory environment of your organization can help identify, quantify, and reduce the security and configuration risks to AD. Such analyses generate a list of issues to address and could also offer remediation guidance and best practices to improve the performance or security of the AD infrastructure.
\textit{See also: Active Directory Security Assessment}

\runinhead{Active Directory Web Services (ADWS)}
Active Directory Web Services (ADWS) is a web service hosted on domain controllers running Windows Server 2008 R2 and later. ADWS provides a protocol for accessing and managing directory services using the standard HTTP and HTTPS web protocols.

\runinhead{Active Reconnaissance} A type of information gathering activity in which the attacker (or ethical hacker) interacts directly with the target entity to identify vulnerabilities, open ports, active services and network topology. Unlike passive reconnaissance, this method is easily detectable through intrusion detection systems.

\runinhead{\texttt{Add-ADComputer}}
\texttt{Add-ADComputer} is a PowerShell cmdlet that can be used to create a new computer object in Active Directory. If misused, this cmdlet can lead to the creation of unauthorized computer accounts, potentially used for persistence or lateral movement.

\runinhead{\texttt{Add-ADComputerServiceAccount}}
\texttt{Add-ADComputerServiceAccount} is a PowerShell cmdlet that can be used to add a service account to a computer object in AD. An attacker who compromises this cmdlet can associate service accounts with unintended systems, potentially gaining unauthorized privileges.

\runinhead{\texttt{Add-ADGroupMember}}
\texttt{Add-ADGroupMember} is a PowerShell cmdlet that can be used to add one or more users, groups, service accounts, or computers to an AD group. The misuse of this cmdlet can lead to unauthorized privilege escalation.

\runinhead{\texttt{Add-ADPrincipalGroupMembership}}
\texttt{Add-ADPrincipalGroupMembership} is a PowerShell cmdlet that can be used to add a user, group, service account, or computer to one or more AD groups. If used maliciously, this cmdlet can grant an attacker access to resources.

\runinhead{\texttt{Add-ADUser}} \texttt{Add-ADUser} is a PowerShell cmdlet that can be used to create a new user object in Active Directory.

\runinhead{Address Resolution Protocol (ARP)} Address Resolution Protocol (ARP) is used to map an IP address to a physical (MAC) address on a local network. Although not specific to AD, spoofing ARP responses is a common attack vector in LAN environments.

\runinhead{ADExplorer} Part of the Microsoft Sysinternals Suite, ADExplorer is a legitimate tool that is used to view the Active Directory structure and objects and edit them. Attackers can use this tool to explore AD structures, analyze objects, permissions, and more.

\runinhead{\texttt{ADfind}} \texttt{ADfind} is a command line tool developed by Joe Richard (DS-MVP) to query Active Directory.

\runinhead{Administrative Privileges} These elevated permissions are granted to specific user accounts or roles within the Active Directory environment, enabling them to perform administrative tasks and manage critical resources. Improper management of administrative privileges can lead to unauthorized access and privilege escalation attacks. For example, the SolarWinds attack involved exploiting administrative accounts to move laterally across networks.

\runinhead{Administrative Tiering}
Administrative tiering helps an organization better secure its digital environment by defining three or more layers of access to resources and systems. This layering creates buffer zones that separate the administration of high-risk or valuable assets such as Active Directory domain controllers.

\runinhead{AdminSDHolder} AdminSDHolder is an Active Directory object that holds the security descriptor for objects that are members of privileged groups. The SDProp process ensures that protected objects’ access control lists (ACL) are always consistent with the AdminSDHolder object. A compromised AdminSDHolder object can lead to an SDProp attack.

\runinhead{\texttt{Adprep}} \texttt{Adprep} is a command line tool that is used to prepare a forest or domain for a Windows Server upgrade. Adprep performs necessary schema and infrastructure updates to support the newer version of Windows Server.

\runinhead{\texttt{ADRecon}} \texttt{ADRecon} is a tool that collects information about AD and generates a report that can provide a holistic picture of the current state of the target AD environment. Cybercriminals can use ADRecon for reconnaissance to identify potential vulnerabilities.

\runinhead{Advanced Group Policy Management (AGPM)} Advanced Group Policy Management (AGPM), a feature of Microsoft Desktop Optimization Pack (MDOP), enables improved control and management of Group Policy Objects (GPO). AGPM includes capabilities for version tracking, role-based delegation, and change approval.

\runinhead{Advanced Intrusion Detection Environment (AIDE)} A host-based intrusion detection system that creates a cryptographic baseline of files and directories and periodically checks for unauthorized changes. Often used to ensure file system integrity in Linux or UNIX systems.

\runinhead{Advanced Persistent Threat (APT)} A long-term, stealthy cyberattack campaign typically orchestrated by well-funded and skilled threat actors, often linked to nation states or organized groups. APTs aim to gain unauthorized access to systems and remain undetected for extended periods to steal and exfiltrate sensitive data, over-surveil, or disrupt operations.

\runinhead{Anonymous Logon} The Anonymous Logon security principle allows anonymous access to certain services on a machine. In the context of AD, Anonymous Logon represents connections from users that do not present a valid set of credentials. This can pose a security risk and is often limited or disabled.

\runinhead{Antivirus} Software designed to detect, block, and remove malicious code such as viruses, worms, trojans, and spyware. Modern antivirus solutions often include heuristic analysis, near-real-time protection, and behavior monitoring to detect both known and unknown threats.

\runinhead{Application Directory Partition} Active Directory uses an application directory partition to hold data specific to a certain application or service, such as DNS. If not properly secured, this partition can be exploited by malicious entities for persistence or data extraction.

\runinhead{Application Inventory} A thorough application inventory is a critical first step in ensuring a successful and seamless AD migration. A comprehensive inventory of all applications and services that are integrated with or dependent on the existing AD environment should detail the purpose, users, data flow, dependencies, and infrastructure requirements of each application. The inventory process can involve manual discovery, reviewing documentation, and using automated tools to identify AD-dependent applications and identify potential compatibility issues, integration challenges, and data migration requirements.

\runinhead{Application Layer} Layer 7 of the OSI reference model, responsible for providing services directly to the user or application, such as HTTP, SMTP, and DNS. Many cyberattacks, including SQL Injection (SQLi) and Cross-Site Scripting (XSS), occur at this layer.
\textit{See also: Open Systems Interconnection (OSI) Reference Model}

\runinhead{Application Programming Interface (API)} A set of defined methods and protocols that allow one software application to interact with another. APIs are widely used in web services, cloud platforms, and Active Directory integrations, but can also be abused by attackers if not properly secured.

\runinhead{Arbitrary Commands} Commands that an attacker or ethical hacker can execute on a system, often due to a vulnerability such as command injection or insecure input handling. "Arbitrary" implies that the attacker can run any command, not just a restricted set.

\runinhead{Artificial Intelligence (AI)} The field of computer science that aims to create machines capable of reasoning, learning, and problem-solving. In cybersecurity, AI is used in threat detection, behavioral analytics, and anomaly detection.

\runinhead{Asymmetric Encryption} A type of encryption that uses two mathematically related keys: a public key to encrypt the data and a private key to decrypt it. This method is based on technologies such as SSL/TLS and is used in secure communications, digital signatures, and key-exchange protocols.

\runinhead{Asset} Any component of an information system that has value and requires protection. Assets can include data, systems, software, networks, devices, and even personnel. Asset identification and classification are foundational steps in risk management.

\runinhead{Asset Discovery} The process of identifying and inventorying devices, systems, and applications within a network. This is a fundamental task in cybersecurity, as defenders cannot secure what they do not know exists. Attackers also conduct asset discovery during reconnaissance. Defenders perform asset discovery to obtain full visibility into the devices they are protecting on the network.

\runinhead{Attack Attribution} The process of identifying the entity responsible for a cyberattack, typically involving the analysis of technical indicators (e.g., IPs, malware signatures), behavioral patterns, and threat intelligence. Attribution can be difficult and often inconclusive without corroborating evidence.

\runinhead{Attack Chain} An attack chain is a step-by-step sequence of actions that an attacker follows to compromise, persist in, and potentially exfiltrate data from a target entity. Each phase of the chain represents a distinct stage in the lifecycle of an attack, from initial reconnaissance to achieving final objectives such as data pilfering, data siphoning, sabotage, or persistent control.

\runinhead{Attack Landscape} The overall view of an organization's types of threats, vulnerabilities, and potential attack vectors an organization or sector faces at any given time. This includes current trends in malware strains, APT groups, and evolving TTPs.
\textit{See also: Tactics, Techniques, and Procedures (TTPs)}

\runinhead{Attack Surface} The total set of points in a system or network where an unauthorized user could attempt to enter, extract, or manipulate data. This includes open ports, APIs, services, code paths, and human elements such as employees. Minimizing the attack surface is a key aspect of security hardening.

\runinhead{Attack Vector} A method or pathway through which an attacker gains unauthorized access to a system. Common attack vectors include phishing emails, vulnerable services, open ports, social engineering, and malicious downloads. Basically, an attack vector is a vulnerable entity that an attacker uses as a springboard to further venture into the network.

\runinhead{Attacker Resistance} A measure of how well a system or environment can prevent, detect, or respond to unauthorized intrusion attempts. High resistance from attackers involves layered defenses, hardened configurations, and real-time monitoring.

\runinhead{Attribute-Based Access Control (ABAC)} A flexible access control model that evaluates access requests based on attributes of the user, the resource, the action, and the environment. ABAC enables fine-grained and context-aware access decisions, which are particularly useful in dynamic or large-scale systems.

\runinhead{Attribute} Active Directory objects have attributes, which define the characteristics of the object (for example, user phone number, group name). Manipulating attributes can sometimes lead to unauthorized activities or information disclosure.

\runinhead{Audit Logs} Active Directory generates various audit logs that record activities and events related to authentication, authorization, access to directory services, and more. Properly configuring and analyzing these audit logs is essential to identify potential security breaches, policy violations, or misconfigurations. Regularly reviewing the audit logs can help identify suspicious behavior.

\runinhead{Audit Policy} Audit policies define the types of security events to be recorded in the security log on domain controllers and computers. A poor audit policy might fail to detect intrusion attempts or other malicious activities.

\runinhead{Audit Trails} Audit trails provide a chronological record of activities and events within the AD environment, allowing organizations to reconstruct and investigate security incidents, policy violations, or operational issues. Maintaining audit trails helps with forensic analysis and is a critical component of effective security monitoring and incident response.

\runinhead{Authentication} The authentication process validates the credentials of a person, a computer process, or a device. Active Directory authentication involves proving the identity of a user logging in to an AD environment, and, if compromised, can lead to unauthorized access.

\runinhead{Authoritative Restore} An authoritative restore updates existing domain controllers with restored data, which then replicates to all other DCs in a multi-DC environment.
\textit{See also: Nonauthoritative Restore}

\runinhead{Authentication Server (AS)} A component of the Kerberos authentication protocol, responsible for validating users and issuing a ticket grant Ticket (TGT). The AS resides within the Key Distribution Center (KDC) in an Active Directory environment.

\runinhead{Authorization} The authorization process, which determines the permissions and rights of an authenticated user, follows the authentication process. In AD, authorization is often managed through group memberships. Improper authorization configurations can lead to unauthorized access or privilege escalation.
\textit{See also: Authentication}

\runinhead{Automated Monitoring Tools} Due to the complexity and scale of modern AD environments, automated tools and solutions are invaluable for continuously monitoring, auditing, and reporting on security configurations, permissions, and potential vulnerabilities. Such tools can alert administrators to unusual activities and significantly enhance an organization’s ability to proactively identify and address security risks.

\runinhead{Auxiliary Class} The auxiliary class is an optional class in the AD schema that can be used to extend the attributes of other classes. Misconfiguration of auxiliary classes can lead to security vulnerabilities.

\runinhead{Availability} Availability is one of the three core security principles in the CIA triad-alongside confidentiality and integrity-and ensures that resources are accessible when needed. In an Active Directory context, availability is based on the resilience of domain controllers and the supporting network infrastructure.

\runinhead{Azure Active Directory (Azure AD, now Entra ID)} Microsoft Azure Active Directory (Azure AD or AAD, now renamed Entra ID) is the Microsoft cloud-based \textit{Identity and Access Management (IAM)} service. Azure AD helps organizations manage and secure access to applications, data and networks in both the cloud and on-premises. In the context of cybersecurity, cyberattackers often target Azure AD to gain unauthorized access or escalate privileges, using tactics such as password spray attacks, consent phishing, or exploiting misconfigurations in security policies and access controls; therefore, securing Azure AD is critical.

\runinhead{Azure Active Directory Join} Azure Active Directory join (Azure AD join) is a process that registers a device to a specific Azure AD tenant, allowing the device to be managed and secured through cloud-based policies and services.

\runinhead{Attack Pathway} An attack pathway is a conjoined series of compromised accounts and systems that an attacker follows to gain access to "crown jewel" assets, such as domain administrator accounts or domain controllers.

\runinhead{Attack Surface} The attack surface in cybersecurity is what constitutes the sum of all vulnerable points in an AD environment that an attacker can exploit, leverage or utilize to compromise the network.

\section*{B}
\runinhead{Bell-LaPadula Model} A security model that is focused on data confidentiality, not integrity. It enforces:
\begin{itemize}
    \item "No Read Up" (users cannot read data at higher classifications)
    \item "No Write Down" (users cannot write to lower classifications)
Commonly used in military and government data systems.

\end{itemize}

\runinhead{BloodHound} A popular tool used to visualize AD attack pathways by mapping relationships between users, computers, and groups, helping attackers find the shortest, most efficient route to high-valued targets.

\runinhead{Back Link Attribute} A back link attribute is a type of attribute in the schema of an Active Directory forest. This attribute is linked to a forward link attribute. Together, these are used to create and manage linked attributes.

\runinhead{BackSync} The BackSync process replicates objects and properties back to a global catalog server from a domain controller within the same domain.

\runinhead{Backup and Disaster Recovery Planning (BDRP)} Backup and disaster recovery planning is critical for business continuity and data integrity in the event of system failure, natural disaster, or other disruptive incidents such as a cyberattack. The process involves creating, documenting, and testing strategies and procedures for backing up essential data and restoring them quickly and efficiently after a failure.
\textit{See also: \texttt{NTDS.dit}; Recovery Time Objective (RTO); Recovery Point Objective (RPO)}

\runinhead{\texttt{BadPasswordTime}} The \texttt{BadPasswordTime} attribute of an AD user object records the time of the last bad logon attempt.

\runinhead{\texttt{BadPwdCount}} The \texttt{BadPwdCount} attribute of an AD user object tracks incorrect password attempts. This attribute can be monitored to detect potential brute-force attacks.

\runinhead{Bare Metal Recovery (BMR)} Bare metal recovery (BMR), also known as bare metal recovery, is a type of disaster recovery that restores a computer system to its original state, including the operating system, applications and data, on a new or wiped machine. It is called "bare metal" because it essentially reverts the system back to its foundational state before restoring the full system image. This method is particularly useful for recovering from catastrophic failures, hardware replacements, or situations where a system needs to be restored to a clean state. 
\textit{See also: Active Directory Backup}

\runinhead{Base Distinguished Name (Base DN)} An LDAP search begins with the base DN. This DN can be a potential starting point for an attacker with unauthorized access to explore the Active Directory structure.

\runinhead{Base64} Base64 is a binary-to-text encoding scheme. In the context of Active Directory, certain attributes, like userPhoto, are Base64 encoded.

\runinhead{Best Practice Analyzer (BPA)} Best Practice Analyzer (BPA) is a server management tool that is available in Windows Server. BPA can help an administrator reduce best-practice violations by scanning an AD DS role and reporting when a role is not in compliance with best practices.

\runinhead{Binary} A binary refers to an executable file containing low-level machine instructions (\texttt{0s \& 1s}) that a computer's processor can directly run. Binary file analysis is crucial for binary analysis, a practice used to identify vulnerabilities, detect malware, and understand the structure of programs without needing their original source code. This involves tools and techniques to dissect and model binary code to find malicious patterns or weaknesses that attackers might exploit.

\runinhead{Binary Large Object (BLOB)} A BLOB is a collection of binary data stored as a single entity in a database, including Active Directory. BLOBs are typically images, audio, or other multimedia objects, although sometimes binary executable code may also be stored as a BLOB.

\runinhead{Binding} In the context of Active Directory, binding is the process of setting up a connection to the directory service, which can then be used to carry out operations. If attackers can bind to your Active Directory, they can begin to execute queries and potentially make changes if permissions allow.

\runinhead{BitLocker} BitLocker is a full volume encryption feature included with Microsoft Windows versions from Windows Vista onward. BitLocker is designed to protect data by providing encryption for entire volumes. If an attacker gains physical access to a server, BitLocker can prevent unauthorized access to the data stored within.

\runinhead{BitLocker To Go} BitLocker To Go extends BitLocker data protection to removable drives, such as external hard drives and USB flash drives. These drives can be locked and unlocked only with a password, a smart card, or a recovery key.

\runinhead{BlackCat (ALPHV)} BlackCat/ALPHV is the first high-profile malware written in Rust, a modern cross-platform programming language. Able to compromise Windows and Linux operating systems, BlackCat is operated as \textit{Ransomware as a Service (RaaS)} by ALPHV, a Russian-speaking group of cyber-attackers. It uses compromised user credentials to gain initial access to targeted systems and then leverages that access to further compromise user and administrator accounts in Active Directory (AD). BlackCat attacks often employ a triple-extortion tactic, whereby they make individual ransom demands for decrypting infected files, not publishing stolen data, and not launching denial-of-service (DoS) attacks.

\runinhead{Blacklist} A blacklist is a basic security control in which a list of IP addresses, users, computers, etc. is blocked or denied access. In the context of Active Directory, a blacklist can help protect the directory from known malicious entities.

\runinhead{Blowfish} A fast block cipher designed by the awesome Bruce Schneier. It operates on 64-bit blocks and uses a variable key length (up to 448 bits). Though now largely replaced by AES, it was widely used for encryption in earlier applications and was the de facto for encryption at that time.

\runinhead{BloodHound} BloodHound is an AD reconnaissance tool. BloodHound visualizes AD environments, highlighting relationships that can be exploited for privilege escalation. It is often used in advanced persistent threat (APT) attacks.
\textit{See also: BloodHound Attacks}

\runinhead{BloodHound Attack} An attacker can use BloodHound, a tool that can map AD relationships, to understand the structure of an organization’s AD environment and plan attacks based on the information.
\textit{See also: BloodHound}

\runinhead{Blue Team} In cybersecurity testing, a blue team is the group of individuals responsible for analyzing and securing an information system, identifying its security vulnerabilities and flaws, and defending the environment against potential attackers (e.g., the red team).
\textit{See also: Red Team}

\runinhead{BlueKeep} BlueKeep is a critical security vulnerability identified in Microsoft's Remote Desktop Protocol (RDP) service, officially designated as CVE-2019-0708. It affects primarily older versions of the Windows operating system, including Windows 7, Windows Server 2008, and Windows Server 2008 R2. The vulnerability allows an unauthenticated attacker to execute arbitrary code on a target system by sending specially crafted RDP requests, potentially enabling Remote Code Execution (RCE) without any need for user interaction. Due to the nature of RDP and its widespread use in remote system administration, BlueKeep is particularly dangerous because it can be exploited to create elf-propagating malware or worms, similar to the infamous WannaCry ransomware attack. Microsoft released patches to address BlueKeep in May 2019, even extending updates to unsupported systems such as Windows XP and Windows Server 2003, highlighting the severity and urgency of the threat. Despite patches being available, BlueKeep remains a significant concern because many vulnerable systems remain unpatched, leaving them exposed to potential widespread attacks that could compromise enterprise networks, disrupt services, and result in substantial data breaches or damage.

\runinhead{Bootstrap Replication} Bootstrap replication is a foundational process used in Active Directory Domain Services (AD DS) to initialize and synchronize new domain controllers (DCs) within an existing domain. When a new DC is introduced, it must obtain an initial copy of the directory data to become fully functional and consistent with the rest of the domain. Bootstrap replication involves the new DC contacting an existing DC using special Lightweight Directory Access Protocol (LDAP) requests to pull a complete snapshot of the directory database replica, including all objects, attributes, and security settings, effectively "bootstrapping" itself into the network. This initial replication is crucial because it establishes the baseline state of the domain controller, allowing us to participate fully in further incremental replication, authentication, and directory services. Without bootstrap replication, a new DC would lack the essential directory data required to process logon requests, enforce group policy, and maintain the integrity of the domain's security model. This process is automatic and transparent, but must be properly secured to prevent unauthorized access during the initial synchronization. In general, bootstrap replication ensures seamless integration of new domain controllers, promotes fault tolerance, and supports the scalability of Active Directory environments.

\runinhead{Bridgehead Server} A bridgehead server is the point of contact for replication between sites in an Active Directory forest. If an attacker compromises a bridgehead server, they can potentially manipulate the replication data.

\runinhead{Bring Your Own Device (BYOD)} BYOD refers to employees who bring their own computing devices - such as smartphones, laptops, and PDAs - to work for use and connectivity. BYOD can pose security challenges for Active Directory if not properly managed and controlled.

\runinhead{Browse} The process of viewing objects within Active Directory. An attacker who can browse your directory can start mapping out the structure and details of your AD environment.

\runinhead{Brute-Force} Brute-Force refers to a method of systematically trying all possible combinations of characters to guess passwords, encryption keys, or other security credentials. It is a trial-and-error approach that relies on automated tools to test a wide range of possibilities until the correct one is found.

\runinhead{Bug Bounty} A program offered by organization's that rewards individuals (usually ethical hackers) for discovering and responsibly disclosing software vulnerabilities. It helps improve software security through community collaboration.

\runinhead{Built-In} The term built-in refers to default groups and user accounts that are automatically created when you install Active Directory Domain Services (AD DS). These groups and accounts have default permissions and rights that are assigned, so it is important to review these defaults and ensure that they align with your organization’s security policies.

\runinhead{Built-In Container} A built-in container is a special Active Directory container that exists in the security context of the local domain controller. This container holds groups that are local to the domain controller and is created by default when AD is installed.

\runinhead{Bulk Import} A process by which large amounts of data can be imported into Active Directory, often using tools like CSVDE. If an attacker can manipulate this process, they can potentially create numerous malicious entries in AD.

\runinhead{Business Continuity Plan (BCP)} A Business Continuity Plan (BCP) is a document that outlines how a business will continue operating during an unplanned service interruption. In the context of Active Directory, a BCP can include plans for how to restore service after a major outage or attack.

\runinhead{Business Impact Analysis (BIA)} A Business Impact Analysis (BIA) is a systematic process to determine the potential consequences of a disruption to critical business operations of an organization. Identifies the most important business functions, assesses the impact of possible disruptions, and helps prioritize recovery efforts. In essence, a BIA helps companies understand what could go wrong and how best to prepare for and recover from those events. 

\section*{C}
\runinhead{Cain \& Abel} Cain \& Abel is a password recovery utility designed for Microsoft Windows. Facilitates the recovery of various passwords employing techniques such as network packet sniffing, dictionary attacks, brute-force attacks, and cryptanalysis.

\runinhead{Canonical Name (CNAME)} In Active Directory, a canonical name refers to the DNS-style path used to identify and reference directory objects. It is often used in scripting or programmatic administration tasks.

\runinhead{Capture the Flag (CTF)} A competitive cybersecurity exercise where participants solve challenges across various categories (e.g., cryptography, reverse engineering, web exploitation) to simulate real-world attack and defense scenarios.

\runinhead{Central Access Policy (CAP)} A Central Access Policy (CAP) in Active Directory is part of\textit{ Dynamic Access Control (DAC)}. It allows administrators to define and enforce access rules across multiple file servers within a domain, based on user claims and resource properties.

PGP, TOR, deep web, dark web, mariana's trench, anonymization, SOCKS/SOCKS5, SSL tunneling, proxy server, proxy relay, IP obfuscator, burner email, burner cell, clearing tracks, cloaking, active and passive reconnaissance, threat hunting, threat modeling, attack tree, heat graphs, denial of service, distributed denial of service, flooding attacks, TCP SYN, NULL, IDLE, XMAS FIN scans, mal intent, MiTM, man in the browser, incidnet response, information assurance, threat detection, EDR, XDR, behavioral analytics, anomaly detection, passive network scanning, rogue machine

 

\runinhead{Certificate Revocation List (CRL)} A Certificate Revocation List (CRL) is a list of digital certificates that have been invalidated by the Certificate Authority (CA) prior to their expiration. Maintaining an up-to-date CRL is critical in preserving secure communication within an Active Directory environment.

\runinhead{Certificate Services} Certificate Services is a server role in Windows Server that provides Public Key Infrastructure (PKI) functionality. It enables organizations to issue, manage, and revoke digital certificates used for authentication and encryption.

\runinhead{Chain of Custody} A legal and procedural process that ensures that the evidence (digital or physical) has been preserved, handled, and transferred without tampering or modification of the original preserved data. Essential in digital forensics, it ensures the integrity and admissibility of data in legal or investigative contexts.

\runinhead{Checksum} A checksum is a value computed from a data set (such as a file or packet) for the purpose of detecting errors that may have been introduced during storage or transmission. Checksums are generated using hash or mathematical algorithms and are commonly used in data validation, file integrity monitoring, and network communications. In an ethical hacking context, checksums help detect tampering or corruption by comparing a newly calculated checksum with the original. If the two values differ, it signals that the data have changed either due to error or a potential attack.

\runinhead{Child Domain (Subdomain)} A child domain is a subordinate domain within a hierarchical Active Directory forest. It inherits the namespace of its parent domain and automatically establishes a transitive two-way trust relationship with it.

\runinhead{Claim Type} Claim type represents an aspect of a user’s identity, such as group membership, and is used in Dynamic Access Control (DAC) for authorization decisions. Misconfigured claim types can lead to privilege escalation or unauthorized access.

\runinhead{Claims-based Authentication} Claims-based authentication is a process in which a user obtains a digitally signed token from a trusted source and presents it to a system. The system can then validate the token and use the claims inside it (e.g., username, role) to identify the user.

\runinhead{Client} Users run applications on computer workstations, also known as client machines. If a workstation is connected to a network, then users can take advantage of services provided by servers. Client machines typically do not store data locally, but rather receive the requested data from servers by running client-server applications.

\runinhead{Cloud Services} Cloud services are available over the Internet from a cloud computing provider. Although not specific to Active Directory, many organizations use cloud services such as Azure Active Directory in conjunction with or as an alternative to their on-premises AD.

\runinhead{Colonial Pipeline Attack} The 2021 Colonial Pipeline ransomware attack is one of the most well-known critical infrastructure attacks in recent history. The Colonial Pipeline attack demonstrated the importance of maintaining a robust Active Directory security posture.

\runinhead{Comma Separated Value (CSV)} A CSV (Comma Separated Values) file is a plain text file format used to store tabular data, such as spreadsheets or databases. Each row in the file represents a record, and the values within each row are separated by commas, making it a simple way to store and exchange data between different applications. 

\runinhead{Comma Separated Value Data Exchange (CSVDE)} CSVDE is a tool for importing and exporting Active Directory data. Comma-separated value (CSV) files can be manipulated, leading to wrong data import or export if not validated properly, which can pose security threats.

\runinhead{Common Internet File System (CIFS)}  Pronounced as \textit{"SIFS,"} CIFS is a network file-sharing protocol. It is essentially a specific version (or dialect) of the \textit{Server Message Block (SMB)} protocol developed by Microsoft. CIFS allows computers to share files and resources like printers over a network, making it easier for different systems to access and work with the same data.

\runinhead{Common Name (CN)} A common name (CN) is the name of an object in Active Directory and must be unique within its container. The CN is part of the object’s distinguished name (DN), which uniquely identifies an object in the LDAP directory. 

\runinhead{Common Vulnerabilities and Exposures (CVE)} A CVE, or Common Vulnerabilities and Exposures, is a standardized identifier assigned to publicly known cybersecurity vulnerabilities and exposures, serving as a universal reference that enables consistent vulnerability tracking, discussion, and remediation across security communities. Managed by MITRE Corporation and sponsored by the US Department of Homeland Security (DHS), the CVE system catalogs unique vulnerabilities found in software, hardware, and firmware by assigning and cataloging each a distinct CVE identification number, such as "CVE-2025-0708." This system facilitates communication among security researchers, vendors, and organizations by providing a clear, common, and understood vocabulary to describe security flaws. When a vulnerability is discovered and verified, it is submitted to the CVE program, which evaluates, assigns the CVE ID, and publishes the details including descriptions, hashes, signatures, and relevant metadata without disclosing exploit specifics. CVEs are essential for vulnerability management processes, helping defenders prioritize patches, perform risk assessments, and automate security tools such as intrusion detection systems and vulnerability scanners by referencing these standardized identifiers.

\runinhead{Common Weakness Enumeration (CWE)} CWE (Common Weakness Enumeration) is a comprehensive catalog of software and hardware weaknesses, essentially providing a detailed taxonomy of the root causes or design flaws that lead to security vulnerabilities. Managed by MITRE, CWE helps developers and developers, security professionals, and organizations better understand the underlying faults in code or systems that can be exploited. For example, common CWEs include buffer overflows, SQL injection (SQLi) flaws, and Cross-Site Scripting (XSS) vulnerabilities. By identifying and categorizing these weaknesses, CWE provides guidance for secure coding practices, risk mitigation, and security testing. It is widely used in the Secure Development Lifecycle (SDLC) processes, vulnerability analysis, and educational resources to promote awareness of fundamental security flaws and how to best avoid them.

\runinhead{Compact Directory Database} This maintenance operation reduces the size of the database file (\texttt{NTDS.DIT}). This operation requires high-privilege access and, if misused, can lead to Denial of Service (DoS) attacks.

\runinhead{Computer Object} In Active Directory, a computer object is a representation of a computer that is part of the domain. Contains various attributes about the computer, such as its name, security settings, and association with user accounts or groups.

\runinhead{Conditional Expression} In the context of Active Directory, a conditional expression can refer to conditional statements in a Group Policy Object (GPO) or within a script or tool used for AD management.

\runinhead{Configuration Container} The Configuration container in Active Directory contains information on the logical structure of the forest, including details on sites, services, and directory partitions. This data is replicated to all domain controllers in a forest. From a cybersecurity perspective, unauthorized changes to the Configuration container could lead to replication issues, impact network performance, or alter the behavior of services relying on this information. Therefore, access to modify the Configuration container should be strictly controlled and monitored.

\runinhead{Container} In Active Directory, a container is an object that can store other AD objects such as user accounts, groups, and even other organizational units (OUs). Containers cannot have group policies applied to them. OUs are also containers and can contain the same objects, plus other OUs, and can have group policies applied.

\runinhead{Context Menu} A context menu in a graphical user interface (GUI) appears in the user interaction, such as a mouse click operation. In Active Directory Users and Computers (ADUC), the context menus offer various options such as resetting passwords, moving objects, or initiating replication.

\runinhead{Create, Read, Update, and Delete (CRUD) Operations} Create, Read, Update, and Delete (CRUD) operations are the fundamental functions performed in any database system, including Active Directory.

\runinhead{Credential Roaming} Credential roaming is a feature of Active Directory that enables user credentials and certificates to be copied and transferred securely across multiple devices. The roaming of credentials helps manage digital identities in different systems.

\runinhead{Credentials} The credentials are the user name and password that a user provides to authenticate. If credentials are not properly secured, they can be targeted in credential stuffing or brute-force attacks.

\runinhead{Cross-Forest Trust} A cross-forest trust relationship can be created between two Active Directory forests. This relationship enables users in one forest to access resources in the other forest, expanding collaboration while maintaining security boundaries.

\runinhead{Cross-Reference Object} A cross-reference object is an object in the configuration partition that associates a naming context with a directory server. An attacker compromising this can cause replication issues and lead to outdated security data.

\runinhead{Cross-Site Scripting (XSS)} Cross-site scripting, or XSS, is a type of security vulnerability that is not specific to Active Directory but can potentially affect any web-based interface used for AD administration if the interface does not properly validate the input.

\runinhead{Cryptography} Cryptography is the practice and study of techniques for secure communication. In Active Directory, cryptography is used in multiple places, including secure LDAP, Kerberos authentication, and Encrypting File System (EFS).

\runinhead{Cyber Kill Chain (CKC)} The Cyber Kill Chain (CKC) is a framework that outlines the steps of a cyberattack. It is generally considered to have seven steps:

1. Reconnaissance
2. Weaponization
3. Delivery
4. Exploitation
5. Installation
6. Command and Control (C2)
7. Action on Objectives

Within hybrid and multi-cloud environments, a vigilant defender supports integrity and availability of critical enterprise directory services at every step in the cyber kill chain.

\runinhead{Cyber Warfare} Cyber warfare is a series of cyberattacks on critical computer systems of a country, state, or organization. One of the most infamous examples is NotPetya, a malware that originated in Russia in 2017 and targeted Ukraine and quickly spread around the world with devastating effects.

\runinhead{Cyberattack} A cyberattack is a malicious attempt to gain unauthorized access to computer information system resources with the purpose of stealing, altering, exposing, and destroying data or disrupting operations. Identity systems such as Active Directory are prime targets for cyberattackers.

\runinhead{Cybersecurity and Infrastructure Security Agency (CISA)} The Cybersecurity and Infrastructure Security Agency (CISA) is an agency of the Department of Homeland Security (DHS) of the United States that is responsible for strengthening cybersecurity and infrastructure against threats.

\runinhead{Credential Compromise} The initial phase of most AD attacks, where attackers steal user or service account credentials through secondary attacks such as phishing, social engineering, malware, or other means to gain initial access to the network.

\runinhead{Cyclic Redundancy Check (CRC)} A Cyclic Redundancy Check (CRC) is a noncryptographic hash function used primarily to detect accidental changes in raw data. It works by treating the data as a polynomial and dividing it by a predetermined polynomial; the remainder becomes the CRC value.

\section{D}
\runinhead{DACL Backdoor Attack} A discretionary access control list (DACL) backdoor attack involves an attacker adding an entry to an object's DACL. This grants the attacker certain permissions or rights to that object without requiring them to compromise an account with those rights.

\runinghead{Data Breach} A cyberattack that occurs with the purpose of stealing or exposing confidential, sensitive, or protected information to an unauthorized person.

\runinhead{DCShadow Attack} In a DCShadow attack, an adversary modifies the Active Directory schema by registering a rogue domain controller. The attacker can then propagate malicious replication changes to the actual domain controllers.

\runinhead{DCSync} DCSync attacks leverage the Active Directory replication feature, using Directory Replication Services (DRS) to impersonate and request password data from a domain controller. The attack can be used to effectively "pull" password hashes from a DC without having to run code on the DC itself. This type of attack is adept at bypassing traditional auditing and detection methods.

\runinhead{Defense in Depth-} Defense in depth is a cybersecurity strategy that uses multiple layered security controls throughout a network to protect against cyberattacks. The thought process behind this approach is to make it harder for attackers to succeed by ensuring that if one layer is breached, additional defenses are still in place. Each layer addresses different types of threats and breach scenarios, creating a system of security redundancies that improves overall security resilience.
\textit{See also: Layered Defense}

\runinhead{Delegation} Delegation in Active Directory refers to the process of assigning specific administrative permissions or control over certain objects or tasks to other users or groups. This allows responsibilities to be distributed without granting full administrative privileges. The thought process behind delegation is to improve operational efficiencies and security by applying the Principle of Least Privilege (PoLP), users get only the access they absolutely need to perform their roles, reducing the risk of accidental or malicious changes.

\runinhead{Delegation Attack} Kerberos delegation is a feature that allows a service to impersonate a user to other services. If not properly configured, an attacker can exploit them to escalate privileges or move laterally through the network.

\runinhead{Diffie-Hellman (DH) Key Exchange} A cryptographic protocol that allows two parties to securely establish a shared secret over an insecure channel. It is the basis for secure session key generation in many secure communication protocols. Vulnerable to Man-in-the-Middle (MiTM) attacks if not authenticated.

\runinhead{Directory Vulnerabilities} Active Directory forests often contain multiple security risks, ranging from management errors to unpatched vulnerabilities. With access to AD or Azure AD, threat actors can gain dominance over your entire infrastructure. Cyberattackers target AD to increase privileges and gain persistence in the organization. To defend AD, administrators must know how attackers are targeting the environment and which vulnerabilities they might exploit.
\textit{See also: Active Directory Privilege Escalation; Indicators of Compromise (IoC); Indicators of Exposure (IoE)}

\runinhead{DNSAdmins} Members of the DnsAdmins group have access to network DNS information. This group exists only if the DNS server role is installed or is installed on a domain controller in the domain. Attackers who gain access to this group can use that access to compromise Active Directory.

\runinhead{DNSAdmins Abuse} An attacker with membership in the DNSAdmins group can load an arbitrary DLL into the DNS service, which runs with system-level privileges, thus achieving privilege escalation.

\runinhead{DNS Attack} Active Directory is heavily based on DNS for the resolution of the name and location of the service. DNS attacks, such as DNS spoofing or DNS poisoning, can redirect or manipulate DNS requests, leading to unauthorized access or disruption of AD services.

\runinhead{Domain} A domain is a logical grouping of users, computers, and other network resources that share a common directory database and administrative structure. It represents a fundamental unit within an Active Directory environment that defines the boundaries of management, security, and replication.  

\runinhead{Domain Controller (DC)} A domain controller is a server that manages access to network resources within a specific domain, primarily handling user authentication and security. It acts as a gatekeeper, verifying user credentials and controlling which network resources users can access. DCs are essential for maintaining network security and centralized management of user accounts and permissions.

\runinhead{Domain Dominance} During a cyberattack, threat actors often seek access to Active Directory. Such access can enable attackers to eventually gain administrative privileges and ultimate power over Active Directory domains, and thus all application and data that rely on Active Directory.
\textit{See also: Active Directory Attack Lifecycle; Cyber Kill Chain}

\runinhead{Domain Naming Master} The Domain Naming Master is a FSMO (Flexible Single Master of Operations) role responsible for managing the forest-wide domain namespace. Specifically, it handles the addition and removal of domains within the forest and prevents naming conflicts when creating new domains. There is only one Domain Naming Master per forest.

\runinhead{Domain Tree} A domain tree is a hierarchical structure of domains that share a common namespace and are linked by trust relationships. Think of it as a family tree, where the "root" domain is the parent, and any domains branching off from it are considered child domains, all sharing the same naming structure.

\runinhead{Domain Trust Abuse} In an environment with multiple trusted domains, an attacker with admin access to a lower trust level domain can leverage the trust relationship to gain access to a higher trust level domain.

\runinhead{Dumping LSASS Memory} Local Security Authority Subsystem Service (LSASS) stores credentials in memory that can be dumped and extracted by an attacker. Tools such as Mimikatz are often used for this purpose.

\runinhead{Dynamic Link Library (DLL) Injection Chains}  DLL (Dynamic Link Library) injection chains involve a malicious actor who exploits the way Windows loads DLLs to execute arbitrary code within a target process. By injecting a malicious DLL, attackers can evade security measures, increase privileges, or perform other harmful actions. This technique is particularly effective because it takes advantage of the fact that many legitimate applications rely on DLLs, making it easier for malicious code to blend in with normal system activity.

\section*{E}
\runinhead{Effective Permissions} Effective permissions are a set of permissions granted to a user or group based on a combination of explicit and inherited permissions. Understanding effective permissions is critical for security auditing and risk assessments.

\runinhead{Elevated Privileges} Elevated privileges are higher-level permissions, typically administrative privileges, that are granted to a user account. An attacker who gains elevated privileges can cause significant damage or data breaches.

\runinhead{Empire} A PowerShell and Python post-exploitation framework, Empire offers a range of tools for exploiting Windows systems. Among its capabilities are features for collecting credentials, creating backdoors, and establishing persistence in an AD environment.

\runinhead{\texttt{Enable-ADAccount}} This PowerShell cmdlet is used to enable a disabled user account in Active Directory. Misuse can reactivate previously disabled malicious accounts.

\runinhead{Encrypting File System (EFS)} This Windows feature enables transparent encryption and decryption of files by using advanced standard cryptographic algorithms. Although EFS can improve data security, it should be properly managed to avoid unauthorized access or loss of data.

\runinhead{Encryption} Encryption is the process of converting data into a coded form to prevent unauthorized access. AD uses encryption in various forms for secure communication, such as Kerberos tickets or LDAPS connections.

\runinhead{Endpoint Protection} Endpoint protection is the practice of securing endpoints or entry points of end-user devices such as desktops, laptops, and mobile devices from being exploited by malicious actors and campaigns. If not properly managed, infected endpoints can compromise AD security.

\runinhead{Enterprise Administrators (EA)} This high-level group in AD has full control over all assets within the entire forest. The Enterprise Admins group is a high-value target for attackers, as compromise of an Enterprise Admins account can lead to complete domain takeover.

\runinhead{Enterprise Directory Services} A common directory, such as Microsoft Active Directory, enables a more secure environment for directory users and common expectations of the role the directory can provide to both users and applications. A common enterprise directory resource facilitates role-based access to computing resources.

\runinhead{Enterprise Mobility Management (EMM)} EMM is a set of services and technologies designed to protect corporate data on mobile devices of employees. EMM is used in conjunction with AD for identity and access management.

\runinhead{Enumeration} Enumeration is the process of extracting detailed information about objects within AD. Uncontrolled enumeration can lead to information disclosure that could aid an attacker.

\runinhead{Escalation of Privilege} Escalation of privilege (or the escalation of privilege) is a type of network intrusion that takes advantage of programming errors or design flaws to grant the intruder elevated access to the network and its associated data and applications. In an AD context, an attacker who can leverage misconfigurations or vulnerabilities to escalate their privileges can potentially gain full control over the domain.

\runinhead{Ethernet} Ethernet is a family of computer networking technologies commonly used in local area networks (LAN), metropolitan area networks (MAN), and wide area networks (WAN). Ethernet was commercially introduced in 1980 and has since been refined to support higher bit rates and longer link distances. Today, Ethernet is the most widely installed local area network technology. Ethernet cables, such as Cat 5e and Cat 6, are commonly used in wired networks. The latest versions of Ethernet can support data transfer rates up to 400 gigabits per second.

\runinhead{Event Logs} Event logs are records of significant incidents in an operating system or other software. In the context of AD, monitoring event logs can help detect security incidents or problematic configurations; however, some attacks are designed to evade event logs.

\runinhead{Event Viewer} This Microsoft Management Console (MMC) snap-in provides a view of the event logs in Windows. Administrators use Event Viewer to monitor, manage and troubleshoot issues within AD, and the tool is crucial to identifying signs of potential cyberattacks.

\runinhead{Exchange Server} Exchange Server is the Microsoft email, calendaring, contact, scheduling and collaboration platform deployed on the Windows Server operating system for use within a business or larger enterprise environment. Exchange Server interacts with AD for user information and authentication.

\runinhead{Explicit Group Membership} Explicit group membership occurs when a user or group is directly added to an AD group rather than gaining membership through nested groups. Understanding explicit and implicit (nested) group memberships is important for managing permissions and access controls.

\runinhead{Export Policy} This Group Policy setting enables the export of user and computer settings. Enabling this capability can be a security concern if not properly controlled, as it can lead to the exposure of sensitive configuration information.

\runinhead{Extended Protection for Authentication (EPA)} EPA is a security feature that enhances the protection and handling of authentication credentials when they are transmitted over the network. This technology is designed to counter man-in-the-middle (MiTM) attacks, which steal or manipulate credentials during transmission. EPA can improve the security of protocols used for communication and data exchange. For example, when used with LDAP, EPA can prevent attacks such as NTLM relaying.

\runinhead{Extended Rights} Extended rights are a set of non-standard permissions that can be granted to a security principal. Extended rights provide specific control access rights to the object to which they are applied. Misconfiguring extended rights can lead to security vulnerabilities.

\runinhead{Extended Schema}
An extended schema is an AD schema that is modified or extended with additional attributes or classes, typically to support third-party applications; however, improper modifications can lead to functionality issues or security vulnerabilities.

\runinhead{Extensible Markup Language (XML)}
XML is a markup language that defines a set of rules for encoding documents in a format that is both human-readable and machine-readable. In the context of Active Directory, XML can be used in many ways, such as creating custom scripts for specific operations, defining Group Policy settings, or formatting data reports.

\runinhead{Extensible Storage Engine (ESE)} This Jet-based ISAM data storage technology (previously known as Jet Blue) is used in Active Directory and Exchange Server. The ESE database engine enables fast and efficient data storage and retrieval using indexed and sequential access.

\runinhead{External Trust} An external trust is a type of trust in Active Directory that is manually defined and does not extend beyond two domains. Security risks can arise from improperly configured trusts, as they can enable unauthorized access across domains.

\runinhead{Extranet} An extranet is a controlled private network that allows partners, vendors, and suppliers, or an authorized set of customers, to access a subset of information accessible from an organization’s intranet. In relation to AD, proper authentication and authorization are essential to secure extranet resources.

\section*{F}
\runinhead{Failback} Failback is the process of restoring a system or other component of a system to its original state after a failure.

\runinhead{Failover} In Active Directory, failover refers to the process by which network services are moved to a standby server in case of a primary server failure. It is a crucial part of ensuring high availability.

\runinhead{Failover Clustering} Failover clustering is a Windows Server technology that enables you to create and manage failover clusters that provide high availability for network services and applications.

\runinhead{File and Printer Sharing} This network feature allows a computer to share data files and connected printers with other computers and devices on the network.

\runinhead{File Integrity Monitoring (FIM)} File Integrity Monitoring (FIM) is a security technique used to detect unauthorized or unexpected changes to critical system files, configuration files, or application binaries. FIM solutions work by creating cryptographic checksums (such as hashes) of monitored files and comparing them over time to detect modifications, deletions, or additions, and to create a baseline that displays current patterns and trends within internal network activities.

\runinhead{File Replication Service (FRS)} FRS is a Microsoft Windows Server service for distributing shared files and Group Policy Objects (GPO). FRS has been replaced by Distributed File System Replication (DFSR) in newer versions of Windows Server.

\runinhead{File System Security} File system security refers to the access controls and permissions assigned to files and directories. In an Active Directory context, file system security often refers to permissions set via Group Policy Objects.

\runinhead{Filtered Attribute Set (FAS)} FAS attributes are not replicated in Read-Only Domain Controllers (RODCs).

\runinhead{Filtering} In the context of Active Directory, filtering is used to limit the objects or attributes on which a replication or query operation acts. Improper filtering can lead to inefficient replication or inaccurate query results, affecting performance, and possibly leading to incorrect data.

\runinhead{Fine-Grained Password Policies} In Windows Server 2008 and later, these policies enable you to specify multiple password policies within a single domain. In this way, you can apply different restrictions for password and account lockout policies to different sets of users in your domain.

\runinhead{Fingerprinting Organizations with Collected Archives (FOCA) FOCA is a tool that is used to find metadata and hidden information in documents. FOCA can be used to extract information from public files hosted on the company website, providing attackers with information on the internal structure of an AD environment.

\runinhead{Firewall} This network security device monitors incoming and outgoing network traffic and decides whether to allow or block specific traffic based on a defined set of security rules.

\runinhead{Firewall Configuration} The settings and rules that determine how your firewall will manage inbound and outbound traffic. Misconfigurations can leave ports open for attackers to exploit, making the firewall a critical aspect of network security.

\runinhead{Firewall Exceptions} Firewall exceptions are configurations that allow specific network traffic to bypass security controls, often necessary for certain applications or services to function properly in a network.

\runinhead{Firewall Rules} The policies that govern how a firewall operates. These rules can define the types of traffic allowed or blocked by the firewall and where that traffic is allowed to go. Proper configuration and management of firewall rules are crucial to maintaining network security.

\runinhead{Flat Name} A flat name is the NetBIOS name of the domain and can be different from the domain’s DNS name.

\runinhead{Flexible Single Master Operations (FSMO) Role Seizure} The process of forcibly transferring FSMO roles from a non-operational domain controller to a functioning domain controller within an Active Directory domain. FSMO seizure is an emergency recovery process in which an Active Directory domain controller forcefully takes over an FSMO role from another domain controller that is malfunctioning or permanently offline. This is typically a last resort measure, as seizing an FSMO role can lead to data inconsistencies in the directory service if the original role holder becomes available again.

\runinhead{Flexible Single Master Operations (FSMO) Role Transfer} The process of transferring FSMO roles from one domain controller to another. FSMO role transfer is usually a planned process, as opposed to FSMO role seizure, which is typically an emergency process. FSMO transfer needs to be managed securely to prevent an attacker from taking control of these crucial roles.

\runinhead{Flexible Single Master Operations (FSMO) Roles} FSMO roles are special roles assigned to one or more domain controllers in an Active Directory environment. These roles manage operations that can be performed by only one DC at a time. The roles help to ensure consistency and to eliminate the potential for conflicting updates in an Active Directory environment; however, improper management or a failure of a server with one of these roles can lead to disruptions in the AD environment.

\runinhead{Forced Change Password Attack} In this type of attack, an attacker forces a user or service account to change its password. The attacker captures the new password hash as it is transmitted to the domain controller, then uses it to authenticate as the user or service account.

\runinhead{Foreign Security Principal}} An object that represents a security principal (such as a user or security group) located in a trusted domain external to the forest. These objects enable external security principals to become members of security groups within the domain.

\runinhead{Forest} In Active Directory, a forest is a collection of one or more domain trees, each with a different DNS namespace. All domain trees in a forest share a common schema and configuration container. When you first install Active Directory, the act of creating the first domain also creates a forest. Forests serve as the topmost logical container in an Active Directory configuration, encapsulating domains.

\runinhead{Forest Druid} Forest Druid is a security tool that identifies and prioritizes attack paths that lead to Tier 0 assets. The tool helps cybersecurity defensive teams quickly prioritize high-risk misconfigurations that could represent opportunities for attackers to gain privileged domain access. Rather than chasing down every avenue, defenders can use Forest Druid to quickly identify undesired or unexpected attack paths for remediation, accelerating the process of closing backdoors into Active Directory.

\runinhead{Forest Root Domain} The forest root domain is the first domain created in the forest. This domain contains some special features and is crucial for the functioning of the entire AD forest. The forest root domain cannot be removed.

\runinhead{Forest Trust} A forest trust relationship is established between two Active Directory forests. A forest trust allows users in different forests to access resources in a reciprocal manner, subject to the configured permissions. Misconfiguring these trusts can expose resources to unauthorized users, potentially leading to data breaches.

\runinhead{Forward Link} A forward link in Active Directory is a type of link attribute that points from one object to another. When the forward link is modified, the system automatically updates the link table for the back link attribute. For example, the member attribute of a group object is a forward link pointing to the users that are members of the group, whereas the memberOf attribute is the related backlink.

\runinhead{Forward Lookup Zone} A forward lookup zone is a part of the DNS server in Active Directory that is used to translate domain names into IP addresses. If not properly secured, this could potentially be exploited by attackers to gain unauthorized access or launch a DNS poisoning attack.

\runinhead{FSMO Role Seizure} The act of forcibly transferring FSMO (Flexible Single Master Operations) roles from one Domain Controller to another. This is typically done when the original Domain Controller is no longer available, and it should be used as a last resort, as it can potentially lead to issues within the domain. An improper seizure can alter AD functionality and introduce security issues.

\runinhead{Full System Restore} The process of returning a computer system to its original state, usually using a full system backup in the event of critical system failure or corruption.

\runinhead{Fully Qualified Domain Name (FQDN)} The FQDN is the complete domain name for a specific computer or host on the Internet. The FQDN consists of two parts: the hostname and the domain name. In the case of Active Directory, the FQDN is used to precisely identify the location of an object within the directory.

\runinhead{Functional Level} In Active Directory, the functional level determines the available AD DS domain or forest capabilities. It also determines which Windows Server operating systems you can run on domain controllers in the domain or forest; however, once the functional level is raised, domain controllers running earlier versions of Windows Server cannot be introduced into the domain or forest.

\runinhead{Flexible Single Master Operations (FSMO} Pronounced "FIZZ-MO," FSMO stands for Flexible Single Master Operations. These are specialized roles assigned to individual domain controllers that are responsible for specific tasks within the Active Directory forest or domain. There are five FSMO roles, two operating at the forest level and three at the domain level. They ensure the consistency and integrity of Active Directory by designating a single controller to perform certain critical operations. 

\runinhead{Forest} In Active Directory (AD), a forest is the highest-level container that acts as the overarching structure for managing network resources. Essentially, it essentially groups one or more domain trees, sharing a common schema and global catalog. Think of it as the top-level organizational unit that defines the scope of the directory service.

\runinhead{Fully Qualified Domain Name (FQDN)} A Fully Qualified Domain Name (FQDN) is the complete and unambiguous address of a device or service on the Internet. Specifies the exact location of a resource within the Domain Name System (DNS) hierarchy. Essentially, it is a domain name that includes all the necessary levels, from the host to the top-level domain (TLD). 

\runinhead{Functional Levels} Active Directory (AD) functional levels determine the features and capabilities available within a domain or forest. They essentially set the compatibility mode and define which AD features are accessible. There are two types of functional levels: Domain Functional Level and Forest Functional Level. 

\section*{G}
\runinhead{\texttt{Get-ADAccountResultantPasswordReplicationPolicy}} This PowerShell command retrieves the resultant password replication policy for an AD account. An attacker can potentially use this information to understand which passwords are being replicated and where, aiding in attack planning.

\runinhead{\texttt{Get-ADDomainController}} This PowerShell cmdlet retrieves a domain controller object or performs a search to retrieve multiple domain controller objects from AD. Inappropriately used, it can provide an attacker with valuable information about the domain controller in an AD environment.

\runinhead{\texttt{Get-ADFineGrainedPasswordPolicy}} This PowerShell cmdlet retrieves fine-grained password policies from the AD. If these policies are incorrectly configured or leaked, it could help an attacker plan a password cracking attack.

\runinhead{\texttt{Get-ADGroup}} This PowerShell cmdlet retrieves a group object or performs a search to retrieve multiple group objects from AD. Misuse or inappropriate exposure can provide an attacker with valuable information about group structure and membership in an AD environment.

\runinhead{\texttt{Get-ADGroupMember}} This PowerShell command retrieves the members of an AD group. An attacker could use this to identify high-privilege accounts to target.

\runinhead{\texttt{Get-ADObject}} This PowerShell command retrieves an AD object or performs a search to retrieve multiple objects. It is commonly used in reconnaissance by an attacker to understand objects within AD.

\runinhead{\texttt{Get-ADReplicationAttributeMetadata}} This PowerShell cmdlet retrieves the attribute replication metadata for AD objects, which can be used to troubleshoot replication issues; however, in the hands of an attacker, it could potentially reveal sensitive information.

\runinhead{\texttt{Get-ADRootDSE}} This PowerShell command retrieves the root of the Directory Information Tree (DIT) of an AD domain. This can be used by attackers to gather information about the domain structure.

\runinhead{\texttt{Get-ADTrust}} This PowerShell cmdlet retrieves a trust object or performs a search to retrieve multiple trust objects from AD. Inappropriately used, it can provide an attacker with valuable information about trust relationships in an AD environment.

\runinhead{\texttt{Get-ADUser}} This PowerShell cmdlet retrieves a user object or performs a search to retrieve multiple user objects from AD. Inappropriately used or exposed, it can provide an attacker with valuable information about user accounts in an AD environment and identify potential targets for attacks within the AD.

\runinhead{Global Address List (GAL)} The GAL is an accessible directory of all users, groups, shared contacts, and resources recorded in the Active Directory Domain Services (AD DS) of an organization. Inappropriate access or manipulation of the GAL can lead to unauthorized access to information or phishing attacks.

\runinhead{Global Catalog (GC)} The Global Catalog is a distributed data repository that contains a partial searchable representation of every object in every domain in a multidomain Active Directory Domain Services (AD DS) forest.  GC is used to speed up searches and logins, especially in large environments. If a GC server becomes unavailable or compromised, it can cause problems with log-ins and searches.

\runinhead{Global Groups} Global groups can have members from their own domain, but can be granted permissions in any domain in the forest. If used incorrectly, these groups can lead to unwanted privilege escalation.

\runinhead{Globally Unique Identifier (GUID)} A unique reference number used in programming, created by the system to uniquely identify an AD object. A GUID in Active Directory is a 128-bit number that is used to uniquely identify objects. Every object created in an Active Directory gets a GUID which remains the same for the life of the object, even if the object is moved or renamed. Manipulation of GUIDs can potentially lead to attacks like object impersonation. 

\runinhead{Golden gMSA Attack} A Golden gMSA attack is a cyberattack in which attackers dump the root keys of the Key Distribution Service (KDS) and generate passwords for all associated gMSAs offline. This two-step process begins with the attacker retrieving several attributes from the KDS root key in the domain. Then, using the Golden gMSA tool, the attacker generates the password of any gMSA that is associated with the key (without having a privileged account).

\runinhead{Golden Ticket Attack} A Golden Ticket attack enables an attacker to forge a Kerberos ticket, giving them unauthorized access to any system in the domain as a highly privileged user, such as a domain administrator. Such elevated privileges can give the attacker almost unlimited access to Active Directory and the resources that depend on it. 

\runinhead{\texttt{GPUpdate}} This command-line tool on Windows operating systems forces an immediate refresh of Group Policy on the local machine. This tool can be useful for immediately applying policy changes, rather than waiting for the automatic refresh cycle.

\runinhead{Granular Audit Policies} Granular audit policies can be configured in AD for more detailed information gathering. Misconfigurations can lead to gaps in monitoring and logging, potentially enabling attackers to avoid detection.

\runinhead{Group Identifier (GID)} This unique value identifies a specific group in an AD environment. In a UNIX context, the GID is often used to map UNIX groups to their Windows counterparts. This capability can be manipulated for access control bypass attacks in mixed-OS environments.

\runinhead{Group Managed Service Account (gMSA)} A Group Managed Service Account (gMSA) is a managed domain account that helps secure services on multiple servers. Introduced in Windows Server 2012, gMSA is a special type of service account in Active Directory and features automatic password rotation every 30 days. It also provides simplified service principal name (SPN) management and the ability to delegate management to other administrators. If compromised, GMSAs can be used to escalate privileges or move laterally across a network.
\textit{See also: Golden gMSA attack}

\runinhead{Group Managed Service Account (gMSA) Passwords} Passwords for Group Managed Service Accounts (gMSAs) are managed by AD. These accounts, if compromised, can allow an attacker to move laterally across a network or escalate privileges.

\runinhead{Group Membership} In Active Directory, users are grouped together to simplify the process of granting permissions or delegating control. Incorrect group memberships can give a user more access rights than necessary, following the principle of least privilege can reduce the risk.

\runinhead{Group Nesting} Group nesting refers to the practice of adding groups as members of other groups. Although nesting can simplify permission management, it can also create complex and hard-to-track permissions structures, potentially leading to excessive permissions and security issues.

\runinhead{Group Policy Management Console (GPMC)} This Microsoft Management Console (MMC) Plugin provides a single administrative interface to manage Group Policy across the enterprise in an Active Directory environment. GPMC simplifies the management of Group Policy by making it easier to understand, deploy, and manage policy implementations.

\runinhead{Group Policy Modeling} This planning and troubleshooting tool for group policies can simulate the potential impact of GPOs, but misuse or misunderstanding of its results can lead to misconfigurations.

\runinhead{Group Policy Object (GPO)} Group policies enable IT administrators to implement specific configurations for users and computers. Group Policy settings are contained in Group Policy Objects (GPOs), which are linked to Active Directory Domain Services (AD DS) containers. A GPO is a component of a Group Policy, used to represent policy settings applied to users or computers. GPOs can become a target for attackers who want to alter security settings at the system level.
\textit{See also: Group Policy Object (GPO) Abuse}

\runinhead{Group Policy Object (GPO) Abuse} Attackers with permission to modify GPOs can take advantage of this ability to execute malicious code, modify system settings, or disrupt system operations on systems where the GPO applies.

\runinhead{Group Policy Preferences (GPP)} Part of the group policy, GPP enables a more advanced configuration of systems. GPP is notable for a security issue: It used to store passwords in a reversible encrypted format, a vulnerability that has been exploited in the past.

\runinhead{Group Policy Preferences (GPP) Password Attack} Before a Microsoft update removed the Group Policy Preferences (GPP) feature, GPP allowed administrators to store passwords in Group Policy Objects (GPO). The encrypted passwords could easily be decrypted, and older GPOs could still contain these deprecated password entries, making them a target for attackers.

\runinhead{Group Policy Results} A report of Group Policy settings within the scope of an object (user or computer). This report can be valuable for troubleshooting, but can also expose potential weaknesses or misconfigurations in GPOs to attackers.

\runinhead{Group Policy Security} Group Policy is an integral feature built into Microsoft Active Directory. Its core purpose is to enable IT administrators to centrally manage users and computers in an AD domain. This includes both business users and privileged users such as IT administrators, and workstations, servers, domain controllers (DCs), and other machines. Group policy security is an important part of AD security.

\runinhead{Group Scopes (Global, Universal, Domain Local)} Group scopes define the reach of AD groups in terms of their ability to include other groups or users as their members, and the extent to which these groups can be granted permissions. Misconfiguration of group scopes can lead to unauthorized access to resources.

\runinhead{Golden Ticket} A Golden Ticket attack is a compromise of the Kerberos Key Distribution Center (KDC) password (also known as the krbtgt password) to generate counterfeit service tickets, allowing attackers to impersonate any user and gain full domain access.

\runinhead{Group Policy Object (GPO)} GPO, or Group Policy Object, is a feature of Microsoft Windows that is used to centrally manage and configure operating systems, applications, and user settings within an Active Directory environment. It allows administrators to enforce specific security settings for users and computers, enhancing network security and efficiency.

\section*{H}
\runinhead{Hardening} Hardening an AD environment involves securing the environment against attacks by reducing the surface of vulnerability. This might include measures such as implementing least-privilege access, monitoring suspicious activity, regularly updating and patching systems, etc.

\runinhead{Hashing} In the context of Active Directory, hashing refers to the way passwords are stored. AD uses a hashing algorithm to store passwords in a nonreversible hashed format, enhancing security; however, attackers can still use techniques like pass-the-hash attacks to exploit these hashed credentials.

\runinhead{Hidden Recipient} A hidden recipient in Active Directory is a user that does not appear on the address lists. If hidden recipients are not properly managed, an attacker could use them to exfiltrate data without raising alarms.

\runinhead{Hierarchy} AD structure is built as a hierarchy, starting from forests to domains, organizational units, and individual objects. An understanding of this hierarchy is crucial for both managing AD and securing it against potential attacks.

\runinhead{Home Directory} In AD, the home directory is a specific network location that is automatically connected each time a user logs in. If not properly secured, these directories can be exploited by attackers to gain unauthorized access to sensitive data.

\runinhead{Honeypot Account} In cybersecurity, a honeypot account is a decoy AD account used to attract and detect malicious activities. If the honeypot account is accessed or altered, it could be an indication of a security breach.

\runinhead{Host} A host is a computer connected to a network.

\runinhead{Host (A) Record} In Active Directory, a host record (A) maps a domain name to an IP address in DNS. If these records are not properly secured, attackers could manipulate them, causing traffic to be redirected to malicious sites.

\runinhead{Host Header} In the context of Active Directory Federation Services (AD FS), a host header is used to route incoming HTTP/HTTPS requests that are sent to a specific AD FS federation server on a farm.

\runinhead{Host-Based Intrusion Detection System (HIDS)} A HIDS is a system that monitors a computer system, rather than a network, for malicious activities or policy violations. Implementing a HIDS on critical AD servers can help detect and prevent potential attacks.

\runinhead{Hotfix} A hotfix is a single, cumulative package that includes information (often in the form of files) that is used to address a problem in a software product like Active Directory. From a cybersecurity perspective, the regular application of hotfixes is essential to protect against known vulnerabilities.

\runinhead{Hybrid Active Directory} A Hybrid Active Directory environment integrates on-premises AD with cloud-based solutions like Azure AD (now Entra ID). This setup enables users to have a single identity for both systems. From a cybersecurity perspective, managing access and identities across on-premises and cloud environments can be complex and requires a comprehensive security approach.

\runinhead{Hybrid Cloud Deployment} In a hybrid cloud deployment, Active Directory might serve to authenticate and authorize users and computers in a network that combines the on-premises infrastructure and cloud services. Security measures must be taken into account in both environments.

\runinhead{Hybrid Identity Protection (HIP)} Many organizations today use both in-premises Active Directory and in-the-cloud Azure AD. This hybrid identity environment enables a common user and system identity for the authentication and authorization of resources regardless of location; however, it also presents unique cybersecurity challenges.

\runinhead{Hydra} A popular brute-force tool, Hydra supports numerous protocols, including SMB and HTTP, which are often used in AD environments. Hydra can be used to guess or crack passwords, allowing unauthorized access to user accounts.

\runinhead{Hypertext Transfer Protocol Secure (HTTP)} HTTPS is often used in AD Federation Services (ADFS) to protect communications. It is important to keep certificates up to date and use strong encryption protocols to maintain security.

\runinhead{Hash / Hashing} A hash is the output of a hash function, which is a mathematical operation that converts data of any size into a fixed-size string of characters, also known as a hash value or hash code. This process is called hashing. Hashing is a one-way function, which means that it is computationally infeasible to reverse the process and recover the original data from the hash.

\section*{I}
\runinhead{Identity and Access Management (IAM)} IAM is a framework of policies and technologies that the right people in an enterprise have the appropriate access to technology resources. IAM systems can be used to initiate, capture, record, and manage user identities and their related access permissions.

\runinhead{Identity Attack} From phishing emails to cyberattacks targeting Active Directory, threat actors love to target identity resources. If a cyber-attacker can gain the identity credentials of a user (for example, through a phishing email), they do not need to enter your environment; they can simply log in. Once inside your environment, the attacker can attempt to take over additional identities, working their way up (through privilege escalation) to admin-level access. At that point, the attacker can make changes to Active Directory to take over, lock or shut down user and system accounts, resources, and data.

\runinhead{Identity Management (IdM)} IdM is a broad administrative area that involves identifying individuals in a system (such as a country, a network, or an enterprise) and controlling their access to resources within that system by associating user rights and restrictions with the established identity.

\runinhead{Identity Provider (IdP)} IdP is a system that creates, maintains and manages identity information for principals and provides principal authentication to other service providers within a federation, such as with AD Federation Services (ADFS).

\runinhead{Identity Threat Detection and Response (ITDR)} Identity systems are under sustained attack. The misuse of credentials is now a primary method that cyber-attackers use to access systems and achieve their goals.

\runinhead{ImmutableID} This attribute in AD links an on-premises user to an Office 365 user. ImmutableID is often used during AD migrations or consolidations.

\runinhead{Impacket} Impacket is a collection of Python classes developed for working with network protocols, often used for creating network tools. Provides a robust and comprehensive framework for crafting and decoding network packets, allowing developers to construct and analyze network traffic. Although Impacket is an important tool for legitimate network administrators and cybersecurity professionals, malicious actors can also exploit it for network attacks, such as NTLM relay attacks on Active Directory.
\textit{See also: NTLM Relay Attack}

\runinhead{Impersonation} Impersonation refers to the ability of a thread to execute in a security context that is different from the context of the process that owns the thread. In a cybersecurity context, impersonation is a common attack method that could lead to unauthorized access or escalation of privileges.

\runinhead{Implicit Identity} Special identities that represent different users at different times, depending on the circumstances. For example: Anonymous Logon, Batch, Authenticated User and more.

\runinhead{Indexed Sequential Access Method (ISAM)} A method of indexing data for fast retrieval used by the Extensible Storage Engine (ESE) used in Active Directory.

\runinhead{Indicators of Attack (IOAs)} Indicators of attack (IOAs) in cybersecurity are security indicators that demonstrate the intent of a cyberattack. Detecting IOAs early in an attack can help defenders prevent further damage.
\textit{See also: Security Indicators; Indicators of Compromise (IOC); Indicators of Exposure (IOE)}

\runinhead{Indicators of Compromise (IOCs)} Indicators of compromise (IOCs) in cybersecurity are security indicators that demonstrate that the security of the network has been breached. Investigators typically spot IOCs after being informed of a suspicious incident, discovering unusual callouts from the network, or during a security assessment.
\textit{See also: Security Indicators; Indicators of Attack (IOA); Indicators of exposure (IOE)}

\runinhead{Indicators of Exposure (IOE)} Indicators of exposure (IOEs) are security indicators that provide information on potential exploitable vulnerabilities before a cybersecurity incident occurs. By understanding these risks, security teams can better prioritize security management efforts and be prepared to contain attacks quickly.
\textit{See also: Security Indicators; Indicators of Attack (IOA); Indicators of compromise (IOC)}

\runinhead{Information Rights Management (IRM)} IRM is a form of IT security technology used to protect information from unauthorized access. In the context of Active Directory, IRM can help protect sensitive data by controlling who has access to it and what they can do with it, such as preventing data from being printed or forwarded.

\runinhead{Information Security Policy} A set of policies issued by an organization to ensure that all IT users within the organization’s domain abide by its rules and guidelines related to information security. Policies are designed to protect organizational data and manage the risk of compromising the confidentiality, integrity and availability of information.

\runinhead{Infrastructure Master} One of the five FSMO roles in AD, the Infrastructure Master is responsible for updating references from objects in its domain to objects in other domains. If all domain controllers are also Global Catalog servers, the infrastructure master role performs no tasks.
\textit{See also: Flexible Single Master Operations (FSMO) Roles}

\runinhead{Inheritance} Inheritance refers to the cascading of permissions from parent objects to child objects within the Active Directory tree. In AD, permissions granted at a higher level in the hierarchy can be inherited by lower-level objects, unless inheritance is explicitly blocked. Inheritance simplifies permission management, but incorrect configurations could expose resources to unauthorized users. 

\runinhead{Insecure AD Replication} Unauthorized domain controllers or domain controller compromise can lead to replication of directory services data to a malicious actor, enabling them to gather sensitive information and credentials.

\runinhead{Install From Media (IFM)} This feature enables administrators to install a domain controller using restored backup files. Using the Install from Media (IFM) option, you can minimize the replication of directory data over the network. This helps you to install additional domain controllers in remote sites more efficiently, especially when the WAN links to these sites are relatively slow and/or the existing AD database size is considerably large.

\runinhead{\texttt{Install-ADDSDomainController}} This PowerShell command installs a new domain controller in AD.

\runinhead{\texttt{Install-ADDSForest}} This PowerShell command installs a new AD DS forest. This is a highly privileged command that, if misused, can lead to the creation of a malicious forest, potentially compromising the entire AD environment.

\runinhead{Integrated DNS} Refers to a DNS that is integrated with an Active Directory domain. An AD-integrated DNS server stores its data in Active Directory. This allows DNS information to be replicated to all other domain controllers in the domain, improving the fault tolerance of your DNS.

\runinhead{Inter-Forest Trust} A trust established between two Active Directory forests. An interforest trust can be a one-way or two-way trust that provides for controlled access to resources in each forest. Managing and monitoring interforest trusts to mitigate the risk of unauthorized access is crucial.

\runinhead{Inter-Site Topology Generator (ISTG)} In Active Directory, the ISTG role is held by one domain controller in each site and is responsible for creating a spanning tree of all site links in the site and constructing a least-cost routing topology for replication between domain controllers within the site.

\runinhead{Interactive Logon} This type of login occurs when a user enters their credentials directly into the system, typically through the system console. In Active Directory, interactive log-ins are logged as a specific event (Event ID 528 on Windows Server 2003 and older, and Event ID 4624 on Windows Server 2008 and newer).

\runinhead{Internet} The Internet is a global network of computers and servers that communicate with each other using standardized protocols, primarily TCP/IP (Transmission Control Protocol/Internet Protocol). The Internet provides various services including the World Wide Web, email, file transfer, and cloud services. In terms of cybersecurity, the Internet is often the main vector for a wide range of threats targeting Active Directory environments, including phishing attacks, malware distribution, and remote exploitation of vulnerabilities. Thus, securing Internet connections and monitoring Internet-facing services are crucial tasks in network security.

\runinhead{Internet Authentication Server (IAS)} IAS is the Microsoft implementation of a Remote Authentication Dial-In User Service (RADIUS) server and proxy in Windows Server 2000 and 2003. IAS performs centralized connection authentication, authorization, and accounting for many types of network access, including wireless and VPN connections. From a cybersecurity perspective, securing the IAS is crucial, as attackers who compromise it could manipulate authentication processes, gain unauthorized network access, or snoop on network traffic. Since Windows Server 2008, IAS has been replaced by Network Policy Server (NPS).

\runinhead{Internet Information Services (IIS)} IIS is a web server software created by Microsoft for use with the Windows NT family. IIS supports HTTP, HTTPS, FTP, FTPS, SMTP, and NNTP. In the context of AD, IIS is often used for hosting necessary web-based services such as ADFS.

\runinhead{Internet Protocol Security (IPSec)} IPsec is a suite of protocols that secures IP communications by authenticating and encrypting each IP packet in a data stream. In terms of Active Directory, IPsec policies can be used to provide security for traffic between AD domain controllers and member servers or clients, thereby adding an extra layer of security.

\runinhead{Intranet} An intranet is a private network within an organization. Intranets are often used to share information and computing resources between employees. Security-wise, uncontrolled or unauthorized access to the intranet can lead to information leakage or other forms of internal attacks.

\runinhead{Intrusion Detection System (IDS)} A device or software application that monitors a network or systems for malicious activity or policy violations. An IDS plays a crucial role in a robust security architecture.

\runinhead{Inventory (hardware/software)} A hardware or software inventory typically refers to the process of collecting detailed information about all the hardware or software used in an organization. An accurate inventory is essential for managing resources, planning for future needs, and maintaining security.

\runinhead{IP Addressing} An IP address is a numerical label assigned to each device that participates in a computer network that uses the Internet Protocol for communication. In an Active Directory environment, proper IP addressing is crucial for network communication and resource access.

\runinhead{IP Security Policy} This series of rules dictates which form of IPsec should be used in a transaction between the server and the client. The misconfiguration of the rules could leave security vulnerabilities in your AD environment.

\runinhead{Isolated Network Segment} A section of a network isolated from the rest of the network. Using isolated network segments can limit the potential damage if a security incident occurs in a different segment.

\runinhead{IT Infrastructure Library (ITIL)} This set of detailed practices for IT service management (ITSM) focuses on aligning IT services with the business needs.

\runinhead{Infrastructure Master} The Infrastructure Master is one of the five Flexible Single Master Operations (FSMO) roles, specifically responsible for maintaining accurate cross-domain object references. This role is crucial in multi-domain environments where objects such as users and groups reside in different domains but need to interact with each side in different domains but need to interact with each other. The Infrastructure Master updates these references when objects are moved, renamed, or when group memberships change across domains.

\section*{J}
\runinhead{Jet Database Engine (Joint Engine Technology)} The Active Directory database is based on the Microsoft Jet Blue engine and uses the Extensible Storage Engine (ESE) to store, edit, delete, and read data. The Active Directory database is a single file named ntds.dit. By default, this database is stored in the \texttt{\%SYSTEMROOT\%NTDS} folder on each domain controller and is replicated between them.

\runinhead{John the Ripper} A fast password cracker, used to detect weak passwords. Attackers use John the Ripper to crack hashed passwords, enabling unauthorized access.

\runinhead{Joining a Domain} An operation where a computer becomes part of an Active Directory domain. Joining a domain allows the system to use the central authentication provided by AD, access resources, and adhere to the policies set by the domain. Mistakes in this process can cause vulnerabilities and improper access controls.

\runinhead{Just Enough Administration (JEA)} This security technology enables delegated administration for anything managed by PowerShell. In an AD context, JEA can help limit privilege escalation attacks by reducing the number of people who have full administrative rights.
\textit{See also: Just-in-Time (JIT) Administration}

\runinhead{Just-in-Time (JIT) Administration} This method of assigning privileges to users is similar to Just Enough Administration (JEA). JIT gives users the privilege they need to perform a task, but only for a certain period of time. This can minimize the risk of privilege escalation or credential theft.
\textit{See also: Just-in-Time (JIT) Administration}

\section*{K}
\runinhead{Kerberoasting} Kerberoasting targets the weakness in the Kerberos authentication protocol used by Active Directory. Attackers request a service ticket for a targeted service account and then crack the encrypted service ticket offline to obtain the password of the account.
\textit{See also: Kerberos; Kerberos Delegation Abuse; Kerberos Password Guessing}

\runinhead{Kerberoasting with Rubeus} Rubeus is a powerful tool for interacting with the Microsoft Kerberos protocol. In Kerberoasting attacks, attackers use Rubeus to request service tickets and crack the tickets offline to gain service account credentials.

\runinhead{Kerberos} Kerberos is the primary authentication method used in Active Directory domains to authenticate users and computers. Older operating systems support DES encryption, while Windows Server 2008 and later support AES encryption. Kerberos is prone to several types of attacks, such as Golden Ticket and Silver Ticket attacks, that exploit the way Kerberos tickets are created and used within an AD environment.

The Kerberos computer network security protocol manages authentication and authorization in Active Directory. Massachusetts Institute of Technology (MIT), which created Kerberos, describes it as using strong cryptography to enable a client to prove its identity to a server on an unsecured network connection. After the client and server use Kerberos to prove their identities, they can also encrypt their communications to ensure privacy and data integrity. Two decades ago, the Kerberos protocol was a game changer in terms of security, unification, and moving AD toward identity management. But the evolution of attack methods and cloud migration has made Kerberos increasingly vulnerable to cyber threats.
\textit{See also: Kerberos Delegation Abuse; Kerberos Password Guessing; Kerberoasting}

\runinhead{Kerberos Constrained Delegation (KCD)} This security feature in Active Directory allows a service to impersonate a user to access a different service. The feature is designed to reduce the number of users with excessive privileges; however, misconfigurations can lead to the ability of attackers to elevate privileges or bypass authentication systems.
\textit{See also: Kerberos Delegation Abuse}

\runinhead{Kerberos Delegation Abuse} Attackers can manipulate constrained, constrained, and resource-based delegation to impersonate other users or elevate privileges within the domain. This abuse takes advantage of the complexities and implicit trust of the Kerberos protocol.
\textit{See also: Kerberoasting; Kerberos; Kerberos Constrained Delegation (KCD); Kerberos Password Guessing}

\runinhead{Kerberos Password Guessing (AS-REP Roasting)} In this attack, an adversary targets user accounts that do not require Kerberos pre-authentication. The attacker attempts to authenticate with the Key Distribution Center (KDC) and receives an encrypted ticket-granting ticket (TGT) that contains the user’s hashed password, which can then be cracked offline.
\textit{See also: Kerberos; Kerberos Delegation Abuse; Kerberoasting}

\runinhead{Kerberos Policy} Kerberos policy defines ticket properties for all domain users, such as ticket lifetime and renewal. This policy is part of Group Policy and, if not properly configured, can allow threat actors to replay old Kerberos tickets to gain unauthorized access.

\runinhead{Kerberos Service Principal Name (SPN)} SPN is used in Active Directory to associate a service instance with a service login account. SPNs can be a target for certain types of attack, such as Kerberoasting, where an attacker uses a valid Kerberos ticket to request service ticket data, which can then be forced offline to reveal the plaintext password of the service account.

\runinhead{Kerbrute} Kerbrute is a tool designed to perform Kerberos pre-auth brute-forcing. It can be used to validate whether usernames exist within an Active Directory environment without the risk of account lockouts.

\runinhead{Key Distribution Center (KDC)} In the Kerberos protocol, the KDC is responsible for authenticating users and providing ticket-granting tickets (TGTs), which are then used to obtain service tickets for various resources in the network. A compromised KDC could have severe implications, as it can lead to the compromise of any user or service in the domain.

\runinhead{\texttt{klist}} This command-line utility lists Kerberos tickets of the user who runs the command. The tool is useful for troubleshooting Kerberos authentication issues.

\runinhead{Knowledge Consistency Checker (KCC)} This Active Directory service generates the replication topology for the Active Directory replication system. If KCC fails or is compromised, it can lead to inconsistencies in the directory data.

\runinhead{Known Secure State} A known secure state represents the state of an environment that is confirmed to not contain any malware or ransomware. Returning to a known secure state after a cyberattack helps prevent loss of confidentiality, integrity, or availability of information.

\runinhead{\texttt{ksetup}} This command-line utility is used to configure a computer that is not joined to a domain to use domain resources. The tool is often used to configure a machine to use Kerberos for authentication in nontraditional scenarios.

\runinhead{Kerberoasting} An Active Directory cyberattack targeting the Kerberos authentication protocol ticketing system. Target service accounts with weak passwords by requesting and cracking their Kerberos tickets to gain unauthorized access.

\runinhead{Kerberos}  Kerberos is a network authentication protocol that uses cryptography to verify user identities and secure communications between clients and servers. It is designed to prevent passwords from being transmitted over the network in plaintext and provides a secure way for users to access resources on a network safely.

\runinhead{Key Distribution Center (KDC)}A Key Distribution Center (KDC) is a crucial component in secure network authentication systems like Kerberos. It acts as a trusted third party that manages and distributes cryptographic keys, enabling secure communications between users and services. The KDC is responsible for authenticating users, managing permissions, and issuing tickets or session keys, or tokens, to access specific resources.

\runinhead{\texttt{krbtgt}}  In Active Directory, \texttt{krbtgt} stands for \textit{"Kerberos Ticket-Granting Ticket."} It is a built-in user account that acts as a service account for the Kerberos Key Distribution Center (KDC). Its password is used to encrypt and sign all Kerberos tickets within the domain, making it a critical component for Kerberos authentication. Compromising the \texttt{krbtgt} account's password can lead to a Golden Ticket attack, allowing an attacker to essentially forge Kerberos tickets and gain unauthorized access.

\section*{L}
\runinhead{L0phtCrack} L0phtCrack is a password audit that helps automate the recovery of passwords from hashes, helping attackers to break into systems by cracking the user’s password.

\runinhead{LastKnownParent} This attribute identifies the last known location of a moved or deleted AD object. If not properly monitored, it can aid in object lifecycle tracking and potentially object restoration, posing a security risk.

\runinhead{LastLogonTimeStamp} This attribute specifies the last time the user has logged in. Irregularities in this attribute might indicate a potential unauthorized access or pass-the-hash attack.

\runinhead{Lateral Movement} Lateral movement occurs when a cyberattacker uses compromised accounts to gain access to additional clients and accounts throughout an organizations network. Cyberattackers use lateral movement in combination with privilege escalation to identify and gain access to sensitive accounts and resources that share stored sign-in credentials in accounts, groups, and machines. A typical goal of successful lateral movement is eventual administrative access to Active Directory domain controllers.
\textit{See also: Domain Dominance; Least Privilege; Privileged Access; Privilege Escalation}

\runinhead{Layered Defense} A layered defense is one that applies multiple layers of protection (e.g., endpoint security, SIEM, and Active Directory security) to help ensure that a cyberattacker who penetrates one layer of defense will be stopped by a subsequent layer.
\textit{See also: Defense in Depth}

\runinhead{LDAP Channel Binding} The process of linking the transport layer and the application layer, creating a cohesive unit. In terms of LDAP channel binding, the LDAP application layer is essentially intertwined with the TLS tunnel. This tight interconnection creates a distinct and unique identifier, or fingerprint, for the LDAP communication, thus any intercepted LDAP communications cannot be reused by attackers.

\runinhead{LDAP Connection Objects} Logical, one-way connections from one domain controller to another for replication purposes. If compromised, LDAP connection objects can be exploited to gain unauthorized control over replication.

\runinhead{LDAP Data Interchange Format (LDIF)} A standard plain text data interchange format. Represents directory content as records for update requests in Active Directory. Used by the LDIFDE command-line utility.
\textit{See also: LDAP Data Interchange Format Directory Exchange (LDIFDE)}

\runinhead{LDAP Data Interchange Format Directory Exchange (LDIFDE)} A Microsoft utility that can be used to import/export AD objects to/from LDIF files. Misuse can lead to unauthorized data export/import.

\runinhead{LDAP Directory Probe (\texttt{\texttt{ldp.exe}})} LDAP Directory Probe is a Windows Support Tool graphical utility that admins use to run LDAP operations against AD. The misuse of this tool can expose sensitive information or alter AD objects.

\runinhead{LDAP Injection Attack} In this type of attack, an attacker manipulates the input fields to insert and execute the light-weight Directory Access Protocol (LDAP) commands. The attacker uses these commands to query and manipulate data stored on an LDAP server, often used in conjunction with Active Directory.

\runinhead{LDAP over SSL (LDAPS)} An extension of LDAP that encrypts LDAP traffic. If not properly configured, LDAPS can make traffic susceptible to interception.

\runinhead{LDAP Policies} Policies that define the behavior of an LDAP server. The misconfiguration of these policies can lead to performance issues and potential security risks.

\runinhead{LDAP Referrals} When an LDAP server cannot answer a query, the server refers the client to another server. In a referral attack, a client can be referred to a malicious server.

\runinhead{LDAP Search Filters} LDAP search filters are used to find and manipulate AD objects. Improper use can lead to unauthorized access or alteration of directory objects.

\runinhead{LDAP Signing}  LDAP signing refers to the process in which LDAP traffic is digitally signed at its source. This digital signature serves to ensure that content within the LDAP traffic remains unchanged during transit, preserving its authenticity and integrity. Moreover, it provides a means for the receiver to confirm the original source of the LDAP traffic. The setup for LDAP signing can be achieved either through tailored group policies or by manipulating registry keys. Turning off LDAP signing may make the network susceptible to man-in-the-middle attacks.

\runinhead{\texttt{Ldapdomaindump}} This tool is used to dump domains using LDAP. The tool provides an attacker with easy access to all sorts of useful information about the domain.

\runinhead{LDAPMiner} LDAPMiner is a tool used for LDAP data extraction, primarily intended for penetration testing and other security audits. Attackers can use this tool to query and gather data from the AD environment, which assists in reconnaissance.

\runinhead{Least Privilege} Sometimes called minimum privilege, the information security principle of least privilege emphasizes that users and applications should be given privileged access only to the data and operations they require to perform their jobs. By taking this approach, IT and security teams can help prevent potential lateral movement in their organization’s networks.
\textit{See also: Domain Dominance; Lateral Movement; Privileged Access; Privilege Escalation}

\runinhead{Least-Privilege User Access (LUA)} A principle of limiting user rights to the minimum bare minimum permissions that they need to perform their work. Not following LUA can open avenues for privilege-escalation attacks.

\runinhead{LegacyExchangeDN} A unique identifier for each object within Active Directory (AD). Manipulation of this identifier could lead to unauthorized access, which is a major cybersecurity concern.

\runinhead{Lightweight Directory Access Protocol (LDAP)}
\textit{Pronounced}: \textbf{/ˈɛl.dæp/} or \textbf{“L-dap”}
LDAP is a protocol used to access and manage directory information services over a network. In Active Directory environments, LDAP enables applications and services to query and authenticate against the directory to retrieve information such as users, groups, and Organizational Units. LDAP provides a standardized and efficient way to interact with directory-based identity systems, supporting functions like authentication, authorization, and directory browsing. It is a foundational protocol for identity and access management in enterprise environments.

\runinhead{Lingering Objects} Lingering objects can occur if a domain controller does not replicate for an interval of time longer than the lifetime of the tombstone (TSL) and then reconnects to the replication topology. These are objects that remain in the directory database after being deleted on other domain controllers and cause inconsistencies and potential security issues if not handled properly.

\runinhead{Link Table} In Active Directory, a link table is a database table that keeps track of linked multi-valued attributes. These include attributes that create a relationship between two AD objects, such as member and memberOf attributes that link users to groups.

\runinhead{Link Identifier} An attribute in the AD schema that uniquely identifies an attribute of an object. If compromised, this attribute can lead to data inconsistencies and potential privilege escalation.

\runinhead{Link-Value Replication} An Active Directory mechanism that allows incremental updates to multivalued attributes. If compromised, this replication can lead to data inconsistency and potentially propagate false information within the directory.

\runinhead{Linked Attribute} Attributes in AD that have a corresponding attribute in another object, such as member and memberOf. An improper configuration can create orphaned links which might confuse the replication process, affecting the consistency of AD data.

\runinhead{Linked Value Replication (LVR)} A feature of Active Directory, introduced with Windows Server 2003, which enables individual updates to multi-valued attributes instead of replicating the entire set of values. For example, adding a new member to a large group only replicates the addition of the new user, not the entire group membership list. When a nonlinked multivalued attribute is updated, the entire attribute must be replicated. Requires Windows Server 2003 interim mode or Windows Server 2003 Forest Functional Level or higher. If manipulated, it can lead to replication errors or unauthorized changes.

\runinhead{Local Administrator Password Solution (LAPS)} A Microsoft tool that helps organizations manage local administrator passwords for computers associated with the domain. It helps mitigate the risk of a pass-the-hash attack by randomly generating and storing a different password for each machine’s local administrator account in Active Directory. In terms of cybersecurity, the implementation of LAPS can significantly improve an organization’s security posture by limiting the possibilities of lateral movement for attackers who have gained access to local administrator credentials on a machine.

\runinhead{Local Groups} Groups that exist on a local machine. If an attacker gains control of a local group, they can change permissions and gain additional privileges.

\runinhead{Local Policies} Sets of rules defined on a local machine that dictate how that specific system behaves. If not configured properly, these can provide a loophole for security breaches.

\runinhead{Local Security Authority (LSA)} The LSA is responsible for local security policy and user authentication. Cyberattacks often target LSA secrets, as they contain sensitive security data and credentials.

\runinhead{Local Security Authority Subsystem Service (LSASS)}
\textit{Pronounced:} \textbf{/ˈɛl.sæs/} or \textbf{“L-sass”}
The Local Security Authority Subsystem Service is a critical Windows process responsible for managing the enforcement of security policies, user authentication, and account management. It plays a vital role in verifying user logins, managing passwords, and generating access tokens. Adversaries often target LSASS because it stores sensitive information, such as credentials, in memory, making it a prime target for attacks such as credential dumping.

\runinhead{Local User} User accounts that exist specifically on a local machine and are not domain-based. If not secured properly, they can be used by attackers to gain a foothold on a network.

\runinhead{Locked Accounts} User accounts that have been locked due to numerous incorrect login attempts. An attacker might deliberately lock accounts to cause a denial of service or to disguise their activities.

\runinhead{Lockout Threshold} The Lockout Threshold parameter defines the number of invalid login attempts allowed before the account is locked. This parameter is crucial to thwarting brute-force attacks.

\runinhead{Login Cache} Active Directory stores a cache of user-login information locally on the system. If this cache is not properly secured, it can be exploited to gain unauthorized access to user accounts.
\textit{See also: Cached credentials}

\runinhead{Logon Hours} Defines the hours during which a user is allowed to log on to the domain. If not properly managed, it could provide a window of opportunity for attackers during off-hours.

\runinhead{Logon Script} A file that is assigned to a user account and runs automatically when the user logs in. A logon script can adjust the settings in the operating system, map network drives for different groups of users, or even display a welcome message that is specific to each user. These scripts reside in a folder on the \texttt{SYSVOL} network share of a domain controller and thus are available throughout the domain. If malicious content is inserted into these scripts, it can result in a widespread compromise of the systems.

\runinhead{Logon Workstations} This parameter specifies the machines from which a user can log in. If not appropriately restricted, it can lead to attacks in lateral movement.

\runinhead{Loopback Processing} This Group Policy setting allows the same user to have different policies when logging on to different machines. The misconfiguration of this setting can lead to privilege escalation attacks.

\runinhead{LSA Protection} A security feature in Windows that prevents the LSASS process from being accessed. It mitigates the risk of attacks aimed at extracting sensitive information from the LSASS process, such as those carried out using the Mimikatz tool.

\runinhead{LSA Secrets} Data objects that are stored by LSA to hold sensitive data such as credentials. The extraction of these secrets is a common tactic in credential theft attacks.

\runinhead{LZ77 Compression} Used in the compression of the AD database (\texttt{NTDS.DIT}) data. Exploitation of this mechanism can lead to data corruption or theft.

\section*{M}
\runinhead{Machine Account} A machine account in Active Directory is an automatically created account that represents a computer when it joins a domain. These accounts have their own credentials (passwords stored in AD) and are used to authenticate the machine to the domain. Attackers often target machine accounts during privilege escalation because they typically have less oversight than user accounts but can still be leveraged for service tickets and lateral movement. The thought process behind securing machine accounts is to treat them as sensitive AD objects, enforce strong password policies, and monitor for abnormal activity.

\runinhead{Machine Learning (ML)} A subset of AI focused on building algorithms that learn from and make predictions or decisions based on ingested data sets. In security, CL enables automated malware classification, spam filtering, and intrusion detection systems.

\runinhead{Machine Security Identifier (SID)} A Machine Security Identifier (SID) is a unique value that Windows assigns to each computer’s Security Accounts Manager (SAM) database. Within Active Directory, machine SIDs help identify and authenticate computers as trusted objects in the domain. The thought process behind protecting machine SIDs is to prevent attackers from impersonating legitimate domain-joined devices. If an attacker obtains or forges a machine SID, they may gain unauthorized access to network resources, move laterally, or exploit trust relationships within Active Directory.

\runinhead{Malspam} Malspam (malicious spam) refers to unsolicited emails used to deliver malware, phishing links, or malicious attachments to unsuspecting users. In an Active Directory environment, malspam is often the entry point for attackers who then take advantage of compromised user credentials, escalate privileges, and move laterally across domain-joined systems. The primary goal behind malspam is to exploit human error and weak email security as a gateway into AD, making email filtering, user awareness training, and AD-specific monitoring critical defenses.

\runinhead{Maintenance Mode} A state in which a server can be placed when applying updates or performing some other maintenance task. Failing to properly secure a server during maintenance can expose the system to potential attacks.

\runinhead{Malware} Malware is \textbf{mal}icious soft\textbf{ware} or code intended to damage or destroy computer systems and other personal devices by various means, including stealing data, gaining unauthorized access, sharing private information, etc. Cyberattackers can use malware to weaponize Active Directory and map possible attack paths, making it crucial for organizations to increase their focus on AD security and recovery. 

\runinhead{Man-in-the-Middle (MiTM) Attack} A man-in-the-middle (MiTM) attack is a cyberattack where the attackers' aim is to intercept, inspect, and even modify data exchanged between two parties (e.g., two users, a user and an application, a workstation and a server computer). This attack can lead to data breaches, exposing sensitive information and providing unauthorized access to network resources. In an Active Directory environment, MiTM can target protocols such as Kerberos or LDAP, allowing attackers to steal session tokens or credentials. The thought process behind MiTM in AD is to impersonate legitimate users or domain controllers, allowing escalation or persistence of privileges.

\runinhead{Managed Service Account (MSA)} A type of domain account that automatically manages password management, eliminating the potential for password expiration that could cause service interruptions. These accounts can be targeted for privilege-escalation attacks.

\runinhead{\texttt{ManagedBy}} An attribute in AD that specifies the user or group that manages an object. Misuse can lead to unauthorized access to resources.

\runinhead{Mandatory Attribute} An attribute defined in the Active Directory Schema as mandatory for a class of objects. For example, for a user object, 'sAMAccountName' is a mandatory attribute.

\runinhead{Mapped Drives} Network drives mapped to an individual system. If an attacker gains access to a system with mapped drives, they could potentially access sensitive data or spread ransomware across the network.

\runinhead{Maze} The Maze ransomware variant, discovered in 2019, is considered the first in which cyber attackers not only encrypted data, but also threatened to leak victims’ confidential data if their demands were not met. Maze usually gains access via phishing emails and then uses various techniques to move laterally through the network. It compromises and leverages Active Directory (AD) to propagate the ransomware payload to as many systems as possible.

\runinhead{Member Server} computer running a Windows Server operating system that is a member of an Active Directory domain but is not a domain controller.

\runinhead{\texttt{MemberOf}} This attribute contains the Distinguished Names of groups to which an object (user or group) belongs. Misconfiguration can lead to unauthorized access to resources.

\runinhead{Membership Caching} A feature in AD that allows caching of universal group membership for a user on a site to improve logon performance. If the cache data are not secured properly, it can be exploited to gain unauthorized access.

\runinhead{Message Queuing (MSMQ)} MSMQ is a messaging protocol that allows applications running on separate servers or processes to communicate. If not properly secured, it can be a potential point of exploitation, allowing unauthorized messages or commands.

\runinhead{Metabase} An AD database that stores the metadata for objects in Active Directory. If this database is compromised, an attacker can change the metadata associated with the AD objects.

\runinhead{Metadata} In the context of Active Directory, the data about the data in the directory refers to the data. This includes information on when and how data objects were created, modified, accessed, or deleted, including by whom. From a cybersecurity perspective, metadata can provide crucial information during a security investigation or audit, as it can reveal unauthorized changes, access patterns, or indicators of compromise.

\runinhead{Metasploit} Metasploit is a framework for penetration testing that makes hacking simple. It is an essential tool in an attacker’s arsenal, with numerous exploits, including those targeting AD environments.

\runinhead{Metaverse} The metaverse is a conceptual, immersive digital universe that blends physical reality with virtual environments, often accessed through 3D avatars, augmented reality (AR), and virtual reality (VR) platforms. In a cybersecurity context, the metaverse presents new challenges and risks, including identity theft, data privacy violations, platform abuse, and digital asset manipulation. With the rise of blockchain technologies, digital economies, and persistent virtual spaces, the metaverse expands the attack surface into areas where traditional security frameworks may not yet apply as of this writing.

\runinhead{Microsoft Challenge-Handshake Authentication Protocol (MS-CHAP)} A Microsoft-created authentication protocol, MS-CHAP has been found to have vulnerabilities and can be exploited if used for network authentication.

\runinhead{Microsoft Defender for Identity (MDI)} Microsoft Defender for Identity (formerly Azure Advanced Threat Protection) is a cloud-based solution that uses on-premises Active Directory signals to detect and respond to cybersecurity threats and compromised identities. Defender for Identity monitors and analyzes user and client activities and information across the network, creating a behavioral baseline for each user. MDI then alerts on unusual client or user activity as established by this baseline.

\runinhead{Microsoft Identity Integration Server (MIIS)} A centralized service for managing identities in multiple directories. If MIIS is breached, an attacker could manipulate identity data across systems.

\runinhead{Microsoft Identity Manager (MIM)} MIM is a service that provides tools and technologies to manage identities, credentials, and identity-based access policies in heterogeneous environments. MIM includes features for identity synchronization, certificate and password management, and user provisioning.

\runinhead{Microsoft Management Console (MMC)} MMC hosts administrative tools called snap-ins, including many for AD, such as the Active Directory Users and Computers snap-in. Inappropriate access to MMC can lead to unauthorized changes in AD.

\runinhead{Mimikatz} Mimikatz is an open source post-exploitation tool primarily used in Windows environments to extract plaintext passwords, hashes, PIN codes, and Kerberos tickets from memory. It can be used to mount Golden Ticket attacks, in particular, that exploit Kerberos vulnerabilities, allowing attackers to generate a ticket-granting ticket (TGT) and gain domain-level privileges. It is also used to perform pass-the-hash attacks that allow an attacker to authenticate to a remote server or service by using the underlying NTLM or LanMan hash of a user’s password. Additionally, the tool can be used to launch Silver Ticket attacks that involve creating fraudulent service tickets and allow access to a specific service on a specific machine, but can go undetected by the domain controller. Mimikatz was developed by Benjamin Delpy and is used by cybersecurity professionals for penetration testing and red team exercises, as well as by malicious actors for credential theft and lateral movement within a network.

\runinhead{MITRE ATT\&CK Framework} The MITRE ATT\&CK Framework is a commonly used tool to understand current security coverage and determine how to improve it. This knowledge base provides foundational information that can be used to develop threat models and is a popular tool for building comprehensive security plans. 

\runinhead{Mixed Mode} This term refers to the domain functional level of Active Directory when there are Windows NT 4.0 domain controllers present. In mixed mode, certain advanced features are disabled, making AD more susceptible to security risks.
\textit{See also: Native mode}

\runinhead{Mount Point} A location in a directory hierarchy where a volume is attached, providing additional file system locations. If a mount point is not secured properly, an attacker could gain unauthorized access to sensitive data.

\runinhead{\texttt{Move-ADDirectoryServer}} This PowerShell cmdlet is used to move a directory server to a new site. It is useful in larger organizations for managing the topology of Active Directory.

\runinhead{\texttt{\texttt{Move-ADDirectoryServerOperationMasterRole}}} This PowerShell command transfers one or more operations master (FSMO) roles to a specified domain controller.
\textit{See also: Flexible Single Master Operations (FSMO) Roles}

\runinhead{\texttt{\texttt{Move-ADObject}}} This PowerShell cmdlet is used to move an object or a container of objects to a different container or domain. 

\runinhead{\texttt{\texttt{MoveTree}}} This command-line tool is used to move AD objects between domains. Misuse can lead to accidental or malicious relocation of objects, leading to inconsistencies and potential breaches.

\runinhead{\texttt{\texttt{msDS-AllowedToDelegateTo}}} This attribute defines which services an account can represent in a Kerberos delegation. A misconfiguration can lead to privilege escalation and Kerberos delegation attacks.

\runinhead{\texttt{\texttt{msDS-Behavior-Version}}} This attribute dictates the Domain and Forest functional level of the AD. A lower functional level may expose the AD to vulnerabilities, as some security improvements are only available at higher levels.

\runinhead{\texttt{\texttt{msDS-ConsistencyGuid}}} This attribute is used as the source anchor attribute in Azure AD Connect. Misconfiguration can cause sync issues between on-premises AD and Azure AD, potentially causing authentication issues.

\runinhead{\texttt{\texttt{msDS-LastSuccessfulInteractiveLogonTime}}} This attribute stores the timestamp of the last successful interactive logon for the user. Unusual log-in times can indicate a potential security breach.

\runinhead{\texttt{\texttt{msDS-LockoutDuration}}} This attribute determines the length of time an account remains locked after exceeding the account lockout threshold. If too short, it may not prevent brute-force attacks effectively.

\runinhead{\texttt{\texttt{msDS-OptionalFeature}}} Optional features that have been enabled or disabled in an Active Directory forest. Misconfiguring these features could expose the forest to security risks.

\runinhead{\texttt{\texttt{msDS-PasswordSettings}}} This attribute contains the Fine-Grained Password Policies that are applied to user or group objects. If these settings are lax or misconfigured, it can leave user accounts vulnerable to brute-force or password-spray attacks.

\runinhead{\texttt{\texttt{msDS-PreviousSuccessfulLogon}}} This attribute stores the timestamp of the second most recent successful user logon. Anomalous login times can indicate a potential security breach.

\runinhead{\texttt{\texttt{msDS-PrimaryComputer}}} This attribute represents a user’s primary computer. If manipulated, the attribute can allow an attacker to impersonate a user machine, potentially leading to unauthorized access.

\runinhead{\texttt{\texttt{msDS-ReplValueMetaData}}} This attribute contains replication metadata for linked attributes, such as group memberships. A threat actor with access to this attribute can potentially alter group memberships, leading to privilege escalation.

\runinhead{\texttt{\texttt{msDS-SupportedEncryptionTypes}} }This attribute indicates the encryption types that the user account supports for Kerberos pre-authentication. Weak encryption types can make the account vulnerable to Kerberos-based attacks.

\runinhead{\texttt{\texttt{msDS-User-Account-Control-Computed}}} This attribute stores flags that dictate the status of the user account, like if it’s disabled, locked out, or has an expired password. Unauthorized manipulation of these flags can lead to privilege escalation or unauthorized access.

\runinhead{\texttt{\texttt{msDS-UserPasswordExpiryTimeComputed}}} This attribute provides the precise time when a user’s password will expire. If not properly managed, it can provide a window of opportunity for attackers to attempt credential-based attacks.

\runinhead{\texttt{\texttt{msiexec}}} A command-line interface utility for the Microsoft Windows Installer, which is used for installing, maintaining, and removing software.

\runinhead{\texttt{\texttt{msv1\_0}}} This authentication package in Windows deals with NTLM hashes. The famous pass-the-hash attack often targets this package, as NTLM hashes can be reused for authentication without cracking.

\runinhead{Multi-Site Clustering} A feature that allows clusters to span multiple Active Directory sites to improve availability. If not properly configured and secured, this type of clustering can become a potential attack vector.

\runinhead{Multi-Valued Attribute} An attribute of an object that can contain more than one value. Multivalued attributes can have no value, one value, or more than one. For example, the \texttt{memberOf} attribute for a user object, which contains a list of all groups the user belongs to.

\runinhead{Multi-Cloud Security} Multi-cloud security solutions help protect your infrastructure, application, and data on the cloud systems of multiple cloud providers.

\runinhead{Multifactor Authentication (MFA)} MFA is a security mechanism that requires users to prove their identity using two or more independent methods, or factors, before they are granted access. These factors can include something you know (like a password), something you have, such as a hardware token or a mobile phone), and something you are (like a fingerprint or other biometric factor). In the context of Active Directory, the implementation of MFA can greatly increase security by making it harder for attackers to gain access even if they have compromised a user’s password, reducing the risk of a successful phishing, password spray, or brute-force attack.

\runinhead{Multimaster Replication} Active Directory’s ability to allow changes to occur in any DC, which then replicates the changes to other DCs. If an attacker compromises a single DC, they could propagate malicious changes to others.

\runinhead{\texttt{\texttt{MutableID}}} A unique identifier used by Microsoft for an object in a directory. Unauthorized changes could lead to loss of access or inconsistencies in the directory.

\runinhead{Mutual Authentication} A security feature in which both the client and the server validate each other’s identity before establishing a connection. Without this, it is easier for an attacker to conduct Man-in-the-Middle (MiTM) attacks.

\section{N}
\runinhead{Name resolution} The process of resolving a hostname to an IP address within a network. Attacks such as DNS spoofing can manipulate this process to redirect network traffic.

\runinhead{Named Pipes} In the context of AD, Named Pipes are a method of interprocess communication (IPC). They are subject to IPC-related vulnerabilities such as DLL Hijacking or Named Pipe Impersonation.

\runinhead{Namespace} In Active Directory, a namespace is a container that holds objects such as users, computers, and other organizational units. A well-designed namespace can help prevent many security issues such as name clashes and replication errors.

\runinhead{Naming Context (NC)} Also known as Directory Partition in Active Directory, an NC is a portion of the directory that can be replicated to domain controllers. There are three types of NCs:

Schema
Configuration
Domain

\runinhead{Native Mode} A Domain Functional Level (DFL) that only applies to Windows 2000 Server, and that does not support Windows NT domain controllers. Once in native mode, the domain will support nested groups. The alternative is mixed mode.
\textit{See also: Mixed mode}

\runinhead{\texttt{nbstat}} A command-line utility to report NetBIOS over TCP/IP statistics.

\runinhead{Nested Groups} In Active Directory, groups can contain other groups, allowing hierarchical organization; however, nesting can lead to unintentional permission escalation and access to resources if not managed carefully.

\runinhead{\texttt{net use}} A command-line tool used to connect, disconnect, and configure connections to shared resources, like network drives and printers. If misused, this tool could lead to unauthorized access to resources.

\runinhead{NetBIOS} The acronym for Network Basic Input/Output System is a networking protocol used by Windows systems for communication on a local network. The NetBIOS name of a computer is generally the first 15 characters of the host name, followed by the character `\$\textperiodcentered` The resolution of NetBIOS names to IP addresses is provided by local broadcasts and the WINS service.

\runinhead{netcat} Known as the “Swiss Army knife” for TCP/IP, Netcat can read and write data across network connections. Attackers can use Netcat to create backdoors, transfer files, or carry out network exploration.

\runinhead{\texttt{netlogon}} A Windows service used for user and computer authentication in older operating systems. A share called Netlogon is automatically created on all domain controllers for backward compatibility and can hold logon scripts. The recent ZeroLogon exploit allowed attackers to gain control over the domain controller using this service.

\runinhead{\texttt{netsh}} A command-line scripting utility that, among other things, enables the modification of network configurations. Attackers could misuse this tool to manipulate network traffic or exfiltrate data.

\runinhead{\texttt{netstat}} Netstat (short for \textit{network statistics}) is a command-line utility available on Windows, Linux, and other operating systems that displays active TCP/UDP connections, listening ports, routing tables, and interface statistics. In Active Directory environments, administrators often use \verb|netstat| to identify which services are listening on domain controllers or to troubleshoot connectivity issues with AD-related services such as LDAP, Kerberos, or DNS. Attackers may also take advantage of  \texttt{netstat} during reconnaissance to map which ports are open and which AD services are reachable, making it a defensive and offensive tool.

\runinhead{Network Access Control (NAC)} Network Access Control (NAC) is a security solution that restricts device access to enterprise networks based on compliance with predefined security policies. Provides mechanisms for installing, enforcing, and locking down security rules that determine access to the internal LAN.

\runinhead{Network Access Protection (NAP)} Network Access Protection (NAP) is a deprecated Microsoft technology that enforced health policies on devices that attempt to connect to an enterprise network. Before access was granted, devices were evaluated for compliance with security requirements, such as up-to-date patches, antivirus status, or firewall settings. In Active Directory–integrated environments, NAP worked alongside Group Policy to enforce these standards. Although Microsoft retired NAP in Windows Server 2012 R2, its legacy still appears in some environments. If left enabled or poorly configured, attackers could bypass weak enforcement, gaining access to internal AD resources with non-compliant or even compromised machines.

\runinhead{Network Address Translation (NAT)}Network Address Translation (NAT) is a process by which a router or firewall translates private IP addresses within a local network into a single public IP address (or a pool of public addresses) for communication with external networks such as the internet. Although NAT is not a core part of Active Directory, it often exists in hybrid AD/cloud deployments or perimeter defenses where AD resources need selective exposure to the Internet. Misconfigured NAT rules can accidentally expose domain controllers, Remote Desktop Protocol (RDP), or LDAP ports, creating serious attack vectors for adversaries to directly target AD infrastructure.

\runinhead{Network Discovery} This refers to the process of identifying all devices in a network. In the context of Active Directory, ensure that only authorized individuals can perform network discovery to prevent unwanted reconnaissance.

\runinhead{Network File System (NFS)} Network File System (NFS) is a distributed file system protocol that allows systems to share files on a network as if they were on a local disk. Although more common in UNIX/Linux environments, Windows also supports NFS through optional services. In mixed AD/UNIX environments, NFS can integrate with Active Directory for authentication via Kerberos; however, misconfigured NFS shares can expose sensitive files (such as AD backup data or credential caches) to unauthorized access. Attackers frequently investigate NFS exports during enumeration to find credentials or AD-related configuration data.

\runinhead{Network Interface} A network interface represents the point of interconnection between a computer and a network. In Active Directory environments, every domain controller, client, and server relies on a properly configured interface to communicate with LDAP, Kerberos, DNS, and other critical AD services. Misconfigured network interfaces (e.g. incorrect DNS settings) can cause AD authentication failures or replication breakdowns. Attackers may manipulate virtual or physical network interfaces to redirect traffic, perform sniffing, or inject malicious responses into AD-related communication flows.

\runinhead{Network Level Authentication (NLA)} Network Level Authentication (NLA) is a security feature of the Remote Desktop Protocol (RDP) that requires a user to authenticate before establishing a full RDP session with a server. When enabled on domain-joined machines, NLA helps reduce the attack surface by requiring valid AD credentials before any RDP logon attempt. Without NLA, attackers can flood RDP with brute-force attempts or exploit unpatched vulnerabilities like BlueKeep. Because domain controllers and Tier 0 systems are often targets for RDP abuse, enforcing NLA is considered an essential AD security best practice.

\runinhead{Network Listener} A network listener is a process or service that actively listens to incoming connections on a specified port and protocol. For example, a domain controller runs listeners for LDAP (389/636), Kerberos (88), and DNS (53). Administrators can use tools such as \texttt{netstat} or \texttt{Get-NetTCPConnection} in PowerShell to enumerate active listeners in AD systems. From a security perspective, each listener represents a potential attack surface; if unnecessary services are listening, attackers may attempt exploitation or brute-force attacks to gain access to Active Directory. Hardening AD often involves auditing and minimizing exposed network listeners to reduce lateral movement paths.

\runinhead{Network Policy Server (NPS)} NPS is the Microsoft implementation of a RADIUS server and proxy. As with any authentication system, it is a critical security component, and any compromise can lead to unauthorized network access.

\runinhead{Network Segmentation} A security practice in which different portions of a network are separated from each other. This can limit the spread of lateral movement in an Active Directory compromise.

\runinhead{Network Services} The services (e.g. DNS, DHCP) that are made available from a server to a private or public network. If these services are compromised, it can have a direct impact on the security of the Active Directory environment.

\runinhead{Network Sniffing} Network sniffing refers to the use of network protocol analyzers or similar tools to capture and analyze network traffic. In the context of AD, this could potentially expose sensitive unencrypted data.

\runinhead{Network Time Protocol (NTP)} A networking protocol for clock synchronization between computer systems over packet-switched, variable-latency data networks. In Active Directory, accurate timekeeping is essential for Kerberos authentication, as there is a maximum time difference (5 minutes by default) allowed between the time on the client and the time on the server. If an attacker can manipulate NTP responses on a network, they could potentially exploit this for replay attacks or even to cause authentication failures across the network; therefore, it is crucial to secure NTP communications.

\runinhead{New Technology Directory Services (NTDS)} NTDS stands for Windows NT Directory Services and describes the fundamental directory services that facilitate the management and organization of network resources.

\runinhead{New Technology File System (NTFS)} The New Technology File System (NTFS) is the file system that the Windows NT operating system uses to store and retrieve files on a hard disk. NTFS is the primary file system for recent versions of Windows and Windows Server.

\runinhead{\texttt{New-ADOrganizationalUnit}} This PowerShell cmdlet is used to create a new AD Organizational Unit (OU). If misused, it can lead to the creation of unnecessary OUs, disrupting the structure of AD, and potentially masking unauthorized changes.

\runinhead{\texttt{New-ADUser}} A PowerShell cmdlet used to create a new user object in Active Directory. The use of an attacker could lead to the creation of backdoor accounts for persistent access.

\runinhead{NIST Cybersecurity Framework (NIST CSF)} The National Institute of Standards and Technology (NIST) is an agency of the US Department of Commerce. An example of how NIST executes its mission, 'to promote US innovation and industrial competitiveness by advancing measurement science, standards, and technology in ways that improve economic security and improve our quality of life', is the development of the NIST Cybersecurity Framework. This popular approach to identifying and resolving high-priority risks to Active Directory (AD) and other crucial systems comprises five phases:

1. Identify
2. Protect
3. Detect
4. Respond
5. Recover

\runinhead{Nmap ("Network Mapper")} Nmap is a security scanner that is used to discover hosts and services on a computer network, thus creating a “map” of the network. Attackers can use this scanner for network discovery and security auditing.

\runinhead{Nonauthoritative Restore} A nonauthoritative restore restores an AD domain controller to a point in time; however, because this type of restore is not marked as authoritative, if another domain controller has updated objects or attributes since the restored date and time of the target domain controller, those updates replicate in the restored domain controller, keeping its data current.
\textit{See also: Authoritative Restore}

\runinhead{Non-Delivery Report/Receipt (NDR)} An NDR indicates that a particular piece of communication (such as an email or a packet) has not been delivered. Cybercriminals can use NDRs to obtain information about the internal structure of an organization’s email system to perform a targeted attack.

\runinhead{Nonrepudiation} In the context of Active Directory, non-repudiation refers to the capability to ensure a party in a dispute cannot deny the validity of the evidence (like a user denying his activities). A weak audit policy can lead to poor non-repudiation.

\runinhead{Normalization} The process of modifying data to fit a desired format. Attackers can bypass input validation checks through normalization inconsistencies.

\runinhead{NotPetya} The “wiper” malware of NotPetya acts like ransomware, but it does not have a way to reverse its encryption. NotPetya particularly affects Active Directory (AD), bringing operations to a halt. NotPetya is famous for its devastating attacks in 2017 that began in Ukraine and continued to cause an estimated—and unprecedented—\$10 billion in damages worldwide.

\runinhead{\texttt{nslookup}} A command-line utility to diagnose Domain Name Service (DNS) infrastructure problems.

\runinhead{NT Service Accounts} Built-in service accounts in Windows systems. If these accounts are compromised, they can often provide high levels of system access.

\runinhead{NT Service Hardening} NT Service Hardening restricts Windows services from performing abnormal activities on the file system, registry, network, or other resources that could be used to allow malware to persist or spread.

\runinhead{NTBackup} A built-in Windows backup utility. If an attacker can manipulate or access these backups, they may gain access to sensitive data.

\runinhead{Ntdetect.com} This is a Windows system file that is used in the boot process. Tampering with or compromising this file can lead to persistent system-level access for an attacker.

\runinhead{NTDS Quotas} NTDS quotas limit the number of objects a security principal can own in the directory. If not properly set, this could be leveraged for a Denial of Service (DoS) attack.

\runinhead{\texttt{NTDS.dit}}The AD database, known as the NTDS.dit file, is a critical directory services component stored on domain controllers. This database contains all information about user objects, including hashed passwords, so it is frequently targeted by attackers.

\runinhead{\texttt{ntdsutil.exe}} A command-line tool that provides management facilities for Active Directory Domain Services (AD DS) and Active Directory Lightweight Directory Services (AD LDS). If used maliciously, it can cause significant damage to AD services.
\textit{See also: Active Directory Domain Services (AD DS); Active Directory Lightweight Directory Services (AD LDS)}

\runinhead{NTDSXtract} A tool for extracting AD data from \texttt{ntds.dit} files (the AD database). NTDSXtract can be used to discover usernames, group memberships, password policies, and more. This data can aid an attacker in planning and conducting attacks.

\runinhead{NTFS Permissions} Permissions on file system objects on NTFS volumes, managed through Active Directory. Misconfigurations can lead to unauthorized access to data or denial of service.

\runinhead{NTLM (New Technology LAN Manager)} NTLM is a collection of security protocols used to authenticate, provide integrity, and confidentiality to users. Although Kerberos is the preferred authentication protocol and is used in modern Windows versions, NTLM is still available for older clients and systems in a workgroup. NTLM has several security flaws, prone to various attacks such as pass-the-hash and pass-the-ticket, that allow attackers to gain access to user passwords and should therefore be avoided.

\runinhead{NTLM Relay Attack} In an NTLM relay attack, an attacker intercepts NT LAN Manager (NTLM) authentication sessions between computers on a network and then forwards (relays) the credentials to another host on the network. This enables the attacker to execute commands or access resources on the second host using the intercepted credentials.

\runinhead{NTLMRecon} A tool built with the aim of making NTLM protocol-based reconnaissance fast and easy. It can be used to identify domains and services which support null sessions.

\runinhead{Null Bind} A type of anonymous bind in LDAP. When null bind is enabled, it can allow an anonymous user to connect to the directory and potentially access sensitive information.

\runinhead{Null Session} Null sessions in Windows are unauthenticated NetBIOS sessions, which can allow an attacker to gather a wealth of information about the system.

\section*{O}
\runinhead{OAuth} OAuth is an open standard authorization protocol that allows applications to access user resources without exposing credentials. Enables delegated access using access tokens, commonly used in APIs and federated authentication systems.

\runinhead{Object} In Active Directory, an object is a distinct and named set of attributes that represent something on the network, such as a user, a computer, or a group. Securing these objects is vital, as their compromise can enable attackers to gain unauthorized access or elevate their privileges.

\runinhead{Object Access Auditing} This feature enables you to collect information whenever a specified type of object is accessed. Lack of appropriate auditing could allow malicious activity to go unnoticed.

\runinhead{Object Deletion} This Active Directory event occurs when an object, such as a user or a group, is removed. Monitoring for unexpected object deletions is important to catch potential malicious activity.

\runinhead{Object Identifier (OID)} This globally unique value is used to identify a variety of things, including schema attributes and classes, security mechanisms, and name forms. OIDs are crucial for the interoperability and extensibility of the directory service. In many LDAP directory implementations, an OID is the standard internal representation of an attribute. Each attribute in the Active Directory schema has a unique X.500 OID. All OID values that are created by Microsoft begin with 1.2.840.113556.

\runinhead{Object inheritance} Object inheritance is a property of Active Directory objects in which child objects inherit the permissions of their parent objects. Incorrect settings can lead to excessive permissions being accidentally granted.

\runinhead{\texttt{ObjectCategory}} This attribute is used to group similar classes together. Incorrect configuration can lead to misclassifications and potential security vulnerabilities.

\runinhead{\texttt{ObjectClass}} This attribute determines the kind of objects (e.g., user, computer, group) that are stored within Active Directory. It also specifies the set of must-have attributes (i.e., every object of the class must have at least one value of each) and may-have attributes (i.e., every object of the class may have a value of each). \texttt{ObjectClass} is defined in a \texttt{classSchema} object. Manipulation of this attribute can enable an attacker to mask malicious activity.

\runinhead{\texttt{ObjectGUID}} A unique identifier for an object in Active Directory, which remains constant. Attackers could use\texttt{ objectGUIDs} to maintain persistence in the environment.

\runinhead{\texttt{ObjectSID}} This unique identifier is assigned to each object in an Active Directory domain. If an attacker can manipulate these identifiers, they could gain unauthorized access.

\runinhead{Offline Domain Promotion} This process enables a computer to join a domain without needing network connectivity. A potential attack vector could force a device to join a domain controlled by an attacker.

\runinhead{Offline NT Password \& Registry Editor} This tool can be used to change the password of any user who has a valid local account on a Windows system. If an attacker gains physical access to a system, they can use such tools to gain access to a local account and potentially escalate their privileges.

\runinhead{Okta} This popular identity and access management (IAM) platform provides cloud software to help companies manage and secure user authentication in modern applications and to build identity controls in applications, web services on the website and devices. It can be integrated with Active Directory to manage user access across on-premises and cloud environments. If compromised, it can lead to widespread unauthorized access.

\runinhead{\texttt{Oldcmp}} This command-line tool, developed by Joe Richard (DS-MVP) to query Active Directory for unused computer or user accounts, can also clean up accounts.

\runinhead{On-Premises Directory Sync (DirSync)} This tool is used to replicate Active Directory information on premises to Microsoft cloud services. A compromise of DirSync could expose sensitive on-premises Active Directory data to an attacker.

\runinhead{One Time Password (OTP)} A password that is valid for only one session or transaction of log-in. It is commonly used in two-factor authentication systems to provide an additional layer of security beyond just a username and password.

\runinhead{One-Way Trust} In Active Directory, a one-way trust is a unidirectional authentication path created between two domains. If an attacker compromises this, they could gain unauthorized access to resources in one domain while originating from the other.

\runinhead{Online Certificate Status Protocol (OCSP)} It’s an internet protocol used for obtaining the revocation status of an X.509 digital certificate. An attacker could perform an OCSP spoofing attack to imitate a certificate authority and issue fraudulent certificates.

\runinhead{Operating System (OS)} The underlying software that controls a computer or server. Hardening the OS of domain controllers is critical for Active Directory security.

\runinhead{OSI Model (Open Systems Interconnection Model)} The OSI model is a conceptual framework, developed by the International Organization for Standardization(ISO), that standardizes the functions of a telecommunication or computing system in terms of seven abstraction layers. Divides the complex process of network communication into seven (7) distinct layers, each with specific functions, roles, and responsibilities to enable network transmissions to be uninterrupted. The OSI model provides a universally understood language for discussing network protocols and troubleshooting network issues.

\runinhead{Operational Attributes} Attributes that the directory automatically provides, such as \texttt{creationTimeStamp}. These attributes are calculated by a domain controller upon request. They can be used to track actions and changes in objects. An LDAP search request requesting 'all attributes' does not return operational attributes and their values. Attackers might attempt to manipulate or clear these to hide their activities.

\runinhead{Optional Attribute} An attribute defined in the schema as optional for a class of objects. Unlike mandatory attributes, optional attributes are not required to have a value.

\runinhead{Organizational Identity} In terms of Active Directory, organizational identity refers to the credentials and identifiers that belong to a company or organization. Protecting these identifiers is crucial to prevent attackers from masquerading as legitimate entities within the organization.

\runinhead{Organizational Unit (OU)} An OU is a container type within an Active Directory in which you can place users, groups, computers, and other organizational units. OUs can be used to assign group policies and manage resources, and misconfiguration can lead to inappropriate access.

\runinhead{Orphaned Object} An orphaned object remains in the directory database but has been deleted in every practical sense because its parent object was deleted. These objects can be exploited by an attacker to hide their activities or maintain persistence in an environment.

\runinhead{OS Provisioning} Operating system (OS) provisioning is the act of installing a given operating system on several hosts. 

\runinhead{Out-of-Band Patch} This type of patch is released at an unscheduled time, usually to address a specific vulnerability. Apply these patches quickly to avoid exploitation.

\runinhead{Out-of-the-Box Configuration} The default settings and configuration that come with a system or software when it is first installed. These configurations can sometimes be insecure, so it is important to harden them based on best practices to avoid easy exploitation.

\runinhead{Outbound Replication} Outbound replication is the process by which a domain controller replicates changes to other domain controllers. If an attacker can manipulate this process, they can propagate malicious changes across the domain.

\runinhead{Outbound Trust} A trust relationship between two domains for the purpose of authentication. The outbound trust is from the perspective of the domain that trusts another domain.

\runinhead{Outlook Anywhere} Outlook Anywhere allows access to your Exchange Server from your Outlook client without using VPN. This can provide a potential entry point for attackers if not properly secured.

\runinhead{Outlook Web Access (OWA)} OWA provides the ability to access email via a Web browser. If not secured properly, this can provide an entry point for attackers.

\runinhead{Overlapping Site Link} In this configuration, two site links have sites in common. This configuration can lead to inefficient replication traffic if not properly managed, providing potential opportunities for an attacker to intercept sensitive data.

\runinhead{Overpass-the-Hash Attack} An overpass-the-hash attack is similar to a pass-the-hash attack but involves Kerberos rather than NTLM. The attacker uses a user password hash to generate a Kerberos preauthentication hash, which is then used to request a ticket granting ticket (TGT) from the domain controller.
\textit{See also: Kerberos; Pass-the-Hash (PtH) attack}

\runinhead{Owner} In Active Directory, every object has an owner, which has certain rights over the object. An attacker who takes ownership of an object could potentially misuse those rights.

\runinhead{Owner Rights} This security principle represents the current owner of an object. An attacker gaining owner rights can modify important attributes of the object, leading to security breaches.

\section*{P}
\runinhead{Parent Container} In Active Directory, the parent container refers to the object of the higher level container that holds other objects within.

\runinhead{Parent Object} An object is either the root of a tree of objects or has a parent object above it in the tree hierarchy. If two objects have the same parent, they must have different relative distinguished names (RDNs).

\runinhead{Parent-Child Domain Relationship} A hierarchical relationship between two domains in Active Directory, where one domain is the parent and the other is the child. Child domains inherit policies from their parent domain.

\runinhead{Partial Attribute Set (PAS)} The subset of attributes that replicate in replicas of partial naming contexts (NC). Specifies which attributes should be replicated to Global Catalog servers.

\runinhead{Partition} A logical division of the Active Directory database that stores objects and attributes. Partitions include the schema partition, the configuration partition, and the domain partitions. Proper partition management is essential for maintaining the integrity and scalability of the directory.

\runinhead{Pass-the-Hash (PtH) Attack} In a pass-the-hash attack, an attacker gains access to the password hash of a user’s account in Active Directory and uses it to authenticate and impersonate the user without knowing the actual password. This attack takes advantage of weak hashing algorithms or stolen password hashes.

\runinhead{Pass-the-Ticket (PtT) Attack} In a pass-the-ticket attack, an attacker steals a Kerberos ticket-granting ticket (TGT) from a user’s machine and uses that TGT to gain unauthorized access to resources, without need to authenticate.

\runinhead{Pass-Through Authentication} This type of authentication enables users to use the same username and password on premises and in the cloud, without the need for a third-party federation system.

\runinhead{Password Change Notification Service (PCNS)} This feature enables synchronization of password changes in Active Directory to other systems. For example, when a user changes their password in Active Directory, the PCNS ensures that this change is reflected in other systems that the user has access to.

\runinhead{Password Complexity} A requirement in the password policy that requires the use of strong passwords containing a combination of upper- and lowercase letters, numbers, and special characters. Enforcing password complexity makes it harder for attackers to guess or crack passwords. Although strong password complexity requirements are important, it is equally crucial to educate users about the importance of password hygiene, avoiding password reuse across multiple accounts, and adopting additional security measures like multifactor authentication for enhanced protection.

\runinhead{Password Expiration} A policy that requires users to change their passwords after a specified time interval. Password expiration helps enforce regular password updates and reduces the risk of compromised credentials being used for an extended period.

\runinhead{Password Filter} A component of Active Directory that intercepts and validates password changes to enforce custom password policies or perform additional checks. Password filters are used to improve password security and prevent weak or easily guessed passwords from being set.

\runinhead{Password Hash} A mathematical representation of a user’s password stored in Active Directory. Attackers target password hashes because they can be cracked offline, allowing unauthorized access to user accounts. Techniques like Pass-the-Hash attacks exploit password hashes.

\runinhead{Password Hash Synchronization} A feature of Azure AD Connect that synchronizes the password hashes of Active Directory user accounts on premises to Azure AD (Entra ID), enabling users to use their on premises passwords to log in to cloud services. Protecting password hash synchronization is crucial to prevent user credentials from being compromised.

\runinhead{Password History} This mechanism in Active Directory prevents users from reusing previously used passwords by maintaining a history of their password changes. Password history policies improve security by discouraging the reuse of old passwords.

\runinhead{Password Lockout} This security feature in Active Directory temporarily locks a user account after a specified number of failed login attempts. Password lockout policies help prevent brute-force attacks and unauthorized access attempts.

\runinhead{Password Policy} A set of rules and requirements that dictate the complexity, length, expiration, and other characteristics of user passwords in Active Directory. Weak password policies can make user accounts susceptible to brute-force attacks and credential guessing.

\runinhead{Password Policy Enforcement} The process of ensuring that the defined password policies are enforced and adhered to by all users within Active Directory. Proper enforcement of the password policy helps mitigate the risk of weak passwords and improves overall security.

\runinhead{Password Reset} The process of changing a user’s forgotten or expired password in Active Directory. Proper implementation of password reset procedures ensures secure and authorized access restoration while minimizing the risk of social engineering or unauthorized password changes.

\runinhead{Password Setting Object (PSO)} Objects in the Active Directory system container that implement fine-grained password policies (FGPP).

\runinhead{Password Spray Attack (Password Spraying)} A technique used by attackers to guess common or weak passwords against a large number of user accounts in Active Directory, with the aim of avoiding account lockouts and increasing the chances of successful authentication. Rather than trying multiple passwords against one user, which can trigger account lockouts, an attacker tries one commonly used password against multiple accounts, reducing the risk of detection and increasing the chance of successful authentication.

\runinhead{Passwordless Authentication} An authentication method that eliminates the need for traditional passwords, often replaced by biometric factors (e.g., fingerprint or facial recognition), hardware tokens, or other secure authentication methods. Passwordless authentication reduces the risk of password-related attacks and can enhance user convenience.

\runinhead{Patch Management} The process of regularly applying software updates (patches) to fix vulnerabilities and bugs in Active Directory components and associated systems. Failing to patch systems can leave them exposed to known exploits and attacks.

\runinhead{PDC Emulator} One of the five Flexible Single Master Operations (FSMO) roles in AD, a PDC emulator acts as the Windows NT Primary Domain Controller (PDC) for backward compatibility. The PDC Emulator is responsible for handling password changes, user lockouts, and Group Policy. It also serves as the primary time source for the domain. The PDC Emulator is targeted by most Group Policy tools. A domain controller in each domain must hold this role.
\textit{See also: Flexible Single Master Operations (FSMO) Roles}

\runinhead{Permission inheritance} The process by which objects in Active Directory inherit permissions from their parent containers or organizational units (OUs). Understanding and properly managing permission inheritance is important to ensure consistent access control and prevent unintended access to sensitive resources.
\textit{See also: Inheritance}

\runinhead{Permissions} Access rights granted to users, groups, or computer accounts that determine what actions they can perform on Active Directory objects. Misconfigured permissions can lead to unauthorized access, increased privileges, or exposure of sensitive information.

\runinhead{Petition Attack (Access Token Manipulation)} An attacker uses a valid account to generate an access token and modifies the token to elevate privileges, allowing the attacker to execute commands or access resources that would normally be beyond their permissions.

\runinhead{PetitPotam} PetitPotam is an attack method in which cyber-attackers force a victim client to authenticate to an arbitrary machine without any user interaction. When PetitPotam is exploited and Windows NT LAN Manager (NTLM) credentials are sent to Active Directory Certificate Services, a cyberattacker can get Domain Administrator privileges without prior authentication to the domain.

\runinhead{Pharming} Pharming is a cyberattack technique that redirects users from legitimate websites to fraudulent ones, usually by compromising DNS settings or modifying local host files. Unlike phishing, which relies on tricking users into clicking malicious links, phishing can silently redirect traffic even when a user typed the correct URL. The thought process behind pharmacy is to capture sensitive information, such as usernames, passwords, or financial data. In an Active Directory environment, phishing attacks can be especially dangerous if AD-integrated DNS servers are compromised, as this would allow attackers to redirect authentication traffic, harvest credentials, or impersonate critical services. Properly securing AD DNS and monitoring for unauthorized changes is essential to mitigate this risk.

\runinhead{Phishing} A deceptive technique used by attackers to trick users into revealing their credentials or sensitive information through fraudulent emails, websites, or other communication channels. Phishing attacks often target Active Directory users to gain unauthorized access to the network. Once they obtain the user credentials, they will begin to gather information on the system resources and attempt to move laterally through the network. Additionally, these credentials can be used to create more precise, personalized, and dedicated attacks on specific high-value company users.

\runinhead{PingCastle} PingCastle is an Active Directory assessment tool written in C#. Based on built-in models and rules, this tool evaluates AD subprocesses and generates a risk report that includes a score for privileged accounts, trust relationships between AD domains, insights into stale objects and security anomalies. For hybrid environments, it can also provide insight into whether the trust relationship with Azure AD is secure. Attackers can use this tool to analyze the state of the AD environment and identify potential vulnerabilities.

\runinhead{PowerSploit} A PowerShell Post-Exploitation Framework, PowerSploit is widely used by attackers for various tasks such as executing payloads, exfiltration, privilege escalation, and more in an AD environment.

\runinhead{Pre-Windows 2000 Compatible Access Group} This built-in group in Active Directory includes all user and computer accounts that need compatibility with older Windows NT 4.0–based systems.

\runinhead{Pre-Windows 2000 Name} Used for compatibility purposes with older operating systems. For user and group objects, this name is the value of the sAMAccountName attribute.

\runinhead{Primary Domain Controller (PDC)} In Windows NT-based domains the primary domain controller (PDC) was the first domain controller in the domain, and it had zero or more backup domain controllers (BDC). A domain controller (DC) designated to track changes made to the accounts of all computers in a domain. Although used by many, the PDC concepts no longer apply in Windows 2000 and above Active Directory, as all domain controllers are essentially equal from a replication perspective, because it uses multi-master replication model. Not to be confused with PDC emulator.

\runinhead{Primary Group} A group assigned to each user in Active Directory, representing their primary affiliation within the domain. Used mostly for POSIX compliance. Primary groups are used for access control and are associated with permissions on resources.

\runinhead{Primary SMTP Address} The email address associated with a user or mailbox that is considered the primary address for communication. Active Directory uses the primary SMTP address as the unique identifier for email-related operations.

\runinhead{Privilege} The right of a user to perform system-related operations, such as debugging the system. Rights and privileges are effectively the same and are granted to security principals such as users, services, computers, or groups. Some (such as Enable computer and user accounts to be trusted for delegation) apply to Active Directory, whereas others (such as Change the system time) apply to the Windows operating system. The user is considered a member of its primary group. The user’s authorization context specifies what privileges are held by that user.

\runinhead{Privilege Delegation} The process of assigning specific administrative privileges or rights to non-administrator users or groups within Active Directory. Privilege delegation allows users to perform specific tasks without granting them full administrative access, minimizing the risk of privilege abuse or unauthorized changes.

\runinhead{Privilege Escalation} The process of gaining higher levels of access and permissions than originally assigned. Privilege escalation attacks within Active Directory can enable attackers to bypass security controls and gain administrative privileges. Once inside your environment, cyberattackers typically seek escalation of privileges, from lower- to higher-privileged accounts, in an effort to gain administrative privilege and access to Active Directory.
\textit{See also: Domain Dominance; Lateral Movement; Privileged Access}

\runinhead{Privilege Separation} The practice of dividing administrative privileges between multiple user accounts in Active Directory, ensuring that no single account has excessive or unrestricted access. Separation of privileges helps minimize the impact of compromised accounts and reduces the risk of unauthorized actions.

\runinhead{Privileged Access} Privileged access grants higher than standard rights and control over resources in an environment higher than standard. This type of access should be granted sparingly, as gaining control of a privileged access account can enable cyberattackers to shut down or disable Active Directory and gain control of your network.
\textit{See also: Domain Dominance; Lateral Movement; Privilege Escalation}

\runinhead{Privileged Access Management (PAM)} A solution in Microsoft Identity Manager (MIM) that helps mitigate security concerns associated with the escalation or misuse of administrative privileges.

\runinhead{Privileged Account} An account with elevated permissions and access rights that can perform administrative tasks within Active Directory. Compromised privileged accounts are a prime target for attackers as they can grant extensive control over the domain.

\runinhead{\texttt{PsExec}} PsExec is a light-weight telnet-replacement that lets you execute processes on other systems. Attackers use this as a tool for lateral movement in an AD environment.

\runinhead{PsTools} PsTools is a set of widely used command-line tools that allow you to manage local and remote systems. These tools can be used by attackers for various tasks, such as running processes remotely, shutting down systems, or viewing system information.

\runinhead{\texttt{PTH-Winexe}} Part of the pass-the-hash toolkit, PTH-Winexe allows execution of commands on Windows systems by passing NTLM hashes instead of plaintext credentials, enabling lateral movement and remote command execution.

\runinhead{Public Key} A cryptographic key used in asymmetric encryption algorithms, consisting of public and private keys. Public keys are widely distributed and are used to encrypt data or verify digital signatures in Active Directory communications.

\runinhead{Public Key Certificate} A digitally signed document that binds a public key to the identity of an entity, validating the authenticity and integrity of the public key. Public-key certificates are used in Active Directory for secure communication, authentication, and encryption.

\runinhead{Public Key Infrastructure (PKI)} A framework of cryptographic services, technologies and protocols used to establish and manage digital certificates, public-private key pairs, and secure communication in Active Directory. PKI is crucial to ensure secure authentication, data integrity, and confidentiality.

\runinhead{\texttt{Pwdump}} Pwdump is a tool that extracts NTLM and LanMan password hashes from the Windows Security Account Manager (SAM), which can then be cracked offline. This can lead to unauthorized access if strong password policies are not in place.

\runinhead{Pass-the-Hash (PtH)} Exploits the NTLM authentication protocol by capturing and reusing a user's password hash to log in as that user, bypassing the need for the actual password.

\runinhead{Pass-the-Ticket (PtT)} A Pass-the-Ticket (PtT) attack is similar in nature to Pass-the-Hash (PtH) but uses Kerberos authentication tickets to move laterally across a domain network.

\runinhead{Persistence} In the context of hacking, persistence refers to techniques used by attackers to maintain access to a compromised system even after system reboots, restarts, reimages, user logoffs, or other disruptions. It allows them to establish a long-term presence on the target network, enabling further attacks, data theft, or other malicious activities.

\runinhead{PowerSploit / PowerView} A collection of PowerShell scripts used for AD enumeration and information gathering to identify vulnerabilities and attack pathways.

\runinhead{Primary Domain Controller (PDC) Emulator} The PDC Emulator is a Flexible Single Master Operations (FSMO) role held by a domain controller. It acts as the primary domain controller for password changes, time synchronization, and Group Policy Object (GPO) updates within a domain.

\section{Q}
\runinhead{Qualitative Risk Management} Write down a description of the glossary term. Write down a description of the glossary term. Write here the description of the glossary term.
\runinhead{Quantitative Risk Management}

\runinhead{Query} A query in Active Directory refers to a request made to the directory service to retrieve specific information, such as the details of a user or a computer. AD queries are essential for managing and controlling access to resources. The lightweight Directory Access Protocol (LDAP) is commonly used to perform queries against the AD database. For example, an administrator might run a query to find all the users in a specific department.

\runinhead{Quotas} The quotations in Active Directory refer to the limits placed on the number of objects that a security principal (such as a user or a group) can own in a directory partition. This is particularly important in a large distributed environment to prevent any user or group from creating so many objects that it affects AD performance or storage.

\section*{R}
\runinhead{Rainbow Table} A precomputed textual table used to reverse cryptographic hash functions, primarily to crack password hashes. Each rainbow table is specific to a certain hash function, character set, and password length. Rainbow tables contain millions or even billions of precomputed hashes for potential passwords. An attacker who obtains the hashed version of a password (perhaps through a data breach or by penetrating an inadequately secured Active Directory system) can use a rainbow table to look up that hash and potentially find the original password.

\runinhead{RainbowCrack} A tool that attempts to crack password hashes with rainbow tables.

\runinhead{Ransomware} A type of malware that encrypts the data of a victim until payment is made to the cyberattacker. Victims are told that if payment is made, they will receive a decryption key to restore access to their files, although this is often a ruse. In a double-extortion attack, not only is the decryption key withheld, but the malicious actor also threatens to publish the data on data leak sites (DLS). Ransomware groups are often connected to criminal or terrorist organizations or to hostile nation states. The payment of a ransom typically funds further criminal activities.
\textit{See also: Malware; Ransomware-as-a-Service (RaaS)}

\runinhead{Ransomware-as-a-Service (RaaS)} A business model in which threat actors lease or purchase ransomware variants from ransomware developers in the same way that organizations lease SaaS products from legitimate software developers. RaaS has grown in popularity in recent years.

\runinhead{Read-Only Domain Controller (RODC)} RODC servers are Domain Controllers that hold a read-only copy of the Active Directory database and do not allow changes to AD. An RODC is typically deployed in locations that require quick access to AD services but are not physically secure enough to host a writable domain controller. While an RODC can authenticate user logins, changes are not written directly to it, but rather to a writable domain controller, then replicated back to the RODC.

\runinhead{Read-Write Domain Controller (RWDC)} Writeable domain controllers can be used to update objects in Active Directory. In Active Directory, all domain controllers are writeable, unless they are a Read-Only Domain Controllers (RODC).

\runinhead{Recovery Point Objective (RPO)} A recovery point objective (RPO) sets a limit on how old the data can be before backing up (e.g., 24 hours old). 
\textit{See also: Recovery Time Objective (RTO)}

\runinhead{Recovery Time Objective (RTO)} A recovery time objective (RTO) sets a limit on the amount of time that an application, system, or process can be unavailable (e.g., no more than 2 hours).
\textit{See also: Recovery Point Objective (RPO)}

\runinhead{Recursive DNS Query} A recursive DNS query is a type of DNS query in which the DNS resolver or server attempts to resolve the query by iteratively querying other DNS servers until it obtains the final answer. Clients commonly use this type of query to resolve hostnames to IP addresses.

\runinhead{Recycle Bin} First introduced as an optional feature in Windows Server 2008 R2, this feature creates a new hidden container in the domain tree and stores deleted objects for a specified number of days before permanently removing them, allowing the option to restore them without loss of object attribute values. This feature can be enabled and accessed through the Active Directory Administrative Center (ADAC) console.

\runinhead{Recycled Object} When an object has been deleted, it remains in a hidden container in the domain tree until a configured period of time (that is, the lifetime of the tombstone) has passed, after which the object is permanently removed from storage. These objects exist only when the optional Recycle Bin feature is enabled.

\runinhead{Red Forest} Red Forest, also known as Enhanced Security Admin Environment (ESAE), was a Microsoft security concept in which all your administrative credentials resided in a separate AD forest, trusted by your production AD forests. The approach aimed to remove admin credentials from AD forests and thus improve security. The concept has been retired.

\runinhead{Red Team} In cybersecurity testing, the red team is the group of individuals responsible for attacking cybersecurity defenses of an organization, exploiting system vulnerabilities, and helping to identify counterattack methods for defenders (e.g., the blue team).
\textit{See also: Blue Team}

\runinhead{Registry} The Windows Registry stores configuration settings and options on Microsoft Windows operating systems, including those related to AD. An attacker gaining access to the Registry could change these settings to disrupt AD functionality or increase their privileges.

\runinhead{Relative Distinguished Name (RDN)} The name of an object relative to its parent. This is the leftmost attribute-value pair in the distinguished name (DN) of an object. For example, in the \texttt{DN “cn=Alice Malice,OU=Company Users,DC=hackherway,DC=com”}, the RDN is \texttt{ 'CN = Alice Malice'}

\runinhead{Relative identifier (RID)} A RID is a unique identifier assigned to each object in an AD domain. It is combined with a domain identifier to form a security identifier (SID) for the object. The RID master, one of the FSMO roles, is responsible for processing RID pool requests from all domain controllers in a particular domain.

\runinhead{Relative Identifier (RID) Master} The RID Master is a FSMO role responsible for managing the allocation of Relative Identifiers (RIDs) to domain controllers. These RIDs are used in conjunction with the domain SID to create unique Security Identifiers (SIDs) for all objects within the domain. The RID Master ensures that every user, group, and other object in AD has a unique ID, preventing conflicts.

\runinhead{Remote Access} Remote access refers to methods that allow users to access an AD network from a remote location. Insecure remote access points can be exploited by attackers for unauthorized access to the network.

\runinhead{Remote Desktop Protocol (RDP)} RDP, or Remote Desktop Protocol, is a secure network communications protocol developed by Microsoft that allows users to remotely access and control another computer or virtual machine over a network connection. It enables users to interact with a remote computer as if they were physically present, viewing its desktop, and using its applications and resources.

\runinhead{Remote Desktop Services (RDS)} RDS, formerly known as Terminal Services, allows a user to take control of a remote computer or virtual machine over a network connection.
\textit{See also: Terminal Services}

\runinhead{Remote Management} Remote management refers to the management of computer systems from a remote location. This is especially relevant for administering Domain Controllers spread across various geographical locations. In the context of Active Directory, tools like Remote Server Administration Tools (RSAT) allow administrators to manage roles and features remotely.
\textit{See also: Remote Server Administration Tools (RSAT)}

\runinhead{Remote Procedure Call (RPC)} This is a communication protocol that client programs use to request a service from a program on another computer in the network. It is widely used for AD replication and administration; however, it can be exploited for lateral movement if not properly secured.

\runinhead{Remote Server Administration Tools (RSAT)} RSAT is a suite of Microsoft tools that allows administrators to remotely manage Windows Server roles and features from a client machine, such as Windows 10 or 11. In an Active Directory environment, RSAT includes utilities such as Active Directory Users and Computers (ADUC), Group Policy Management Console (GPMC), and DNS management tools. RSAT was created to give administrators centralized remote access to manage AD resources without needing to log directly onto domain controllers; however, because RSAT provides powerful administrative capabilities, attackers who compromise an account with RSAT privileges can manipulate Active Directory objects, group policies, and domain configurations-making it a critical security concern for defenders and security teams alike.

\runinhead{\texttt{Remove-ADOrganizationalUnit}} A PowerShell cmdlet used to delete an Organizational Unit (OU) from Active Directory. OUs are containers that store and organize AD objects often with Group Policy Objects (GPOs) linked to them. This command provides administrators with a direct and automated way to clean up or restructure AD environments. From a security perspective, misuse of this cmdlet-whether intentional or accidental-can result in the deletion of critical AD structures, breaking inheritance, or removing or tampering with security policies; therefore, access to this command should be restricted to trusted admins and monitored closely.

\runinhead{\texttt{Remove-ADUser}} A PowerShell command that removes an AD user. If used maliciously, it can lead to the deletion of legitimate user accounts, potentially causing operational disruptions or data loss.

\runinhead{\texttt{Repadmin}} A command line utility to diagnose Active Directory replication between domain controllers.

\runinhead{Replica} A copy of an Active Directory namespace (NC or naming context) on a domain controller that replicates with other domain controllers within the AD forest.

\runinhead{Replication} Replication in Active Directory refers to the process of copying data from one domain controller to another. This process ensures that each domain controller has the same information as other domain controllers and enables the distribution of the AD database across multiple servers. If replication fails or is delayed, it can lead to inconsistencies, known as replication errors.

\runinhead{Replication Bridgehead Server} A Replication Bridgehead Server is a domain controller designated to manage replication traffic within a site in Active Directory. It acts as a central point for receiving and sending replication data between sites, reducing replication traffic over wide-area networks (WANs).

\runinhead{Replication interval} The replication interval in Active Directory defines the time duration between two consecutive replication cycles between domain controllers. Ensure that changes are propagated efficiently across the network without causing excessive replication traffic.

\runinhead{Replication Latency} Replication latency is the time lag between the final update originating in an AD object and all replicas.

\runinhead{Replication Metadata} Replication metadata in AD is used to resolve conflicts that may occur during replication. Metadata keeps track of which changes have already been applied to prevent old changes from overwriting more recent ones.

\runinhead{Replication Rights} In Active Directory, replication rights refer to permissions that allow a user or application to request and receive information from other domain controllers, including potentially sensitive data such as password hashes. These rights are essential for legitimate directory replication, but can be exploited by attackers.

\runinhead{Replication Topology} Replication topology refers to the structure of connections between domain controllers to replicate data in Active Directory. Describes how changes are propagated across the network to ensure data consistency.

\runinhead{\texttt{Reset-ComputerMachinePassword}} A PowerShell command that resets the machine account password for the computer. If misused, it can disrupt the secure channel between the computer and its domain, leading to a potential denial of service or unauthorized access.

\runinhead{Resilient File System (ReFS)} ReFS is a file system developed by Microsoft for use on Windows operating systems and is designed to overcome some of the limitations of NTFS.

\runinhead{Responder} A network tool that can manipulate network communications and respond to network broadcast requests, Responder is often used in man-in-the-middle attacks, helping attackers intercept and manipulate traffic.

\runinhead{Resultant Set of Policy (RSOP)} A tool in Windows that administrators use to determine the combined effect of Group Policies applied to a system and/or user. Essentially, it provides a cumulative view of all policies from various sources that apply to a specific user or system. This tool can be particularly useful in troubleshooting scenarios, when an administrator needs to understand why a certain policy is or is not taking effect.

\runinhead{REvil (Ransomware Evil)} A Russia-based or Russian-speaking private ransomware as a service (RaaS) operation. REvil (also known as Sodinokibi) ransomware often spreads via brute-force attacks and server exploits, but it can also spread via malicious links and phishing. Cyberattackers can use REvil to exploit Active Directory (AD) misconfigurations or weak passwords to spread across the network.

\runinhead{Revocation} This is the process of invalidating a certificate, a process that can be managed through AD’s Certificate Services. Proper certificate revocation is important to prevent unauthorized use of a certificate in a Man-in-the-Middle attack or other types of cyberattack.

\runinhead{RID Hijacking} In this type of attack, the relative identifier (RID) of a standard domain account is modified to match the RID of a domain admin account, effectively promoting the standard account to the domain admin status.

\runinhead{RID Master} One of the five FSMO (Flexible Single Master Operations) roles in AD, the RID Master is responsible for processing RID pool requests from all DCs within the domain. When objects such as users and computers are created in Active Directory, they are assigned a unique security ID (SID) and a relative ID (RID). The RID master role ensures that no objects in AD are assigned the same SID and RIDs. Any failure of the RID Master role could affect the creation of new objects within the domain. A domain controller in each domain must play this role.
\textit{See also: Flexible Single Master Operations (FSMO) Roles}

\runinhead{Rivest Cipher 4 (RC4)} RC4, also known as Rivest Cipher 4, is a stream cipher algorithm that encrypts data one byte at a time. It was created by Ron Rivest in 1987 and is known for its speed and simplicity. While once widely used in protocols like SSL/TLS and WEP, RC4 is now considered cryptographically weak to support today's current encryption needs and requirements and is no longer recommended for use.

\runinhead{Rivest Cipher 5 (RC5)} In the context of cryptography, RC5 stands for "Rivest Cipher 5," named after its designer Ronald Rivest, an acclaimed cryptographer. The "RC" part can also informally refer to "Ron's Code," and the "5" indicates that it was the fifth cipher developed by Rivest. A symmetric block cipher developed by Ronald Rivest. It is parameterizable: the block size, key size, and number of rounds can be adjusted. Offers better flexibility and performance, but is rarely used in modern systems.

\runinhead{Rivest-Shamir-Adleman (RSA)} RSA is one of the most widely used public-key cryptographic algorithms, named after its inventors Ron Rivest, Adi Shamir, and Leonard Adleman. It enables secure key exchange, digital signatures, and encryption using a pair of keys: a public key for encryption and a private key for decryption. Within Active Directory environments, RSA is fundamental for services like Kerberos authentication, SSL/TLS communications, and Active Directory Certificate Services (AD CS). Because RSA-protected certificates can be leveraged for user and machine authentication, attackers who obtain private keys or misconfigured certificates can bypass password-based protections and impersonate privileged accounts, making RSA central to both AD security and potential abuse.
 
\runinhead{Role-Based Access Control (RBAC)} Role-Based Access Control (RBAC) is a method of restricting system access based on predefined job roles within an organization. Instead of assigning permissions directly to individual users, RBAC assigns them to roles (such as 'Domain Administrator' or 'Helpdesk Technician'), and users inherit those permissions by being added to the role. In Active Directory, RBAC is implemented through security groups, Organizational Unit (OU) delegation, and administrative templates, ensuring that access to sensitive systems is consistent and based on least privilege. Misconfigured RBAC in AD, such as granting overly broad group memberships or failing to separate Tier 0 accounts from Tier 1/2 operations, can result in privilege escalation paths that attackers exploit to gain control of the domain.

\runinhead{Root Directory Service Entry (RootDSE)} The Root Directory Service Entry (RootDSE) is a special LDAP object that exists at the root of the Active Directory forest or domain. It is not tied to a specific domain partition, but instead provides clients with information about the AD environment, such as supported LDAP features, naming contexts, and the functional level of the domain or forest. For administrators, querying RootDSE can reveal useful metadata about an AD deployment. However, for attackers, RootDSE is an attractive reconnaissance tool, as it does not require authentication to query. By issuing anonymous LDAP queries, an adversary can gather details about the AD schema, naming contexts, and capabilities without needing valid credentials, making RootDSE a legitimate and exploitable feature of Active Directory.

\runinhead{Root Domain} The root domain is the first domain created in an Active Directory forest and serves as the foundation for the entire forest structure. It sits at the top of the forest DNS namespace and holds critical roles such as the Enterprise Admins and Schema Admins groups, which have the highest level of privilege across all domains in the forest. Because of this, the root domain is considered a Tier 0 asset, which means that its compromise equates to the compromise of the entire forest. Attackers who escalate privileges to the root domain can manipulate trust relationships, modify the schema, and ultimately control every child domain in the forest, which is why protecting root domain controllers and accounts is a top security priority.

\runinhead{Routing and Remote Access Service (RRAS)} Routing and Remote Access Service (RRAS) is a Windows Server role that provides routing, Virtual Private Network (VPN) and remote access capabilities within a network. In Active Directory environments, RRAS is often used to allow secure remote connectivity for users or to connect branch offices. However, if improperly configured, RRAS can introduce vulnerabilities such as weak authentication, insecure tunneling, or exposed administrative interfaces that attackers can exploit to gain access to the network. Since RRAS often interacts with Active Directory for authentication, a compromised RRAS server could provide an attacker with valid credentials, remote access, or even a direct path into the domain, making it essential to harden and monitor RRAS configurations.

\runinhead{Ryuk} A type of ransomware that is known to target large, public-entity Windows systems. Ryuk encrypts computer files, system access, and data, making it impossible for users to retrieve information or access programs. This attack has an unusual aspect for Active Directory (AD): the ransomware is pushed to unsuspecting users via AD Group Policy Objects (GPO).

\section*{S}
\runinhead{Salting} To defend against rainbow table attacks, it is common to use a method called salting the hashes, where a random value is added to the password before it is hashed. This makes rainbow tables far less effective because even a small change in the input (such as adding a 'salt') results in a drastically different hash. It is worth noting that protecting your password hashes is a crucial aspect of securing an Active Directory environment.

\runinhead{\texttt{sAMAccountName}} The users’ logon name that is used to support clients and servers running earlier versions of Windows. Also called the 'Pre-Windows 2000 logon name'.

\runinhead{SaveTheQueen} In a SaveTheQueen attack, cyberattackers target AD through a strain of ransomware that uses the SYSVOL share on AD domain controllers to propagate throughout the environment. Accessing the SYSVOL share, which is used to deliver policy and log-in scripts to domain members, typically requires elevated privileges and indicates a serious AD compromise.

\runinhead{Schema} The Active Directory schema defines every object class that can be created and used in the Active Directory forest. By default and out of the box, the schema defines every attribute that can exist in an object, the relationships between the various attributes, which one is mandatory, what permissions each one has, and many other parameters. The schema is basically a blueprint or template of how data and what type of data can be stored in Active Directory. From a cyber security perspective, unauthorized changes to the schema can be dangerous because they cannot be undone and may have a destructive effect on the Active Directory data.

\runinhead{Schema Admins Group} The Schema Admins group in Active Directory is a highly privileged group that has full control over the Active Directory schema. Members of this group can modify the schema, including adding or deleting attribute and object class definitions. For example: The Schema Admins group would typically include IT administrators responsible for managing the Active Directory schema, such as implementing schema extensions or performing schema updates.

\runinhead{Schema Extensions} Schema extensions in Active Directory refer to the process of modifying or adding new attribute and object class definitions to the existing Active Directory schema. This enables organizations to customize the schema to accommodate specific application requirements or store additional attributes for objects. For example: An organization may extend the Active Directory schema to include custom attributes for user objects to store additional information such as employee ID numbers or department names.

\runinhead{Schema Master} This is one of the five FSMO (Flexible Single Master Operations) roles in AD, responsible for handling all changes to the Active Directory schema. Unauthorized access to Schema Master can lead to critical changes in the AD structure, enabling severe attacks. A domain controller in the entire forest must hold this role.
\textit{See also: Flexible Single Master Operations (FSMO) Roles}

\runinhead{Schema Object} An object that defines an attribute or an object class.

\runinhead{Secure LDAP (LDAPS)} A version of the LDAP protocol that establishes a secure connection to the LDAP server by applying SSL/TLS. It is crucial to protect the exchange of information between clients and domain controllers from eavesdropping or manipulation.

\runinhead{Secure Sockets Layer (SSL)} A standard protocol used in various protocols such as HTTPS, LDAPS, and others that supports the confidentiality and integrity of messages in client and server applications that communicate over open networks. SSL supports server and, optionally, client authentication using X.509 certificates. SSL is superseded by Transport Layer Security (TLS). TLS version 1.0 is based on SSL version 3.0.

\runinhead{Security Account Manager (SAM)} The SAM is a database that stores users and group objects used by Windows client operating systems to authenticate local users. The user passwords are stored in a hashed format. SAM uses cryptographic measures to prevent unauthenticated users from accessing the system. If an attacker gains access to this database, they can attempt to extract details about the user accounts and their passwords.

\runinhead{Security Assertion Markup Language (SAML)} An XML-based open standard for exchanging authentication and authorization data between parties, specifically between an identity provider and a service provider. SAML is critical in scenarios where secure and seamless single-sign-on (SSO) is needed across different domains. It helps organizations provide a smooth user experience, reducing the need for multiple passwords and log-ins, while maintaining high levels of security and control over user access.

\runinhead{Security Descriptor} In Active Directory, a security descriptor is a data structure that contains the security information associated with a secure object such as users, groups, or computers. Includes owner SID, primary group SID, DACL, and SACL. A security descriptor defines who has what type of access to the object. Misconfiguration in a security descriptor can lead to unauthorized access or privilege escalation.

\runinhead{Security Descriptor Propagator (SDProp)} A process within Active Directory that maintains the consistency of access control lists (ACLs) throughout the directory. It operates at the domain level and ensures that permission changes made to a parent object are properly propagated to all child objects within the hierarchy. Thus, it plays a vital role in the enforcement of security and access control policies across AD infrastructure.

\runinhead{Security Event Log} A collection of Windows logs that capture a range of security-related information, including logon attempts and resource access, which can be invaluable in detecting suspicious activity. If not monitored regularly, malicious activities can go unnoticed, leading to breaches.

\runinhead{Security Group} Security groups can contain multiple accounts such as user objects, computer objects or even other group objects that can be used to easily assign permissions to a resource or apply for permissions. Security groups play a crucial role in securing resources and managing access rights within an Active Directory environment as they are used to apply the permissions to a folder or object to the group instead of every individual account.

\runinhead{Security Identifier (SID)} A SID is a unique value used to identify user, group, and computer accounts in Windows. They play an essential role in managing the permissions and controlling access to resources in an Active Directory environment. if an attacker is able to forge an SID, they might impersonate another user or gain unauthorized privileges. The SID value for all objects in a domain is identical. To create a unique value for security principals, the SID value is combined with a unique RID value that is controlled by the RID pool assigned to DCs by the FSMO role holder of the RID Master in the domain.

\runinhead{Security Indicator} Security indicators are values based on metrics obtained by comparing logically related attributes about the behavior of an activity, process, or control in a specified time. These critical indicators are derived from predefined criteria and may be predictive of the overall security posture of an organization. Security indicators include indicators of attack (IOAs), indicators of compromise (IOCs), and indicators of exposure (IOEs).
\textit{See also: Indicators of Attack (IOA); Indicators of Compromise (IOC); Indicators of Exposure (IOE)}

\runinhead{Security Information and Event Management (SIEM)} A technology that aggregates and analyzes log and event data generated by various sources in the IT infrastructure of an organization. Organizations use SIEM to gather, centralize and store logs from various sources, in real time.

\runinhead{Security Operations Center (SOC)} A SOC is a unit that operates as the cybersecurity hub of an organization, tasked with strengthening security measures and responding to threats in real time. Monitors various systems including identities, endpoints, servers, and databases, while also leveraging up-to-date threat intelligence to identify and rectify vulnerabilities before they can be exploited by cyber attackers.

\runinhead{Security Orchestration, Automation, and Response (SOAR)} A suite of solutions designed to streamline security operations by automating threat detection and response. SOAR integrates various security tools and systems, providing a unified platform to collect data and execute appropriate responses to threats. This allows security teams to manage and respond to a larger volume of threats more efficiently, improving the overall security posture of an organization.

\runinhead{Security Perimeter} With the advent of cloud services, mobile devices, and remote work, the security perimeters of organizations have changed from the servers on site that make up a network to a new frontier: identity. 

\runinhead{Security Principal} A unique entity, also called a principal, that can be authenticated by Active Directory. Typically a user object, a security group object, or a computer object. All security principals in Active Directory have a Security ID (SID).

\runinhead{Security Support Provider Interface (SSPI)} SSPI allows applications to use various security models available on a computer or network without changing the interface to the security system. Its misuse can lead to token manipulation attacks.

\runinhead{Security Token Service (STS)} STS is a software-based identity provider that issues security tokens as part of a claims-based identity system. It is commonly used in federation scenarios, playing a critical role in ensuring security. If compromised, it can lead to unauthorized access to multiple services.

\runinhead{Server Core} Server Core is a minimal server installation option for Windows Server that provides a low-maintenance server environment with limited functionality. It is mainly used for infrastructure roles, including an Active Directory Domain Services role. From a cybersecurity perspective, the smaller attack surface of the Server Core can reduce the potential risk of security vulnerabilities.

\runinhead{Server Message Block (SMB)} SMB is a protocol for sharing files, printers, serial ports, and communications abstractions such as named pipes and mail slots between computers. An example of a cybersecurity issue is the SMB Relay attack where an attacker sets up an SMB server and gets the target machine to authenticate to it, allowing for credential theft or execution of arbitrary code.

\runinhead{Service Account} A user account that is created explicitly to provide a security context for services running on Windows Servers. Mismanagement of service accounts can expose them to attackers, allowing lateral movement or escalation of privileges.

\runinhead{Service Connection Point (SCP) Object} The Service Connection Point (SCP) object in Active Directory is used to define the configuration information that clients or services need to find and connect to specific services or resources within the organization’s infrastructure. For example: Microsoft Exchange creates SCP objects to specify the endpoint of the Autodiscover service in Active Directory. This allows Outlook clients to automatically discover the Exchange settings and connect to the appropriate Exchange server.

\runinhead{Service Principal Name (SPN)} An SPN is a unique identifier tied to each instance of a Windows service. SPNs are used in conjunction with the Kerberos authentication protocol to associate a service instance with a service logon account. In a cyberattack known as Kerberoasting, an attacker may request Kerberos tickets for SPNs to crack their passwords offline.
\textit{See also: Kerberoasting; Service Principal Name (SPN) Scanning}

\runinhead{Service Principal Name (SPN) Scanning} A method often used in attacks to discover service accounts in an Active Directory environment. Attackers enumerate services running under domain accounts via exposed Service Principal Names (SPNs). These accounts often have elevated privileges and weaker passwords, making them prime targets for compromise.

\runinhead{Session Hijacking} An attack where an attacker takes over a user session. In the context of Active Directory, this could include taking over a Kerberos ticket-granting ticket session, for instance.

\runinhead{\texttt{Set-ADAccountExpiration}} This PowerShell cmdlet sets the expiration date for an AD account. If improperly used, it could lead to denial of service by setting an immediate expiration on valid user accounts.

\runinhead{\texttt{Set-ADAccountLockout}} This PowerShell command sets the account lockout protection for an AD user.

\runinhead{\texttt{Set-ADAccountPassword -LogonWorkstations}} This PowerShell command restricts logon to specific workstations for a user. If used maliciously, it can lead to unauthorized access or a possible denial of service.

\runinhead{\texttt{Set-ADComputer}} This PowerShell cmdlet is used to modify the properties of a computer object. If used maliciously, it can lead to unauthorized changes to computer properties, potentially leading to operational disruptions or breaches.

\runinhead{\texttt{Set-ADDefaultDomainPasswordPolicy}} This PowerShell cmdlet is used to modify the default password policy for an AD domain. If misused, it can weaken the organization’s password policy, making it easier for an attacker to guess or crack passwords.

\runinhead{\texttt{Set-ADUser}} This PowerShell cmdlet is used to modify the attributes of a user object in Active Directory. In the wrong hands, this command could be used maliciously to alter user account properties, such as the description field, for stealthy persistence or privilege escalation.

\runinhead{Shadow Copy} Shadow Copy is a technology in Windows systems that allows to take manual or automatic backup copies or snapshots of data, even if it is in use. It can be used to restore previous versions of files and directories.

\runinhead{Shadow Credentials Attack} When users share a username and password between multiple systems (including non-AD systems), an attacker can use weak or breached credentials from one system to gain unauthorized access to another.

\runinhead{Shadow Group} Shadow groups in Active Directory are used to mirror the membership of a given dynamic distribution group. This can be especially useful when permissions need to be assigned to a dynamic distribution group; however, if not properly managed, shadow groups can pose a security risk through inadvertent granting of permissions.

\runinhead{SID Filtering} A security mechanism in Active Directory that removes foreign SIDs from a user’s access token when accessing resources through Forest Trust. This feature, which is enabled by default between forests, helps protect against malicious users with administrative privileges in a trusted forest from gaining control over a trusted forest. When SID filtering is active, only SIDs from the trusted domain are used on a user's token, while SIDs from other trusted domains are excluded.

\runinhead{SID History} SID History is an attribute of a user object, which assists in the migration of resources from one domain to another. It stores former SIDs of a user account, allowing access to resources that recognize the old SID. This can be abused by attackers using a method called SID History Injection to escalate privileges.

\runinhead{SID History Attack} Security Identifier (SID) history attribute can be manipulated to increase the privileges of a user. An attacker can add the SID of a privileged group to the SID history attribute of their account, granting them the corresponding privileges.

\runinhead{Silver Ticket Attack} A Silver Ticket attack focuses on forging a session ticket (ST). This attack enables the attacker to impersonate a legitimate user and gain unauthorized access to resources within the domain. In this type of attack, an attacker gains unauthorized access to a service by forging a Kerberos ticket for that service. By obtaining the NTLM hash of the service account and other required information, the attacker can create a malicious ticket, granting them access to the service without having to authenticate or know the actual user’s password.

\runinhead{Simple Authentication and Security Layer (SASL)} A framework that provides a mechanism for authentication and optional security services in internet protocols. In the context of Active Directory (AD), SASL is used to ensure the integrity and security of data during transmission. When an AD client wants to authenticate with a server, it can use SASL to specify the method of authentication it prefers. SASL supports several authentication methods, such as Kerberos, NTLM, and Digest-MD5, and is often used in protocols such as LDAP. Some SASL mechanisms also provide security services beyond authentication, such as data integrity checks and encryption, to protect data during transmission.

\runinhead{Single Sign-On (SSO)} SSO is an authentication process that allows a user to access multiple resources with a set of login credentials. SSO can be used to provide users with seamless access to network resources, improving the user experience by reducing the number of passwords they need to remember. SSO can increase business efficiency, but also presents a potential security issue if the single authentication point is compromised.

\runinhead{Site} A collection of one or more well-connected (reliable and fast) TCP/IP subnets represented as objects in the AD database. Sites help administrators optimize both Active Directory logon traffic and Active Directory replication with respect to the physical network and WAN connection speeds. When users log in, domain member machines find domain controllers (DCs) that are on the same site as the user, or near the same site if there is no DC in the site. When DCs replicate, they will perform an almost immediate replication between all DCs in a site and postpone the replication traffic to other sites based on the replication window and interval. Misconfiguring sites and services can lead to replication inefficiencies and can impact the availability of AD services.

\runinhead{Site Link} Site links in Active Directory represent reliable IP paths between sites. They are used by the Knowledge Consistency Checker (KCC) to build the replication topology.

\runinhead{Site Object} An object of the class site, representing a site.

\runinhead{Skeleton Key Attack} In a skeleton key attack, an adversary deploys a malicious piece of software on a domain controller. This malware enables the attacker to log into any account within the domain using a password known only to the attacker, without disrupting normal operations or changing actual passwords.

\runinhead{Smart Card Authentication} This is a strong form of two-factor authentication that is used to log into an AD domain. A smart card contains a certificate associated with a user account, providing a robust defense against credential theft.

\runinhead{\texttt{SMBExec}} A rapid psexec style attack with the added benefit of pass-the-hash capability. It runs commands on remote computers using the SMB protocol.

\runinhead{Smbmap} A tool that allows users to enumerate samba share drives across an entire domain. Attackers can use it to find out about file, directory, and share-level permissions.

\runinhead{Snapshot} This refers to a 'photo' or a stored state of the Active Directory database at a given time, which can be used for backup purposes. Unauthorized access to snapshots can reveal sensitive data, and outdated snapshots can contain vulnerabilities that have been patched in the live environment.

\runinhead{Software Restriction Policies (SRP)} SRPs identify software programs running on computers in a domain and control the ability of the programs to run. This is an effective method to prevent the execution of malicious software or software that is not trusted. SRP can be employed to establish a highly secure configuration for systems by permitting the execution of only pre-approved applications. Integrated with Microsoft Active Directory and Group Policy, SRPs can be created in both network environments and on independent computers.

\runinhead{SolarWinds Attack} In early 2020, cybercriminals secretly broke into the systems of Texas-based SolarWinds and added malicious code to the company's software system. The system, called “Orion”, is widely used by companies to manage IT resources. SolarWinds had 33,000 customers using Orion, according to SEC documents; however, around March 2020, up to 18,000 SolarWinds customers installed updates that left them vulnerable to cyberattackers. Among the others were several high-profile SolarWinds clients, including Fortune 500 companies and multiple agencies in the US government, including parts of the Pentagon, the Department of Homeland Security, and the Treasury.

\runinhead{Special Identity} Special identities (also called implicit identities) are predefined groups that serve unique, often dynamic roles within the infrastructure. Unlike typical groups, these identities do not have a static list of members. Instead, they represent different users under different circumstances. Some examples of these special identities include Anonymous Login, Batch, and Authenticated User.

\runinhead{SPN-Jacking (SPN Hijacking)} In an SPN-jacking attack, cyberattackers manipulate the SPN of computer/service accounts to redirect pre-configured Constrained Delegation to unintended targets, even without obtaining \texttt{SeEnableDelegation} privileges.

\runinhead{Spoofing} In the context of Active Directory, spoofing generally refers to a situation where a malicious entity impersonates another device or user on the network. For example, DNS spoofing might be used to divert traffic to a malicious server.

\runinhead{SRV Record} In DNS, a type of information record stored in the zone database on DNS, which maps the name of a particular service to the DNS name of a server that offers that service. Active Directory uses SRV records heavily to allow clients and other DCs to locate services such as Global Catalog, LDAP, and Kerberos, and DCs automatically advertise their capabilities by publishing SRV records in DNS.

\runinhead{SSL/TLS Handshake} The process of negotiating and establishing a connection protected by the Secure Sockets Layer (SSL) or Transport Layer Security (TLS).

\runinhead{Supply Chain Attacks} Also known as third-party attacks or value chain attacks, they occur when an attacker infiltrates your system through an outside partner or provider with access to your systems and data. Essentially, instead of attacking the primary target directly, the threat actor targets less secure elements in the network supply chain. Notorious examples include the SolarWinds attack, where malicious code was inserted into software updates, affecting thousands of customers worldwide.

\runinhead{Symmetric Encryption} A cryptographic method in which the same key is used for both encryption and decryption. It is efficient and fast, but requires a secure key distribution between parties. Common algorithms: AES, DES, blowfish.

\runinhead{Synchronization} In the context of Active Directory, synchronization is the process of ensuring that multiple copies of a data object, such as a user account or group, are the same across all domain controllers. If synchronization fails, it can lead to inconsistencies that might be exploited by an attacker.

\runinhead{System Access Control List (SACL)} An access control list (ACL) that controls the generation of audit messages for attempts to access a securable object in AD. The resulting audit messages can be seen in the security log in the Windows Event Viewer. Ignoring SACLs can leave a system vulnerable by failing to record and alert about suspicious activities.

\runinhead{\texttt{SYSVOL}} The \texttt{SYSVOL} share is a very important folder that is shared on each domain controller in the AD domain. The default location is \\\texttt{\%SYSTEMROOT\%SYSVOL} and it typically contains Group Policy Objects, Folders, Scripts, Junction Points and more. Each Domain Controller gets a replica of the \texttt{SYSVOL} share. If not properly secured, attackers may gain access to this share and whatever is placed there is replicated by default throughout the AD forest. Unauthorized modifications in \texttt{SYSVOL} can also lead to GPO-related security issues.

\runinhead{Schema Master} The Schema Master is a flexible single master operation role responsible for managing updates to the Active Directory schema. This schema acts as a blueprint, defining the types of objects (like users, computers, groups) and their attributes (like name, email, password) that exist within an Active Directory forest. Essentially, it is the "master copy" of the schema, and only one Schema Master exists per forest.

\runinhead{Server Message Block (SMB)} SMB is a network file sharing protocol that allows computers to access files and other resources on a network as if they were local. It is commonly used in Windows environments for file and printer sharing and is also implemented in Linux and Unix systems via the Samba protocol.

\runinhead{Shell} A shell refers to a program that provides a user interface for interacting with an operating system's services. It allows users to execute commands and manage system resources via a command-line interface or terminal. \textit{"Web shells"} are malicious scripts that give attackers remote access to a web server, often used to compromise other systems remotely.

\runinhead{Silver Ticket} Similar to the case of a golden ticket attack, a silver ticket attack focuses on specific services and portions of the AD network within the domain rather than the entire domain.

\runinhead{Structured Query Language (SQL)} Pronounced as \textit{"SEQUEL,"} Microsoft SQL Server (MSSQL) is a \textit{Relational Database Management System (RDBMS)} developed by Microsoft. It is a software product that acts as a server, storing and retrieving data as requested by other software applications. It uses the Structured Query Language to manage and query the data stored in its databases.

\section*{T}
\runinhead{Tactics, Techniques, and Procedures (TTPs)} TTPs describe the behavior and methodology of threat actors. \textit{\textbf{Tactics}} refer to the general objective (e.g., privilege escalation). \textbf{\textit{Techniques}} describe how that objective is achieved (e.g. Pass-the-Hash (PtH). \textbf{\textit{Procedures}} are specific implementations or variations used by an attacker. TTPs are used for threat intelligence, modeling adversary behavior, and aligning with frameworks like the MITRE ATT\&CK framework.

\runinhead{TCPView}  Sysinternals graphic network monitoring utility that shows a representation of all currently active TCP and UDP endpoints on a system.

\runinhead{Terminal Access Controller Access-Control System (TACACS)} TACACS is a remote authentication protocol commonly used in UNIX networks. TACACS allows a remote access server to forward a user's log-in password to an authentication server to determine whether access can be allowed to a given system. Although not directly part of Active Directory, it is often used in conjunction with AD in mixed environments.

\runinhead{Terminal Services} Terminal Services, now known as Remote Desktop Services, allow users to access Windows-based programs or the full Windows desktop remotely. Although it provides convenience and flexibility, it can also pose a security risk if not properly secured, as attackers could exploit it for unauthorized remote access.
\textit{See also: Remote Desktop Services (RDS); Remote Desktop Protocol (RDP)}

\runinhead{Test-ComputerSecureChannel} This PowerShell command checks the secure channel between the local computer and its domain. If it shows an insecure connection, it could indicate possible MiTM attacks or other network compromises.

\runinhead{Threat Hunting} Threat hunting is a proactive cybersecurity process of searching through networks to detect and isolate advanced threats that evade existing security solutions. It is crucial in Active Directory environments to identify potential intrusions or malicious activities that have bypassed traditional security measures. A well-known example of a threat hunting case in AD would be to look for signs of 'Golden Ticket' attacks, where attackers forge a TGT.

\runinhead{Threat Landscape} An umbrella term to describe the types of vulnerabilities, attacks, and threat actors that exist at any given time, within a certain context. Computer and information technologies are advancing at lightning speed; however, cyberattackers are keeping pace by constantly evolving their methods of exploiting system vulnerabilities. The volatility in today’s cyber threat landscape makes it critical to use a layered security approach and solutions designed specifically to protect and quickly recover Active Directory.

\runinhead{Ticket Granting Service (TGS)} TGS is a critical component of the Kerberos authentication protocol used in Active Directory. After initial authentication, the Key Distribution Center (KDC) issues a ticket grant (TGT). The TGS uses this TGT to issue service tickets for access to other resources within the domain. If an attacker gains access to a valid TGT, they can request tickets to any network service, leading to potential unauthorized access.
\textit{See also: Ticket Granting Ticket (TGT)}

\runinhead{Ticket Options} In the Kerberos authentication protocol, Ticket Options is a field in the ticket that specifies flags such as whether the ticket is renewable or if it is valid for proxy use. Misconfiguration of these options or exploitation by an attacker can lead to security issues, such as unauthorized ticket renewal.

\runinhead{Ticket Granting Ticket (TGT)} A Ticket Granting Ticket (TGT) is a security token issued by the Key Distribution Center (KDC) as part of the Kerberos authentication protocol in Active Directory. The TGT is created after a user successfully authenticates and acts as a "passport" for accessing network resources. It allows users to request service tickets from the Ticket Granting Service (TGS) without having to re-enter credentials. A TGT contains information such as the session key, expiration date, and client IP address. TGT is meant to streamline secure authentication across the network; however, it is also a high value target for attackers, since possession of a valid TGT enables impersonation attacks and unauthorized access to restricted resources.

\runinhead{Tier 0} Tier 0 assets are those that are critical to the operation of your IT environment. Such assets include Active Directory and AD domain controllers, which, in turn, control access and privileges to every user, system, and resource in the organization.

\runinhead{Tiered Administrative Model} This security model for administrative access segregates privileges into separate tiers to prevent credential theft and unauthorized access. For instance, admins with access to the domain controller (Tier 0) should not use the same accounts or machines to manage less trusted assets like user workstations (Tier 2). This model is essential in minimizing the risk of privilege escalation or lateral movement attacks.

\runinhead{Time Synchronization} Active Directory uses the Kerberos protocol for authentication that relies on time sensitive tickets. It is crucial to maintain accurate and synchronized time across all systems in an AD environment to prevent authentication issues. Incorrect time settings can even lead to Kerberos-based attacks, such as replay attacks.

\runinhead{Time-To-Live (TTL)} TTL is not unique to Active Directory, but it plays a critical role in DNS, which is a significant component of AD. In the context of AD-integrated DNS, TTL is a value in a DNS record that signifies the duration for which the record is valid before it needs to be refreshed. An excessively long TTL might lead to the use of outdated DNS information, which could disrupt AD services.

\runinhead{Token} A token in Active Directory is a representation of user rights and permissions. Every time a user logs in, a token is generated that identifies the user and the groups to which the user belongs. Tokens can be targeted by cybercriminals to perform token impersonation attacks, stealing the token to gain unauthorized access.

\runinhead{Token Bloat} Token bloat is a condition in which a user accumulates so many security identifiers (SIDs) in their access token due to membership in many groups that they experience logon or resource access issues. From a cybersecurity point of view, token bloat can impact user productivity and serve as a sign of excessive permissions that could be exploited by an attacker if the user’s account were to be compromised.

\runinhead{Tombstone} When an object is deleted from Active Directory, it is moved to the Deleted Objects container. The object retains most of its attributes. Objects remain in this container for the tombstone period (by default 180 days), after which they are permanently deleted. If the tombstone period has not yet passed, the deleted objects can be reanimated. Restoring incorrectly tombstoned objects might lead to inconsistencies and possibly orphaned objects.

\runinhead{Tombstone Lifetime (TSL)} The number of days before a deleted object is removed from the directory services. The default value was used to be 60 days when no value is entered. In modern operating systems, the value is 180 days.

\runinhead{Transitive Trust} A two-way relationship that is automatically created between parent and child domains in a Microsoft Active Directory forest and that can allow users from one domain in the forest to log into resources on any other domain in that forest. This means that if Domain A trusts Domain B and Domain B trusts Domain C, then Domain A automatically trusts Domain C. Attackers can exploit this transitivity to gain unauthorized access to resources.

\runinhead{Transmission Control Protocol (TCP / Internet Protocol (IP) Stack} The TCP/IP Stack (or suite), and also known as the Internet protocol suite, is a set of communication protocols that governs how data is transmitted across networks, including the internet. It is a foundational model that enables devices to communicate withe each other by organizing protocols into distinct layers, each with specific functions. The Internet protocol suite represents network protocols in practical internet and network communications:
1. Link
2. Internet
3. Transport
4. Application
Used in all modern networking, including web traffic, email, and file transfers.

\runinhead{Transmission Control Protocol/Internet Protocol (TCP/IP)} TCP/IP is the suite of communication protocols used to interconnect network devices on the Internet or on a private network. In the context of Active Directory, it is critical, as it forms the backbone of network communications. Ensuring that proper security measures such as IPsec are implemented is important to protect network traffic against sniffing or spoofing attacks.

\runinhead{Transport Layer Security (TLS)} TLS is a protocol that ensures privacy between communicating applications and users on the Internet. When a server and client communicate, TLS ensures that no third party can eavesdrop or manipulate any message. If not implemented correctly, attackers may exploit vulnerabilities in the protocol or use downgraded attacks to weaken the security of the connection.

\runinhead{Tree} A tree is a collection of Active Directory domains in hierarchical order with a contiguous namespace.

\runinhead{Tree Root Trust} A tree root trust is an automatic transitive trust relationship established between the root domains of two trees in the same Active Directory forest. This trust allows all domains in one tree to trust all domains in the other tree; however, similar to transitive trusts, tree root trusts could be potentially exploited by attackers for lateral movement within the forest.

\runinhead{TrickBot} TrickBot is a Trojan type malware that was first identified in 2016. Its original purpose was to target banks and steal financial data, but TrickBot has evolved into a multi-stage modular malware. The most common initial infection vector is malspam that contains malicious, macro-laden office documents such as invoices, holiday greeting cards, traffic violations, and more.

\runinhead{Trust} A trust is a relationship between domains that allows objects in one domain to have access to resources in another. It is established between two domain trees or forests to enable users in one domain to access resources in the other. For example, a user from one domain can log in and access resources from another domain.

\runinhead{Trust Attributes} These define the type of access that is given to a trusted domain. Trust attributes include settings such as selective authentication, which restricts access to only certain resources in a domain. Misconfigurations in trust attributes can lead to unauthorized access to resources.

\runinhead{Trust Boundary} In Active Directory, a trust boundary is a logical boundary that separates different security domains or realms. It represents the extent to which trust relationships can be established between entities within and outside the boundary. Establishing a trust relationship across this boundary allows security principals (such as users or computers) from one domain to access resources in another domain.

\runinhead{Two-Factor Authentication (2FA)} 2FA adds an extra layer of security to the authentication process by requiring users to verify their identity using two different factors: something they know (like a password) and something they possess (like a token or mobile device). It makes it harder for attackers to gain access even if they manage to compromise one factor.


\runinhead{Trust Forgery} Trust forgery refers to a sophisticated hacking technique in which an attacker exploits weaknesses in trust relationships between domains or forests to gain unauthorized access and elevate privileges. This method leverages forged Kerberos tickets, particularly "Trust Tickets," to impersonate users and gain access to resources across those trusts.

\section*{U}
\runinhead{Universal Groups} Universal groups in Active Directory are groups that can include users, groups, and computers from any domain within its AD forest. This attribute makes them ideal for large-scale access management across multiple domains, but misconfigurations or excessive use can increase replication traffic within the forest, potentially affecting performance.

\runinhead{\texttt{Unlock-ADAccount}} This PowerShell command unlocks an AD account that has been locked out. Misuse could lead an attacker to unlock accounts that were locked due to suspicious activities.

\runinhead{Update Sequence Number (USN)} A USN is a 64-bit number in Active Directory that increases as changes occur to objects or attributes. Used to control the replication of these changes throughout the entire AD forest.

\runinhead{User Account Control (UAC)} While not strictly an AD term, UAC is a Windows security feature that can interact with AD. Controls the privileges of a user account and requests confirmation whenever a change requires administrative rights. If UAC settings are not appropriately configured, it may allow unauthorized changes or malware propagation.

\runinhead{User Identifier (UID)} In Active Directory Services Interface (ADSI), each object has a unique identifier known as a UID. It is often used when interacting with AD via scripts or programming languages, serving as a distinct pointer to an object in the directory.

\runinhead{User Object} In Active Directory, a user object is a distinct set of attributes that represent a network user. It includes information such as username, password, and various other details about the user. The security of user objects is paramount to ensure data privacy and prevent unauthorized access.

\runinhead{User Principal Name (UPN)} An Internet-style login name for a user object based on the Internet standard RFC 822. The UPN is shorter than the distinguished name and easier to remember, and simplifies the login process, especially in environments with trust relationships. By convention, this should map to the user email address, which makes it easier to remember.

\section*{V}
\runinhead{Virtual Directory} A virtual directory is a directory name, also known as a path, which is used to reference the physical directory (or directories) where files are actually stored. This concept is important in AD because it allows resources to be located and managed effectively without directly handling the underlying complexities of their physical locations.

\runinhead{Virtual Local Area Network (VLAN)} A VLAN is a logical division in a network, which group a set of devices that can communicate as if they were on the same physical network, even if they are not. From an AD perspective, VLANs can influence the design and performance of AD replication, as well as the application of policies.

\runinhead{Virtual Private Network (VPN)} A VPN is a secure network connection that uses encryption and other security mechanisms to ensure that only authorized users can access the network and that the data cannot be intercepted. This has implications for Active Directory because it allows remote and secure access to the organization's network where AD resources reside.

\runinhead{Virtualization} The process of creating a virtual version of something, including but not limited to a hardware platform, operating system, a storage device, or network resources. In AD contexts, domain controllers might be virtualized to save on hardware costs or for disaster recovery purposes; however, virtualization of AD components needs to be managed carefully, as incorrect configurations (such as having a virtual domain controller keep its own time) can lead to significant issues.

\runinhead{Virus} A virus is a malicious program that replicates itself to spread to other computers. It can potentially impact Active Directory if it infects systems that interact with AD or if it specifically targets AD components. Virus protection and timely intervention are crucial to maintaining the integrity and availability of the AD environment.

\runinhead{Visual Basic Scripting Edition (VBScript)} VBScript is a lightweight scripting language, developed by Microsoft, that is often used for server-side scripting in Active Directory environments. Despite its age, many AD admins have legacy VBScripts in their environment or may use VBScript for quick, simple tasks.

\runinhead{Volume Shadow Copy Service (VSS)} This is a Windows service that enables manual or automatic backup of computer files and volumes. It is essentially Windows's native backup tool, capable of creating 'shadow' copies at specified intervals or when triggered by a system event. VSS can be used to backup an Active Directory database while it is still running and plays an essential role in system restore and data recovery operations.

\runinhead{Vulnerability Scanning} This security technique is used to identify security weaknesses in a computer system. In the context of Active Directory, vulnerability scanning can uncover issues such as unpatched software, security misconfigurations, or the use of weak passwords. Regular vulnerability scanning is a critical part of maintaining the security of an Active Directory environment.

\section*{W}
\runinhead{WannaCry} The WannaCry ransomware worm exploited the EternalBlue vulnerability and targeted Windows-based computers in 2017. Microsoft released a security patch for EternalBlue shortly before the attacks began, but many Windows users did not immediately update their systems or were using out-of-date versions of Windows. As a result, WannaCry infected more than 200,000 computers in 150 countries and caused \$8~billion in damages.

\runinhead{Wbadmin} A command prompt utility that allows administrators or backup operators to backup and restore an operating system (OS), volume, file, folder, or application. Wbadmin replaced NT backup, the tool used to create backups in systems before Windows Server 2008.

\runinhead{Wildcard Certificate} In the realm of SSL/TLS, a wildcard certificate is a certificate that can secure any subdomain of a domain. For example, a single wildcard certificate for \texttt{*.hackherway.com can secure www.hackherway.com, mail.hackherway.com.} In AD, a wildcard certificate can be used to secure multiple services without the need for multiple certificates.

\runinhead{Windows Azure Active Directory (WAAD)} Also known as Azure Active Directory (AAD). This is the Active Directory Domain Services in the Windows Azure cloud.

\runinhead{Windows Defender} Windows Defender is the built-in real-time security system on Windows that offers protection against a wide range of threats such as malware, spyware, and viruses. It plays a critical role in the security of devices that are part of the Active Directory.

\runinhead{Windows Event Log} A Windows application that allows administrators to view detailed records of the operating system, security, and application notifications. Used by administrators to diagnose system problems and predict future issues.

\runinhead{Windows Internet Naming Service (WINS)} A Windows-based service that resolves NetBIOS computer names into IP addresses. WINS was designed to solve the problems that arise from NetBIOS name resolution in routed environments.

\runinhead{Windows Management Instrumentation (WMI)} An infrastructure built into Microsoft Windows operating systems that allows management data and operational parameters to be retrieved and configured on any device that “connects” to the system. Provides a unified way for software to request system information and manage system components locally or remotely. WMI can be used for tasks such as querying system settings, setting system properties, or triggering specific actions on systems.

\runinhead{Windows PowerShell} Windows PowerShell is a command-line shell and scripting language based on tasks designed especially for system administration. Built on the.NET Framework, PowerShell helps IT professionals control and automate the administration of the Windows operating system and applications that run on Windows. For example, administrators can use PowerShell to automate the process of creating users in Active Directory.

\runinhead{Windows Script Host (WSH)} A language-independent scripting host for Windows Script compatible scripting engines. It provides a set of objects and services that enable system-level scripting, allowing scripts written in JScript or VBScript, for instance, to automate administrative tasks or interact with the Windows operating system directly. WSH scripts can be executed directly from the desktop or command prompt, or they can be embedded into a webpage, providing a versatile platform for automating routine tasks.

\runinhead{Windows Server} Windows Server is a group of operating systems designed to support enterprise-level management, data storage, applications, and communications. Active Directory is one of the critical services that run on Windows Server, providing a variety of directory services.

\runinhead{Windows Server Update Services (WSUS)} 
\textit{Pronounced:} \textbf{/ˈwʌs.əs/} or \textbf{“WUH-sus”} (alternatively: “double-u-SUS”)
Windows Server Update Services (WSUS) is a Microsoft tool used to centrally manage and deploy operating system updates and patches within a Windows domain. In Active Directory environments, WSUS works alongside Group Policy to control how updates are deployed and distributed by domain-joined machines (e.g., patch windows might be different for select groups of machines). The main benefit of WSUS is to streamline patch and update management, reduce external bandwidth usage, and maintain security by ensuring that systems are consistently patched against known vulnerabilities.

\runinhead{Windows Time Service \texttt{W32Time}} The Windows Time Service (\texttt{W32Time}) is a core Windows component that synchronizes the system clock across devices in a networked environment. In Active Directory, accurate time synchronization is critical, particularly for Kerberos authentication, which depends on time-stamped tickets to validate user and service logins. The Network Time Protocol (NTP) is used in conjunction to maintain secure and consistent timekeeping across all domain controllers and domain-joined devices. Time drift beyond an acceptable threshold can cause authentication failures, replication issues, and security vulnerabilities, making the W32Time configuration and monitoring essential for AD integrity and performance.

\runinhead{WinNT} The Windows NT namespace provider that supports the Windows NT SAM account database.

\runinhead{Wireshark} Widely used in network troubleshooting and protocol analysis, Wireshark is a legitimate network capture tool that attackers can use to capture and analyze network traffic, potentially uncovering sensitive data transmitted in plaintext.

\runinhead{WMI Query Language (WQL)} A SQL-like language used to filter and query information from the Windows Management Instrumentation (WMI) framework. It is utilized to write queries that return specific information from the vast amount of data that WMI can deliver, such as querying for events or data objects, calling methods, or accessing or modifying system properties. WQL offers a robust toolkit for administrators to automate tasks, troubleshoot, or gather system information in a Windows environment.

\runinhead{Workstation} In the context of Active Directory, a workstation typically refers to a computer that is connected to the network and under the control of AD. Policies can be applied to workstations, users can log into them using their AD credentials, and they can access resources based on their AD user rights and permissions.

\runinhead{Writeable Domain Controller (WDC)} A writeable domain controller is a server that hosts a writable copy of the AD database. This is in contrast to a Read-Only Domain Controller (RODC). WDCs allow changes to the database that are then replicated to other DCs. Any compromise of a WDC can have a significant impact due to its ability to alter directory data.

\section*{X}
\runinhead{X.25} X.25 is an open source packet-switched wide area network (WAN) protocol that was widely used in the 1970s-1990s to connect remote systems before the rise of modern TCP/IP networking. Although not widely deployed today, some legacy systems and critical infrastructure may still rely on X.25-based communications. The thought process behind including X.25 in a cybersecurity context is to highlight that outdated protocols can pose security risks if left unmonitored or unprotected. In relation to Active Directory, if AD-integrated services or remote authentication rely on legacy X.25-linked infrastructure, attackers could potentially exploit weak encryption and poor access controls, making modernization or network segmentation essential.

\runinhead{X.500} X.500 is a set of standards developed by the International Telecommunications Union (ITU) for directory services that define how directory information should be structured and accessed in a distributed environment. It introduced concepts such as the hierarchical directory structure and Distinguished Names (DNs) that later influenced the Lightweight Directory Access Protocol (LDAP). X.500 provides a universal framework for identifying, storing, and retrieving information about users and resources across networks. Active Directory is heavily based on the X.500 principles-its schema, object hierarchy, and use of Distinguished Names all stem from this standard-making X.500 a foundational concept for understanding how AD organizes and manages identity data.

\section*{Y}
\runinhead{YubiKey} A YubiKey is a physical security token that supports multifactor authentication (MFA). Connects via USB, NFC, or Bluetooth and can be used to generate One-Time Passwords (OTP), support public key cryptography, and enable FIDO2/WebAuthn authentication. The thought process behind using a YubiKey is to strengthen user authentication by requiring a physical device, making it significantly harder for attackers to compromise accounts even if they obtain the user's password.

\section*{Z}
\runinhead{Zero-Day (0day)} A zero-day refers to a previously unknown software vulnerability for which no patch or fix has been developed or released by the software vendor. This means that when a zero-day vulnerability is discovered, the software vendor has had zero days to address the issue, hence the name.

\runinhead{Zero Trust} Zero Trust is a cybersecurity model that operates on the principle that no user, device, or system should be inherently trusted, regardless of whether they are inside or outside the network perimeter. The thought process behind Zero Trust is to eliminate implicit trust by continuously verifying identity, enforcing least-privilege access, and applying strict access controls across all resources. All access requests must be authenticated, authorized, and encrypted. Identity security plays a central role in Zero Trust, as it ensures that only verified users and devices can access critical applications and data.

\runinhead{Zero Trust Network Architecture (ZTNA)} Zero Trust Network Architecture (ZTNA) is a cybersecurity framework based on the principle of "never trust, always verify." Unlike traditional perimeter-based security models, ZTNA assumes that threats can originate both outside and inside the network. The thought process behind ZTNA is to require strict identity verification and continuous access validation for every user, device, and application, regardless of their location on or off the network. Access is granted based on the least privileges and contextual factors, such as user role, device posture, and location. ZTNA is increasingly adopted to reduce attack surfaces and improve resilience in hybrid and cloud environments.

\runinhead{Zerologon (CVE-2020-1472)} Zerologon is a critical vulnerability in the Microsoft Netlogon Remote Protocol that affects Active Directory domain controllers. The flaw stems from a weakness in the cryptographic implementation that allows an attacker to authenticate as any computer, including a domain controller, without valid credentials. The thought process behind exploiting Zerologon is to manipulate the Netlogon process to reset a domain controller’s machine account password to a blank value. This can result in full domain compromise or denial of service (DoS). Due to its severity, Zerologon highlights the importance of patching critical systems and securing authentication protocols.

\runinhead{Zone Transfers (DNS)} A zone transfer is a process in the Domain Name System (DNS) where a DNS server shares a copy of its zone data with another DNS server to maintain consistency and redundancy. In the context of Active Directory, zone transfers are a key part of replicating AD-integrated DNS zones across domain controllers. The thought process behind enabling zone transfers is to support fault tolerance and efficient DNS resolution; however, if not properly secured, zone transfers can unintentionally expose detailed information about internal network infrastructure to unauthorized parties. To mitigate this risk, zone transfers should be restricted to trusted servers only.
