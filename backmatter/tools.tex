\renewcommand{\tipbox}[1]{\begin{quote}\textbf{#1}\end{quote}}
%%%%%%%%%%%%%%%%%%%%% chapter.tex %%%%%%%%%%%%%%%%%%%%%%%%%%%%%%%%%
%
% sample chapter
%
% Use this file as a template for your own input.
%
%%%%%%%%%%%%%%%%%%%%%%%% Springer-Verlag %%%%%%%%%%%%%%%%%%%%%%%%%%
%\motto{Use the template \emph{chapter.tex} to style the various elements of your chapter content.}
\chapter{Tools}
\label{intro} % Always give a unique label
% use \chaptermark{}
% to alter or adjust the chapter heading in the running head

\abstract*{Each chapter should be preceded by an abstract (no more than 200 words) that summarizes the content. The abstract will appear \textit{online} at \url{www.SpringerLink.com} and be available with unrestricted access. This allows unregistered users to read the abstract as a teaser for the complete chapter.
Please use the 'starred' version of the new \texttt{abstract} command for typesetting the text of the online abstracts (cf. source file of this chapter template \texttt{abstract}) and include them with the source files of your manuscript. Use the plain \texttt{abstract} command if the abstract is also to appear in the printed version of the book.}
%\begin{tips}{W}
If you want to emphasize complete paragraphs of texts in an, we recommend to
use  \begin{verbatim}\begin{tips}{Tips}
...
\end{tips}\end{verbatim}
%\end{tips}
%
%
\tipbox {\textbf{PowerView} and its .NET counterpart, \textbf{SharpView}, are reconnaissance tools designed to provide situational awareness in Active Directory environments. Functionally, they act as powerful replacements for many traditional Windows \texttt{*net} commands, but with far greater flexibility and depth. Both tools allow you to enumerate users, groups, computers, shares, and access rights throughout the domain, and in many ways, the data you collect with either tool will overlap with what BloodHound provides you; however, unlike BloodHound-which automatically builds visual relationship graphs-you must manually interpret and correlate the results to uncover meaningful attack paths. These tools are especially useful when testing new credentials, since they allow you to quickly determine what additional access is unlocked by a new account. They also enable targeted queries against specific users or systems, helping you to identify "quick wins" such as accounts vulnerable to Kerberoasting or AS-REPRoasting attacks.}

\tipbox{BloodHound is a tool used to visually map Active Directory relationships and uncover potential attack pathways that might otherwise go unnoticed. It relies on the SharpHound ingestor, available in both PowerShell and C\# to collect data from the environment. This information is then imported into the BloodHound application-built with JavaScript (Electron) and backed by a Neo4j database-where it can be fully analyzed through a user-friendly and intuitive graphical interface, providing clear insights into privilege escalation opportunities and attack vectors within the AD environment.}

\tipbox{SharpHound is a C\# data collector is designed to gather comprehensive information from Active Directory, including details about users, groups, computers, Access Control Lists (ACLs), Group Policy Objects (GPOs), user and computer attributes, active sessions, and more. Once collected, the data is output in JSON format, which can then be ingested into the BloodHound graphical interface for analysis. This process allows security professionals and defenders to visualize complex AD relationships, identify privilege escalation opportunities, and map potential attack pathways across the domain.}

\tipbox{BloodHound.py is a Python-based BloodHound ingestor, built on the Impacket toolkit and provides a flexible alternative for collecting Active Directory data. It supports most of the standard BloodHound collection methods and has the added advantage of being able to run from a non-domain joined attack host. The data it gathers is exported in a format compatible with the BloodHound GUI, where it can be ingested and analyzed to reveal relationships, privilege escalation opportunities, and potential attack pathways within the targeted environment.}

\tipbox{Kerbrute is a Go-based tool that leverages Kerberos pre-authentication to interact with Active Directory. It can be used to enumerate valid domain accounts, conduct password spraying attacks, and perform brute-force attempts against user credentials. By exploiting the way Kerberos responds during authentication, Kerbrute allows attackers to efficiently identify valid usernames and weak passwords without needing prior domain access.}

\tipbox{The Impacket toolkit is a collection of Python-based tools designed for interacting with network protocols, offering a wide range of capabilities for penetration testers and red team operators. It includes numerous scripts that can be used to enumerate Active Directory environments, extract valuable information, and perform targeted attacks. With modules that support SMB, Kerberos, LDAP, and other key protocols, Impacket provides the building blocks for both reconnaissance and exploitation across Windows networks.}

\tipbox{Responder is a specialized tool designed to poison name resolution protocols such as LLMNR, NBT-NS, and mDNS, making it highly effective for capturing authentication requests on local networks. By responding to broadcast name resolution queries, it can trick systems into sending credentials or session information to the attacker's machine. Beyond poisoning, Responder includes several additional functions, such as setting up rogue SMB, HTTP, or LDAP servers to relay or capture authentication data across the wire, making it a versatile utility for both credential harvesting and network pivoting in Active Directory environments.}

\tipbox{Inveigh.ps1 is a PowerShell-based tool that functions in a manner similar to Responder, providing capabilities for conducting a variety of network spoofing and poisoning attacks within Windows environments. It can intercept and manipulate protocols such as LLMNR, NBT-NS, and mDNS to capture authentication hashes, relay credentials, and assist in credential harvesting operations. Since it runs natively in PowerShell, Inveigh.ps1 is particularly useful for red teamers or attackers operating without the ability to introduce external binaries, making it an effective alternative for network-based attacks in Active Directory environments.}

\tipbox{InveighZero is the C\# implementation of the Inveigh tool, designed to provide the same spoofing and credential capture capabilities as its PowerShell predecessor but with additional flexibility. It features a semi-interactive console that allows you to view and interact with captured data in near-real-time, such as usernames, password hashes, and challenge-response exchanges. By running as a standalone executable, InveighZero avoids some of the restrictions associated with PowerShell execution policies, making it a more versatile option for performing network spoofing and poisoning attacks within Active Directory environments.}

\tipbox{The \texttt{rpcinfo} utility is used to query the status of Remote Procedure Call (RPC) programs and enumerate the RPC services available on a remote host. By using the \texttt{-p} option, you can specify a target system and retrieve a list of all active RPC services, including their program numbers, version numbers, and supported protocols. For example, running \texttt{rpcinfo -p 10.0.0.1} will display detailed information about the RPC services running on the host at that IP address. Because this utility interacts directly with system-level services, it must be executed with elevated administrative privileges to return meaningful results.}

\tipbox{rpcclient is a command-line tool included with the Samba suite on Linux distributions that enables interaction with the Remote Procedure Call (RPC) service on Windows systems. It is frequently used for Active Directory enumeration, allowing you to query information such as domain users, groups, shares, policies, and other directory objects without needing a full Windows environment. By establishing a connection to a domain controller or other Windows host, \texttt{rpcclient} provides a flexible interface for performing reconnaissance and gathering intelligence in preparation for more targeted attacks.}

\tipbox{CrackMapExec (CME) is a versatile enumeration, attack, and post-exploitation framework designed to streamline operations in Active Directory environments. It is particularly effective for leveraging the data gathered during reconnaissance, allowing attackers to quickly validate credentials, enumerate users and shares, and execute commands across multiple systems. CME follows a “living off the land” approach by abusing built-in AD protocols and services such as SMB, WMI, WinRM, and MSSQL, making it both powerful and stealthy. Its ability to automate common tasks and chain attacks makes it a go-to tool for red teamers and penetration testers working against Windows domains.}

\tipbox{Rubeus is a powerful C\# tool specifically designed for abusing and manipulating Kerberos authentication within Active Directory environments. It supports a wide range of Kerberos-related attacks, including ticket extraction, pass-the-ticket, overpass-the-hash, Kerberoasting, AS-REP Roasting, and ticket renewal or manipulation. Because Kerberos is the backbone of authentication in AD, Rubeus provides red teamers and penetration testers with the ability to explore, exploit, and persist within domains by directly interacting with the ticketing system. Its versatility makes it one of the most widely used tools for both offensive operations and for defenders seeking to understand and detect Kerberos-based attack techniques.}

\tipbox{\texttt{getuserspn.py} is an Impacket module designed to identify Service Principal Names (SPNs) that are associated with regular user accounts in an Active Directory environment. Since SPNs linked to user accounts often indicate service accounts with elevated privileges, these findings are valuable for Kerberoasting attacks, where attackers request and extract Kerberos service tickets to attempt offline password cracking. By automating the discovery of these user-tied SPNs, \texttt{getuserspn.py} helps streamline reconnaissance and highlights potential targets that could lead to privilege escalation within the domain.}

\tipbox{Hashcat is a highly versatile and efficient tool for hash cracking and password recovery, widely used by penetration testers, red teamers, and security researchers. Supporting a vast range of hashing algorithms and password formats, it leverages CPU and GPU acceleration to perform fast and large-scale brute-force, dictionary, rule-based, and hybrid attacks. In Active Directory contexts, Hashcat is frequently used to crack password hashes obtained through attacks such as Kerberoasting, AS-REP Roasting, or NTLM hash extraction, helping to reveal weak or reused credentials. Its performance, flexibility, and extensive community support make it one of the most popular and effective tools for password auditing and recovery.}

\tipbox{Enum4linux is a Linux-based tool used to extract valuable information from Windows systems and Samba shares through SMB protocol interactions. It automates many common enumeration techniques, allowing you to gather details such as domain users, groups, shares, password policies, and operating system information. Because it does not require authentication for many queries, Enum4linux is particularly useful in the early stages of reconnaissance against Active Directory environments, helping attackers quickly build a profile of the domain and identify potential entry points. Its simplicity and effectiveness make it a staple utility for penetration testers working in mixed Windows and Linux environments.}

\tipbox{Enum4linux-ng is a modern rework of the original Enum4linux tool, designed to provide more reliable and flexible enumeration of Windows and Samba systems. While it serves the same core purpose—gathering information such as users, groups, password policies, and network shares—it introduces updated code, improved performance, and expanded functionality compared to its predecessor. Enum4linux-ng also handles SMB interactions differently, making it more compatible with current Windows environments and better suited for integration into automated workflows. This makes it a valuable utility for penetration testers conducting reconnaissance and information gathering in Active Directory networks.}

ldapsearch is a command-line utility commonly used to query and enumerate information from LDAP directories, including Microsoft Active Directory. It allows you to perform flexible searches against directory services, retrieving details such as users, groups, organizational units, policies, and other directory objects. By crafting specific queries, penetration testers can gather a wealth of information about the domain structure, trust relationships, and security configurations without requiring privileged access. Because it is lightweight, scriptable, and widely available on Linux systems, \verb|ldapsearch| is a go-to tool for both reconnaissance and targeted enumeration in AD environments.

\tipbox{Windapsearch is a Python-based script designed to enumerate Active Directory objects such as users, groups, and computers by leveraging LDAP queries. It simplifies the process of interacting with the directory service and is particularly useful for automating custom LDAP queries that might otherwise be repetitive or time-consuming to perform manually. By running Windapsearch, penetration testers can quickly extract key information about the domain structure, identify high-value accounts or groups, and map relationships that may be useful for privilege escalation or lateral movement within an AD environment.}

\tipbox{DomainPasswordSpray.ps1 is a PowerShell-based tool designed to execute password spraying attacks against Active Directory domain accounts. Instead of brute-forcing a single user with many passwords—which risks triggering account lockouts—this tool attempts one or a few commonly used passwords across a large set of domain users. This technique helps attackers identify weak or reused credentials while reducing the likelihood of detection through lockout policies. DomainPasswordSpray.ps1 streamlines the process by automating user enumeration, authentication attempts, and result tracking, making it an effective utility for both penetration testers and red teamers assessing password hygiene within enterprise environments.}

\tipbox{The LAPS Toolkit is a collection of PowerShell functions built on top of PowerView, designed to audit and exploit Active Directory environments that have implemented Microsoft’s Local Administrator Password Solution (LAPS). While LAPS is intended to improve security by randomizing and centrally managing local administrator passwords, the toolkit can be used to assess whether the deployment has been configured securely or if it introduces new attack opportunities. With its functions, penetration testers and red teamers can query LAPS-related attributes, identify accounts with access to stored passwords, and in some cases extract or misuse those credentials to escalate privileges within the domain.}

\tipbox{SMBMap is a penetration testing tool used to enumerate SMB shares across an Active Directory domain. It allows you to quickly identify accessible shares on remote systems, determine user permissions, and search for sensitive files such as configuration data, credentials, or scripts that may aid further compromise. By automating share enumeration, SMBMap helps attackers and red teamers uncover misconfigurations like overly permissive access rights or world-readable directories. Its ability to test authentication with provided credentials also makes it useful for validating account access and assessing the overall security posture of SMB resources within the domain.}

\tipbox{\texttt{psexec.py} is a module within the Impacket toolkit that replicates the functionality of Microsoft’s PsExec utility, providing attackers with the ability to execute commands on remote Windows systems. By leveraging SMB and Windows services, it creates a semi-interactive shell on the target host, allowing for remote code execution once valid credentials are supplied. This makes it a powerful tool for lateral movement, privilege escalation, and post-exploitation within Active Directory environments. Because it relies on standard Windows protocols and services, \texttt{psexec.py} is also an effective “living off the land” technique that can blend into normal administrative activity if not closely monitored.}

\tipbox{\texttt{wmiexec.py} is a tool that allows command execution over Windows Management Instrumentation (WMI). This provides a stealthy alternative to PsExec, often blending in with normal administrative activity and making it useful for remote code execution and post-exploitation.}

\tipbox{Snaffler is a reconnaissance utility designed to automatically search network file shares for sensitive information, such as credentials, configuration files, or scripts. It helps attackers and red teamers quickly identify files of interest across large Active Directory environments.}

\tipbox{smbscript.py is an Impacket script that sets up a simple SMB server on Linux to interact with Windows hosts. It is frequently used for transferring files within a network, staging payloads, or exfiltrating data using standard SMB communications.}

\tipbox{setspn.exe is a native Windows command-line tool that allows administrators (and attackers) to add, read, modify, or delete Service Principal Names (SPNs) associated with service accounts. It is commonly used during Kerberos-related attacks, such as Kerberoasting. }

\tipbox{Mimikatz— A well-known post-exploitation tool capable of extracting plaintext passwords, NTLM hashes, and Kerberos tickets directly from memory. It supports techniques such as Pass-the-Hash (PtH), Pass-the-Ticket (PtT), and Golden Ticket attacks, making it one of the most widely used offensive tools in Active Directory exploitation.}

\tipbox{secretsdump.py is part of the  Impacket toolkit, and this script remotely extracts sensitive information such as SAM and LSA secrets, cached credentials, and Kerberos keys from a Windows host. It is highly effective for credential dumping without requiring local code execution.}

\tipbox{evil-winrm — A tool that provides an interactive PowerShell-based shell on a remote host via the Windows Remote Management (WinRM) protocol. It is commonly used by red teamers for remote administration, file uploads, and command execution once valid credentials are available.}

\tipbox{mssqlclient.py is an Impacket script that enables interaction with Microsoft SQL Server (MSSQL) databases. It allows for authentication, query execution, and in some cases, abuse of SQL-linked privileges for lateral movement or privilege escalation.}

\tipbox{noPac.py is an exploit that chains vulnerabilities CVE-2021-42278 and CVE-2021-42287, enabling a standard domain user to impersonate a Domain Administrator. This proof-of-concept demonstrated one of the most critical AD privilege escalation flaws discovered in recent years.}

\tipbox{rpcdump.py is another Impacket tool that queries and enumerates RPC endpoints on a remote system. It is useful for identifying available RPC services, which can guide further enumeration and potential exploitation.}

\tipbox{CVE-2021-1675.py (PrintNightmare) is a Python proof-of-concept exploit for the PrintNightmare vulnerability, which abuses the Windows Print Spooler service to escalate privileges and execute code remotely.}

\tipbox{ntlmrelayx.py is part of the  Impacket toolkit, and this tool performs NTLM relay attacks by capturing NTLM authentication attempts and relaying them to other services to gain unauthorized access. It supports SMB, HTTP, LDAP, and other protocols, making it highly versatile.}

\tipbox{PetitPotam.py is also a proof-of-concept exploit for CVE-2021-36942 that coerces Windows hosts into authenticating to other machines via the MS-EFSRPC protocol. It is often used in relay attack chains against Active Directory Certificate Services.}