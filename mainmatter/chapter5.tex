%%%%%%%%%%%%%%%%%%%%% chapter.tex %%%%%%%%%%%%%%%%%%%%%%%%%%%%%%%%%
%
% sample chapter
%
% Use this file as a template for your own input.
%
%%%%%%%%%%%%%%%%%%%%%%%% Springer-Verlag %%%%%%%%%%%%%%%%%%%%%%%%%%
%\motto{Use the template \emph{chapter.tex} to style the various elements of your chapter content.}
\chapter{Information Security: Cybersecurity Frameworks, Laws, Rules, and Regulations}
\label{intro} % Always give a unique label
% use \chaptermark{}
% to alter or adjust the chapter heading in the running head

\abstract*{Each chapter should be preceded by an abstract (no more than 200 words) that summarizes the content. The abstract will appear \textit{online} at \url{www.SpringerLink.com} and be available with unrestricted access. This allows unregistered users to read the abstract as a teaser for the complete chapter.
Please use the 'starred' version of the new \texttt{abstract} command for typesetting the text of the online abstracts (cf. source file of this chapter template \texttt{abstract}) and include them with the source files of your manuscript. Use the plain \texttt{abstract} command if the abstract is also to appear in the printed version of the book.}



\abstract{It goes without saying that the world of cybersecurity is indeed a complex one. To add to the burden, it is no wonder why today's organizations continue to struggle to keep up strong security postures in conjunction with already established complex compliance needs, regulations, and mandates. Within this context, this chapter conducts an in-depth analysis of the role of the NIST SP 800 series in developing standard methodologies of employing ethical hacking and penetration testing, focusing in particular on SP 800-115s formal framework of technical security testing and assessment. Based on an assessment of methodologies of implementation and compliance needs, readers will have a thorough knowledge of how guidelines from NIST convert security testing from discrete exercises into blended components of an organizational security strategy.}
  

Areas of focus include:
\textbf{1. Framework Implementation and Technical Methodology}
\begin{itemize}
    \item In-depth analysis of SP 800-115s five-phase testing methodology
    \item Strategies to integrate automated and manual testing methods
    \item Implementation considerations for various organizational contexts
    \item Procedures and documentation requirements for technical verification
\end{itemize}
\textbf{2. Alignment with Regulations and Compliance}
\begin{itemize}
    \item Mapping across security frameworks to major compliance standards
    \item Audit preparation documentation requirements
    \item Integration of risk assessment into existing security control posture
    \item Continuous monitoring and maintenance procedures
\end{itemize}
\textbf{3. Technical Testing Procedures}
\begin{itemize}
    \item Methods and tools of vulnerability scanning
    \item Penetration testing techniques and limitations
    \item Protocol assessment procedures
    \item Remediation techniques and post-testing activities
\end{itemize}
\textbf{4. Organizational Integration}
\begin{itemize}
    \item Software Development Life Cycle (SDLC) incorporation strategies
    \item Cross-functional and cross-domain team collaboration models
    \item Risk visibility frameworks
    \item Stakeholder communication procedures
\end{itemize}
In close analysis of actual real-world implementation challenges and successful aspects, this chapter provides security professionals practical guidelines on implementing NIST-compliant ethical hacking methods that defenders can utilize within their organization's security position. Special emphasis is placed on how compliance to the SP 800 series documents enhances security effectiveness and regulatory compliance in tandem, maintaining operational efficiency and cost-effectiveness in place. This comprehensive examination serves as a definitive resource for a defender who seeks to establish compliant security testing programs that align with industry best practices while addressing contemporary cybersecurity challenges.

\section{NIST-Compliant Ethical Hacking}

\subsection{The Role of Ethical Hacking and Penetration Testing in NIST SP 800}
Ethical hacking and penetration testing are fundamental security practices that simulate cyberattacks to aid a defender in identifying weaknesses in an organization's security defenses. The National Institute of Standards and Technology (NIST) is a non-regulatory agency within the United States Department of Commerce, and its wide focus is on promoting innovation and industrial competitiveness by advancing measurement science, standards, and technology. The NIST plays a crucial role in cybersecurity by providing security frameworks, guidelines, and standards to help defenders and organizations manage and reduce cybersecurity risks. Many organizations voluntarily follow NIST guidelines, while others are required to do so based on their specific security and compliance needs. The NIST provides structured guidance to ensure that ethical hacking engagements, penetration tests, and red team security assessments are thorough, effective, meaningful, actionable, and aligned with an organization's security compliance requirements.
Two significant documents within the NIST Special Publication (SP) 800 series detail methodologies for ethical hacking and penetration testing:
\begin{itemize}
    \item \textbf{NIST SP 800-53 Revision 5}, \textit{"Security and Privacy Controls for Information Systems and Organizations,"} provides a catalog of security and privacy controls for information systems and organizations to protect organizational operations and assets, individuals, other organizations, and the nation from a diverse set of threats and risks, including hostile attacks, human errors, natural disasters, structural failures, foreign intelligence entities, and privacy risks. The controls are flexible and customizable and implemented as part of an organization-wide process to manage risk. The controls address diverse requirements derived from mission and business needs, laws, executive orders, directives, regulations, policies, standards, and guidelines.
    \item \textbf{NIST SP 800-115}, \textit{"Technical Guide to Information Security Testing and Assessment,"} assists organizations in planning and conducting technical information security tests and examinations, analyzing findings and developing mitigation strategies. The guide provides practical recommendations for designing, implementing, and maintaining technical information security test and examination processes and procedures. These can be used for several purposes, such as finding vulnerabilities in an information system or network and verifying and validating compliance with a policy or other requirement. However, the guide is not intended to present a comprehensive information security testing and examination program, but rather an overview of key elements of technical security testing and examination with an emphasis on specific technical techniques, the benefits and limitations of each, and recommendations for their use.
\end{itemize}
\section{Attacker Privilege Escalation to Domain Administrator in Active Directory}
