Book aims to teach:
\begin{itemize}
    \item Incident detection, prevention, and response through continuous security monitoring for success in a high-stakes analysis.
    \item Processes in security operations, vulnerability management, incident response and management, and how to apply best practices for reporting and communication.
    \item Processes in security operations to differentiate between cyber threat intelligence (CTI) and threat hunting concepts; identify and analyze malicious activities using appropriate tools and techniques.
    \item Implement and analyze vulnerability assessments, prioritize vulnerabilities, and make recommendations on mitigating and minimizing attack surfaces and vulnerability response.
    \item How to apply updated concepts of attack methodology frameworks, perform blue team incident response activities and understand the incident management lifecycle.
    \item Apply communication best practices in vulnerability management and incident response as it relates to stakeholders, action plans, escalation, and metrics.
\end{itemize}

\begin{abstract}
Identify network hardware components in LANs, WANs, and PANs
Understand the OSI Reference model and the TCP/IP security stack.
Perform domain management and administrative tasks using Windows Server Active Directory and Group Policy tools.
Explain basic theory, methodologies, technologies, and components that facilitate network data transmissions.



\begin{itemize}
\item Explain the OSI and TCP/IP models
\item Explain properties of network traffic
\item Understand difference between switched and routed networks
\item Understand IP networks, port monitoring, and protocols
\item Monitor and troubleshoot network issues
\item Explain network attacks and mitigation
\item Install and configure security devices
\item Explain network authentication, authorization, accounting, and network access control (NAC)
\item Deploy and troubleshoot cabling solutions
\item Troubleshoot wireless technologies
\item Compare and contrast WAN technologies
\item Use remote access methods
\item Identify site policies and best practices
\item Classify network intrusion elements
\item Understand artifacts related to network intrusions
\item Explain the basics of networking and network architecture
\item Install Nmap in a Windows and Linux environment
\item Determine what hosts, ports, and services are available on a network through host discovery scanning
\end{itemize}
\end{abstract}


\begin{itemize}

        \item Network Traffic Collection
            \item Explain basic theory, technologies, and components that facilitate network data transmissions
            \item Examine network traffic and previously sniffed data
            \item Perform a logical and physical assessment of a network to identify potential witness devices and the data they contain
            \item Assess a network and configure  and place a network tap
            \item Configure network data acquisition tools
            \item Use common internet and OSINT-related research utilities
            \item Analyze ingress / egress network traffic and system artifacts to identify probing and intrusion techniques and Indicators of Compromise (IOC)
    \end{itemize}

\begin{abstract}
        \item Network Scanning
            \item OpenVAS
                \item Configure the target, parameters 
\end{abstract}
  









