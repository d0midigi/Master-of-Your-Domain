

\chapter{Attacking and Defending Active Directory (AD) for Ethical Hackers Handbook}

1. Chapter 1: The Blueprint of Enterprise Control: Unmasking Active Directory's Core Security Architecture Objective: Deconstruct the fundamental components of Active Directory and their inherent security implications, establishing a proactive framework for identifying critical attack surfaces and preparing for sophisticated engagements.


Understanding Active Directory's Architecture: The Attacker's View
Domain Controllers and Global Catalogs: Core Trust Boundaries and Replication Targets
NTDS.DIT Database: The Crown Jewels of Many Organizations AD Credentials and Objects
SYSVOL and Group Policy objects (GPOs): Policy Enforcement and Configuration Weaknesses
DNS Integration: Critical for Resolution, Critical for Reconnaissance
Authentication Mechanisms: Cracking the Enterprise Gateways
Kerberos Protocol: Golden Tickets, Silver Tickets, and Beyond
NTLM: Hash Harvesting and Pass-the-Hash (PtH) Vulnerabilities
Multi-Factor Authentication (MFA) Bypass Techniques: Beyond Simple Credential Theft
Active Directory Objects: Enumerating and Exploiting Permissions
Users and Groups: Identifying privileged Accounts and Their Relationships
Service Accounts: Weak Passwords and Kerberoasting Targets
Organizational Units (OUs): Delegation Misconfigurations

- Offensive Reconnaissance: Unveiling Hidden Pathways to Complete Domain Control
-- Passive Information Gathering: Low-Risk Footprinting for High-Value Targets
--- Open-Source Intelligence (OSINT) for AD: Public Records and Exposed Data
--- DNS Enumeration: Mapping the Domain Landscape
--- Email and Employee Information Harvesting: Social Engineering Pre-cursors
-- Active Enumeration: Interrogating Active Directory Directly
--- LDAP Querying: Discovering Objects, Attributes, and Permissions
--- SMB Enumeration: Identifying Shares, Sessions, and Potentially Sensitive Data
--- Service Principal Name (SPN) Scanning for Kerberoasting: Identifying Service Accounts for Credential Theft
--- Vulnerability Scanning for Known Flaws: Automated Discovery of Weak Links

- Defensive Measures: Establishing a Resilient Active Directory Perimeter
-- Principle of Least Privilege Enforcement: Minimizing the Attack Surface
--- Tiered Administration Models: Isolating Critical Assets
Just-in-Time (JIT) Access: Ephemeral Privileges for High-Risk Tasks
Role-Based Access Control (RBAC) in AD: Granular Permission Management