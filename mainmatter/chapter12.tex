%%%%%%%%%%%%%%%%%%%%% chapter.tex %%%%%%%%%%%%%%%%%%%%%%%%%%%%%%%%%
%
% sample chapter
%
% Use this file as a template for your own input.
%
%%%%%%%%%%%%%%%%%%%%%%%% Springer-Verlag %%%%%%%%%%%%%%%%%%%%%%%%%%
%\motto{Use the template \emph{chapter.tex} to style the various elements of your chapter content.}
\chapter{Guide to AD GPOs and OUs}
\label{intro} % Always give a unique label
% use \chaptermark{}
% to alter or adjust the chapter heading in the running head

\begin{abstract}
Active Directory (AD) serves as the lifeblood of enterprise network identity and access management, with \textit{Group Policy Objects (GPOs)} and Organizational Units (OUs) forming critical components of centralized, hierarchical administration. Organizational Units function as logical containers within the AD hierarchy, enabling administrators to organize users, computers, and other objects based on administrative needs, geographical locations, or functional requirements. This hierarchical directory structure facilitates scalable management and delegation of administrative responsibilities.

Group Policy Objects represent collections of configuration settings that define user and computer environments across a network. GPOs encompass security policies, software deployment parameters, registry modifications, and administrative templates that standardize system behaviors. The strategic linking of GPOs to OUs enables targeted policy deployments, ensuring appropriate configurations reach specific user groups or computer collections without affecting the broader network.

The synergistic relationship between OUs and GPOs provides administrators with granular control over network resources while maintaining centralized governance. Proper OU design coupled with well-structured GPO implementation reduces administrative overhead, enhances security posture, and ensures consistent user experiences. Effective utilization of these technologies is essential for maintaining secure, compliant, and efficiently
\end{abstract}

\section{The Importance of Cyber Threat Intelligence and the MITRE ATT\&CK Framework}

\begin{quote}
    "I must not fear. Fear is the mind-killer. Fear is the little-death that brings total obliteration. I will face my fear. I will permit it to pass over me and through me."
    -Frank Herbert, Dune
\end{quote}
Before an organization can effectively defend its environment, it must first