\documentclass{article}
\usepackage{blindtext}
\usepackage{titlesec}
\title{Social Engineering Chapter}
\author{Yvonne Silva-Ewry}
\date{ }
\begin{document}

\maketitle

\tableofcontents

\section{Introduction}
\subsection{Overview}
\subsection{Recognizing the Threat}
\subsection{Psychology and Social Engineering}
\subsection{Resistance to Persuasion and Social Engineering}
\subsection{Academic Challenges of Studying Social Engineering}
\section{Literature Review}
\subsection{Defining Social Engineering}
\subsection{Trade Literature and Social Engineering}
\subsection{Current Preventive Techniques Deemed Inadequate}
\subsection{Introduction to Psychological Triggers}
\subsection{Cialdini and the Six Principles of Persuasion}
\subsection{Resistance to Persuasion}
\subsection{Sagarin's Experiments and Their Application to Social Engineering}
\section{Methodology}
\subsection{Problem Definition}
\subsection {Goals}
\subsection{Approach}
\section{Results and Analysis}
\subsection{Comparing Psychological Triggers to Principles of Persuasion}
\subsection{Comparing Definitions of Illegitimate Persuasion and Social Engineering}
\subsection{Comparing Objectives and Targets}
\subsection{Comparing Methods and Techniques}
\subsection{The Conclusion: Illegitimate Persuasion = Social Engineering}
\section{Chapter Overview}
\subsection{Constructivist Theory and Its Applicability in a Social Engineering Context}
\subsection{Other Training Techniques}
\subsection{Bloom's Taxonomy, Its Revision, and Its Applicability in a Social Engineering Context}
\subsection{Essential Components Required for a Successful Training Program}
\subsection{Limitations of Applicability and Potential for Future Research}

\

This is the first section.

\blindtext

\addcontentsline{toc}{section}{Unnumbered Section}
\section*{Unnumbered Section}

\blindtext

\section{Second Section}

\blindtext
\end{document}