\documentclass{article}
\usepackage{blindtext}
\usepackage{titlesec}
\title{Sections and Chapters}
\author{Gubert Farnsworth}
\date{ }
\begin{document}

\maketitle

\tableofcontents

\section{Part I-Foundations of Active Directory (AD) Security}

What This Book Will Teach You (and What It Will Not)
Offensive vs. Defensive Lenses in AD Security
Why AD Continues to Remain Vulnerable
AD Crown Jewels-An Attractive Target for Attackers
Ethical and Legal Considerations for Hacking Active Directory Environments

\subsection{Core Networking and Security Concepts for AD Environments}
Networking 101: How Data Moves
TCP/IP: IP Addressing, Subnets, and Routing Basics
Common Network and AD-Related Protocols: SMB, LDAP, Kerberos, DNS, RPC, RDP
Packet Flow in AD Authentication

\subsection{Active Directory Architecture}
Domains, Trees, and Forests
Organizational Units (OUs) and Group Policy Objects (GPOs)
Security Principles: Users, Groups, Computers, Service Accounts
AD Relationships and Trusts: Internal and External Attack Surfaces

\subsection{Identity, Authentication, and Authorization in Active Directory}
Kerberos Authentication Deep Dive
NTLM and Legacy Protocols
Access Control Lists (ACLs) and Security Descriptors
Privilege Models and AD Tiers (Tier 0, 1, 2)

\subsection{Regulatory and Compliance Drivers in AD Security}
NIST, CIS, ISO, and Microsoft Baselines
Common Enterprise AD Compliance Requirements
How Compliance Affects Offensive and Defensive Operations

\section{Part II-Offensive Security: Attacking Active Directory}

\subsection{The AD Attack Surface}
Mapping Attack Entry Points
Common Misconfigurations
The Role of Shadow IT in AD Risk

\subsection{Reconnaissance and Enumeration}
Passive Recon with OSINT
Internal AD Enumeration:
LDAP, PowerView, BloodHound
Mapping Trusts and Privilege Escalation Pathways

\subsection{Credential Access and Abuse}
Credential Dumping
LSASS
SAM
\subsubsection{\texttt{NTDS.dit}}
Pass-the-Hash (PtH)
Pass-the-Ticket (PtT)
Overpass-the-Hash
Harvesting and Reusing Kerberos Tickets

\subsection{Privilege Escalation in AD}
Exploiting Delegation
Unconstrained Delegation
Constrained Delegation
Resource-based Constrained Delegation (RBCD)
Abusing Service Principal Names (SPNs)
Kerberoasting
Exploiting Weak ACLs and GPO Permissions

\subsection{Persistence Techniques}
Golden and Silver Tickets
Skeleton Keys and Security Support Provider (SSP) Injection
DCShadow Attack
Abusing \texttt{AdminSDHolder}
SIDHistory Abuse

\subsection{Lateral Movement}
Pass-the-Hash / Ticket Techniques
Remote Service Exploitation
\subsubsection{WMI: (Windows Management Instrumentation}
\subsubsection{Sysinternals \texttt{PsExec}}
WRM: (Windows Remote Management)
Credential Relay Attacks
NTLM Relay Attack

\subsection{AD Attacks in Hybrid Environments}
Azure AD and On-Prem Sync Abuse
OAuth Token Theft and Cloud-Based Persistence
Cross-Tenant and Cross-Forest Exploitation

\section{Part III-Defensive Security: Protecting Active Directory}
\subsection{Building an AD Defense Strategy}

Adopting and Establishing Zero Trust in Active Directory
The Tiered Administration Model
The Principle of Least Privilege (PoLP) and Just-in-Time (JIT) Access

\subsection{Hardening Authentication}
Enforcing Kerberos Pre-Auth and Disabling NTLM
MFA for AD Accounts
Securing Service Accounts and SPNs

\subsection{Group Policy and OU Security}
Locking Down GPO Permissions
GPO Attack Detection and Prevention
Securing OU Structure Against Misuse

\subsection{Monitoring and Detection}
Event Logging for Defenders
SIEM Integration and Use Cases

\subsection{Detecting Common AD Attacks}
DCSync
DCShadow
AS-REP Roasting
LLMNR and NBT-NS

\subsection{Incident Response (IR) in AD Breaches}
Containment and Eradication Procedures
Putting Out Forest Fires: DC Reinstallation and Forest Recovery
Post-Incident Hardening

\subsection{Threat Hunting in AD}
Threat Hunting Methodologies
Using BloodHound for Defensive Strategies
Canary Objects and Decoy Accounts

\subsection{Defending Against Hybrid AD Attacks}
Azure AD Conditional Access Policies
Securing Azure AD Connect
Cloud and On-Prem Identity Threat Detection

\section{Part IV-Advanced Scenarios and Case Studies}
\subsection{Full Kill Chain Walkthrough: From Initial Toehold to Domain Admin (DA)}

Step-by-Step Red Team Attack Simulation
Blue Team Detection and Mitigation at Each Stage

\subsection{Case Study: Real-World AD Breach Analysis}
How the Attack Unfolded
Missed Defensive Opportunities
Lessons Learned and Preventative Measures

\subsection{Designing and Secure AD From the Ground Up}
Defender Security Best Practices
Secure Build Standards and Templates
Long-Term Maintenance Strategy

\subsection{Initial Access and Active Directory Exploitation}
\begin{itemize}
    \item Exploit unpatched, uncovered vulnerabilities to gain initial access
    \item Perform application allowlisting attack for executing malicious scripts inside the domain structure
    \item Set up an in-house Active Directory simulated environment and gain offensive and defensive learning by simulating APT TTPs and mapping attacker TTPs to their respective security framework
    \item Other Topics to be Covered:
        \begin{itemize}
            \item Lab Setup
        \end{itemize}
        \begin{itemize}
            \item     Abusing Server Message Block (SMB)
        \end{itemize}
        \begin{itemize}
            \item SMB DLL Delivery
        \end{itemize}
        \begin{itemize}
            \item LLMNR Poisoning Attack
        \end{itemize}
        \begin{itemize}
            \item Capturing NTLMv2 Hashes
        \end{itemize}
\end{itemize}

\subsection{Active Directory Enumeration}
Perform domain reconnaissance on domain entities such as domains, users, groups, ACLs, OYs. Forests, Trust Relationships, GPOs and GPPs using Microsoft built-in utilities
Hunt and map Active Directory resources using PowerShell enumeration scripts, and the tools BloodHound, and RPCClient
Seek out hidden Domain Admin and Administrator accounts to use for privilege escalation, lateral movement, and persistence.

\subsection{Post-Enumeration Activities in Active Directory}
RPCClient
BloodHound
PowerView

\subsection{Abusing Kerberos}
Discover Service Principal Names (SPNs) that are associated with normal user accounts
Use Kerberoasting attacks to steal service tickets or RC4 HMAC hashes tied to service accounts
Perform AS-REP Roasting to retrieve an encrypted AS-REP with a user's RC4 HMAC password
Kerberos Authentication Protocol and Delegation
AS-REP Roasting Attack
Kerberoasting
Kerberos Brute-Force Attacks

\subsection{Credential Dumping}
Domain Cache Credential
Local Administrator Password Solution (LAPS)
DCSync Attack
NTDS.dit
Golden Ticket
Silver Ticket

\subsection{Privilege Escalation}
Unconstrained Delegation
Resource-based Constrained Delegation (RBCD)
HiveNightmare
sAMAccountName Spoofing
SeBackupPrivilege
PrintNightmare

\subsection{Establishing and Maintaining Persistence in Active Directory}
Golden Certificate Attack
DSRM
AdminSDHolder
DC Shadow Attack
Skeleton Key

\subsection{Lateral Movement}
Pass-the-Ticket (PtT)
Pass-the-Cache
Overpass-the-Hash
Pass-the-Hash (PtH)

\subsection{Tools}
PowerShell-Empire
CrackMapExec
Impacket Suite
A to Z Mimikat \& Rubeus






\section{Part ?-References and Tools}
\section{AD Security Tools Catalog}
\subsection{Offensive Tools}
\subsubsection{PowerView}
\subsubsection{Mimikatz}
\subsubsection{Rubeus}
\subsubsection{Impacket Suite}
\subsection{Defensive Tools}
\subsubsection{ATA}
\subsubsection{Defender for Identity}
\subsubsection{PingCastle}
\subsubsection{\texttt{SYSMON}}

\section{Command and Script Reference}
\subsection{Common PowerShell, CMD, and Bash Scripts for AD Operations}
\subsection{Tool Syntax and Usage Examples}

\section{Glossary of Active Directory Security Terms}

\section{Appendices}
\subsection{A: AD Security Checklist}
\subsection{B: Detection Rule Samples}
\subsubsection{Sigma}
\subsubsection{Splunk Queries}
\subsubsection{KQL}
\subsection{C: AD Hardening GPO Examples}

\section{Part VI-Active Directory Enumeration and Local Privilege Escalation}
\subsection{Enumerating Useful Active Domain-Related Information}
\subsection{AD Local Privilege Escalation Techniques}
\subsection{Hunting for Domain Local Admin Privileges Using Multiple Methods}
\subsection{Abusing Enterprise AD Applications to Discover Attack Pathways}
\subsection{Security Defense Evasion}
\subsubsection{Techniques for Bypassing Security Control Mechanisms}
\subsubsection{Bypassing Signature-Based Antivirus}
\subsubsection{Pivoting and Obfuscation Methods}
\section{Part VII-Lateral Movement, Domain Privilege Escalation, and Persistence}
\subsection{Hunting for Domain Administrator Accounts}
\subsection{Hijacking High-Privileged Sessions}
\subsection{AD Credential Extraction}
\subsubsection{Credential Theft and Replay Attacks}
\subsubsection{Exfiltrating Credentials from Restricted Environments}
\subsubsection{Bypassing Allowlists}
\subsection{Leveraging Built-In Protocols to Escalate Privileges}
\subsection{Abusing Derivative Domain Local Admin Privileges}
\subsection{Understanding the Classic Kerberoast}
\subsubsection{Variants to Escalate Privileges}
\subsection{Exploiting Delegation Issues}
\subsection{Protected Groups Privilege Abuse}
\subsection{Kerberos}
\subsubsection{Abusing Kerberos Functionality to Persist Using DA Privileges}
\subsection{Golden and Silver Tickets}
\subsection{Abusing DC Safe Mode for Persistence}
\subsection{AdminSDHolder Manipulation}
\section{Part IIIX-Domain Persistence and Escalation to Enterprise Admins}
\subsection{ACL Modification for DCSync Attack}
\subsection{Security Host Descriptor Modification}
\subsection{Methods for Domain Controller Persistence}
\subsection{Abusing Trust Keys and \texttt{krbtgt}}
\subsection{Executing Intra-Forest Trust Attacks}
\subsection{AD Database Abuse to Achieve RCE}
\section{Part IX-Advanced Threat Analytics and Deception}
\subsubsection{Temporal Group Memberships}
\subsubsection{ACL and ACE Auditing}
\subsubsection{LAPS-Local Administrator Password Solution}
\subsubsection{Hardening SIDs}
\subsubsection{Kerberos Pre-Auth}
\subsubsection{SID Filtering}
\subsubsection{Selective Authentication}
\subsubsection{Credential Guard and Device Guard}
\subsubsection{PAWs and SAWs}
\subsection{Tiered Administration}
\subsubsection{Tier 0}
\subsubsection{Tier 1}
\subsubsection{Tier 2}
\subsection{EASE and Red Forests}
\subsection{Defensive Deception Tactics}
\subsubsection{Lures and Decoys}
\subsubsection{Canaries and Canary Tokens}
\subsubsection{Honeypots, Honeynets, and Honey Tokens}

\section{Introduction to Red Teaming}
\subsection{Defining Red Team}
\subsection{Red Team vs. Pentesting}
\subsectiob{Objectives and Benefits}
\subsection{Rules of Engagement (ROE) and Legal Aspects}
\subsection{Tools and Red Team Methodologies}
\section{Infrastructure and Command \& Control (C2)}
\subsection{Acquisition of Domains for Operations}
\subsection{Infrastructure Requirements}
\subsection{Setting Up and Securing C2 Infrastructure}
\subsection{Domain Fronting}
\subsection{HTTPS and DNS Redirectors}
\section{Initial Recognition}
\subsection{Passive Recognition: WHOIS, DNS, and Social Networks}
\subsection{Active Recognitiion: Network Scanning and Service Discovery}
\subsection{Identity by Email}
\subsection{Organizational Data Analysis}
\subsection{Best Practices for Keeping a Low Profile}
\section{Initial Compromise}
\subsection{Compromise Techniques}
\subsubsection{Web Shells}
\subsubsection{SQL Injection (SQLi)}
\subsubsection{Password Spraying}
\subsection{Preparation and Use of Social Engineering Attacks}
\subsection{Creation of Malicious Payloads}
\subsection{Payload Distribution}
\subsection{Attack Vector Customization}
\subsection{Cyberattack Simulation}
\section{Establishing a Bridgehead}
\subsection{Obfuscation Techniques}
\subsection{Bypassing and Evading Traditional Security Detection Mechanisms}
\subsection{Executing Malicious Code with .NET and PowerShell}
\subsection{Command Execution}
\subsection{Coded Scripts to Evade Detection}
\subsection{Post-Attack and Intelligence Gathering}
\subsectiob{Access and Detection Avoidance Strategies}
\section{Attacking Active Directory Red Team Style}
\subsection{Lateral Movement}
\subsection{Enumeration and Privilege Escalation in Active Directory}
\subsection{Mapping AD Relationships}
\subsection{Lateral Movement Propagation}
\subsectiob{The Importance of Situational Awareness and Personal OPSEC}
\section{Goal Attainment and Reporting}
\subsection{Database Attack Strategies}
\subsection{Sensitive Data Exfiltration Techniques}
\subsection{Preparing Detailed Vulnerability Identification and Remediation Reports}
\subsection{Verifying and Validating Corrective Measures: Re-Verification and Re-Validation Testing Procedures}
\subsection{Analyzing Post-Attack Effectiveness of Attack Methodologies}
\subsection{Measuring Impact of Attack and TTP Strategies}
\section{Frameworks and Methodologies}
\subsection{Introduction to Cyber Security Frameworks}
\subsubsection{MITRE ATT\&CK Framework}
\subsubsection{Lockheed Martin Cyber Kill Chain (CKC)}
\subsection{Understanding and Using Threat Intelligence}
\subsection{Planning and Executing Opponent Emulations}
\subsection{Analysis of Indicators of Compromise (IoC)}
\subsection{Technology Watch and Strategy Adaptation}
\section{Attack Infrastructure and Operational Security}
\subsection{Configuring and Managing Red Team Tools}
\subsection{Securing Attack Infrastructure}
\subsection{Setting Up Beacons and Redirectors}
\subsection{Maintaining Access and Continued Persistence}
\subsection{Risk Management}
\subsection{Reducing Your Attack Footprint}
\section{Malware Analysis and Reverse Engineering}
\subsection{Malware Analysis Tools}
\subsubsection{Ghidra}
\subsubsection{IDA Pro}
\subsubsection{OllyDBG}
\subsection{Malware Reverse Engineering}
\subsection{Portable Execution (PE) File Formats and Internal Structure}
\subsection{Malware Tactics}
\subsection{Importance of Malware Analysis}
\section{Capture the Red Team Flag}



















\subsection{}

This is the first section.

\blindtext

\addcontentsline{toc}{section}{Unnumbered Section}
\section*{Unnumbered Section}

\blindtext

\section{Second Section}

\blindtext
\end{document}