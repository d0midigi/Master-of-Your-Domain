%%%%%%%%%%%%%%%%%%%%% chapter.tex %%%%%%%%%%%%%%%%%%%%%%%%%%%%%%%%%
%
% sample chapter
%
% Use this file as a template for your own input.
%
%%%%%%%%%%%%%%%%%%%%%%%% Springer-Verlag %%%%%%%%%%%%%%%%%%%%%%%%%%
%

\motto{Use the template \emph{chapter.tex} to style the various elements of your chapter content.}

\chapter{Chapter Heading}

\label{intro} % Always give a unique label
% use \chaptermark{}

% to alter or adjust the chapter heading in the running head

\abstract*{Each chapter should be preceded by an abstract (no more than 200 words) that summarizes the content. The abstract will appear \textit{online} at \url{www.SpringerLink.com} and be available with unrestricted access. This allows unregistered users to read the abstract as a teaser for the complete chapter.

Please use the 'starred' version of the new \texttt{abstract} command for typesetting the text of the online abstracts (cf. source file of this chapter template abstract) and include them with the source files of your manuscript. Use the plain abstract command if the abstract is also to appear in the printed version of the book.}

\abstract{Enter content for abstract}

\section{Enter Section Title 1}
\label{sec:1}

Content Para1 lorem ipsum lorem ipsum lorem ipsum

Content Para2 lorem ipsum lorem ipsum lorem ipsum

template \emph{chapter.tex} together with the document class SVMono (monograph-type books) or SVMult (edited books) to style the various elements of your chapter content conformable to the Springer Nature layout.

\section{Section Content Title 2}
\label{sec:2}

% Always give a unique label
% and use \ref{<label>} for cross-references
% and \cite{<label>} for bibliographic references
% use \sectionmark{}
% to alter or adjust the section heading in the running head

Content Para1 lorem ipsum lorem ipsum lorem ipsum

Content Para 2 lorem ipsum lorem ipsum lorem ipsum

Content Para3 lorem ipsum lorem ipsum lorem ipsum

%Here is an itemized bulleted list below:%

Among the key use cases addressed by BTA are:
    \item Investigating the use of AdminSDHolder, a mechanism in AD that can silently propagate permissions to protected accounts like Domain Admins.
    \item Comparing snapshots of ADs state over time using exported NTDS.dit database files to detect unauthorized changes or backdoor insertion during a breach.

\section{Section Content Title 3}
\label{sec:3}

Content Para1 lorem ipsum lorem ipsum lorem ipsum

\subsection{Section Content Title 3.1}
\label{sec:3.1}

Content Para1 lorem ipsum lorem ipsum lorem ipsum

Content Para2 lorem ipsum lorem ipsum lorem ipsum

Content Para3 lorem ipsum lorem ipsum lorem ipsum

Content Para4 lporem ipsum lorem ipsum lorem ipsum

Content Para5 lorem ipsum lorem ipsum lorem ipsum

