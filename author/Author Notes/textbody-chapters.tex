% notes.tex
% My scratch notes — no preamble
Just some random text, code, or half-written LaTeX.

TEXT BODY / CHAPTERS
- Chapters contain the actual content of the book, i.e., text, figures, tables, and references.
- Chapters can be grouped together in parts; subparts are not possible. Only one chapter (e.g., an introduction) may precede the first part and would be the first chapter.
- Decide the numbering style for the chapters and apply this style consistently to all chapters: consecutively numbered (monographs or textbooks) or unumbered (edited books).
- If an introduction to the subject of the book (historical background, definitions, or methodology) is included, it should appear as the first chapter and thus be included in the chapter numbering. It can contain references, figures, and tables, just as any other chapter.

CHAPTER TITLES AND AUTHORS
- For edited books, include each chapter author's names (spelled out as they would be cited), affiliations, email addresses and ORCID (if available) after the chapter title.
- Ensure that the sequence of the author names is correct, the corresponding author is highlighted, and the title of your book is final when you submit your manuscript.
- Supply all emails, telephone numbers, and address of each author and editor. Once the manuscript has been delivered to Production, changes to title or authorship are no longer possible.

ABSTRACT
Chapter abstracts are strongly encouraged because they have been proven to significantly increase a book's visibility.

These will appear online at SpringerLink and other sites and will be available with unrestricted access to facilitate online searching and allow unregistered users to read the abstract as a teaser for the complete chapter.

- Begin each chapter with an abstract that summarizes the content of the chapter in no more than 200 words.
- Note that abstracts will not always appear in the print version of the book. For further details, contact your editor.

*If no abstract is submitted, they will use the first paragraph of the chapter instead.

KEYWORDS
- Each keyword should not contain more than two compound words, and each keyword phrase should start with an uppercase letter.
- When selecting keywords, think of them as terms that will help someone locate your chapter at the top of the search engine list using, for example, Google.
- Very broad terms (e.g., 'Case study' by itself) should be avoided as these will result in thousands of search results but will not result in finding your chapter immediately.
- When required, they allow three to six keywords per chapter.

HEADINGS AND HEADING NUMBERING
Heading levels should be clearly identified and each level should be uniquely and consistently formatted and/or numbered.
- Use the decimal system of numbering if headings are numbered.
- Never skip a heading level. The only exception are run-in headings which can be used at any hierarchical level.

TERMINOLOGY, UNITS, AND ABBREVIATIONS
- Technical terms and abbreviations should be defined the first time they appear in the text.
- Always use internationally accepted signs and symbols for units (also called SI units).
- Numerals should follow the British/American method of decimal points to indicate decimals and commas to separate thousands.

FORMAL STYLE AND TEXT FORMATTING
Springer Nature follows certain layouts and standards with regard to the presentation of the content, and the copy editors make sure that the manuscript conforms to these styles.
- When you receive the page proofs during the production of your book, do not make any changes that involve only matters of style.

EMPHASIS AND SPECIAL TYPE
- Italics should be used for emphasized words or phrases in running texts, but do not format entire paragraphs in italics.
- Use italics for species and genus names, mathematical / physical variables, and prefixes in chemical compounds.
- Bold formatting should only be used for run-in headings and small capitals for indicating optical activity (D- and L-dopa)
- Sans serif (e.g., Arial) and nonproportional font (e.g., Courier) can be used to distinguish the literal text of computer programs from running text.

BOXES
- Take note to not set entire pages as boxes because this affects online readability.
- For additional didactic elements such as examples, questions, exercises, summaries, or key messages in textbooks and in professional books, use a consistent style for each of these elements and submit a list of the styles used together with your manuscript.
- For LaTeX users, use the Springer Nature macro package to highlight these elements.

EQUATIONS AND PROGRAM CODE
- In Word, use the Math function, MathType, or Microsoft Equation editor to create your equations. Do not include the equations as images.
- In LaTeX, use the Math environment to create your equations.

TABLES
- Give each table a heading (caption). Add a reference to the table source at the end of the caption if the table was not created by yourself.
- Number tables consecutively using the chapter number (e.g., Do not write "the following table," instead write "see table 1.1," or "as table 1.1 shows") and ensure that all tables are cited in the text in sequential order. Do not write "the following table."
- Use the table function to create and format tables. Do not use the space bar or multiple tabs to separate columns.
- Do not use Excel to create tables as this can cause problems when converting the tables into the typesetting program and other formats.

FIGURES AND ILLUSTRATIONS
NUMBERING
- Number the figures using the chapter number (e.g., Do not write "the following figure," instead write "see figure 1.1," or "as figure 1.1 shows").
- Ensure that all figures are cited in the text in sequential order. Do not write "the following figure."

FIGURE CAPTIONS
- Give each figure a concise caption, describing accurately what the figure depicts. Include the captions at the end of the text file, not in the figure file.
- Identify all elements found in the figure in the figure caption; and use boxes, circles, etc., as coordinate points in graphs instead of color lines.
- If a figure is reproduced from a previous publication, include the source as the last item in the caption and ensure to deliver the corresponding permission forms from the rights holder(s).

FIGURE AND ILLUSTRATION FILES
- A figure is an object that is drawn or photographed; it does not consist solely of characters and thus cannot be keyed.
- Do not submit tabular materials as figures.
- Graphics and diagrams should be saved as an EPS file with the fonts embedded. MS Office files (Excel or PowerPoint) can be submitted in the original format (.xls, .xlsx, .ppt, .pptx). Scanned graphics in TIFF format should have a minimum resolution of 1200 dpi.
- Photos or drawings with fine shading should be saved as TIFF with a minimum resolution of 300 dpi.
- A combination of halftone and line art (e.g., photos containing line drawing or extensive lettering, color diagrams, etc.) should be saved as TIFF with a minimum resolution of 600 dpi.

REFERENCES
REFERENCE CITATIONS
Cite references in the text with author name(s) and year of publication in parenthees ("Harvard system").
- One author: (Miller 1991) or Miller (1991)
- Two authors: (Miller and Smith 1994) or Miller and Smith (1994)
- Three authors or more: (Miller et al. 1995) or Miller et al. (1995)
- If it is customary in your field, you can also cite with reference numbers in square brackets either sequential by citation or according to the sequence in an alphabetized list: [3, 7, 12].

REFERENCE LIST
- Include a reference list at the end of each chapter so that readers of single chapters of the book can make full use of the citations.
- References at the end of the book cannot be linked to citations in the chapters.
- Do not include reference lists at the end of the chapter section, at the end of a book part, in a preface or an appendix.
- Include all works that are cited in the chapter and that have been published (including URLs from the internet) or accepted for publication. Personal communications and unpublished works should only be mentioned in the text. Do not use footnotes as a substitute for a reference list.
- Entries in the list must be listed alphabetically except in the numbered system of sequential citation.

THE RULES FOR ALPHABETIZATION ARE:

1. First, all works by the author alone, ordered chronologically by year of publication.
2. Next, all works by the author with a coauthor, ordered alphabetically by coauthor
3. Finally, all works by the author with several coauthors, order chronologically by year of publication.
4. For authors using EndNote software to create the reference list, Springer Nature provides output styles that support the formatting of in-text citations and reference lists.
5. For authors using BiBteX, the style files are included in the Springer Nature's LaTeX package.

REFERENCE STYLES
Springer Nature follows certain standards with regard to the presentation of the reference list. They are based on reference styles that were established for various disciplines in the past and have been adjusted to facilitate automated processing and citation linking. This allows them to easily cross-link the cited references with the original publication. References will be revised in production in accordance with their house styles.

Choose the appropriate style for your subject from the list below. Note that the adapted and standardized forms are based on, but differ slightly from, certain recommended styles (e.g., APA, Chicago).















