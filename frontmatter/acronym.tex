%%%%%%%%%%%%%%%%%%%%%%acronym.tex%%%%%%%%%%%%%%%%%%%%%%%%%%%%%%%%%%%%%%%%%
% sample list of acronyms
%
% Use this file as a template for your own input.
%
%%%%%%%%%%%%%%%%%%%%%%%% Springer %%%%%%%%%%%%%%%%%%%%%%%%%%

\extrachap{Abbreviations and Acronyms}

This list contains commonly used abbreviations and acronyms related to cybersecurity, information, system, and network security, along with their generally accepted or preferred definitions. It is intended to serve as a practical reference for IT professionals, cybersecurity practitioners, and others working within system and network security domains.

It is important to note that the spelling, capitalization, and definitions of abbreviations and acronyms can vary significantly. This variation is understandable: while certain terms - such as \textbf{WWW} - have a universally recognized meaning within the field, others - like \textbf{IA} or \textbf{MAC} - may have multiple valid interpretations depending on the context. Some acronyms bear little resemblance to the words they represent, such as \textbf{TMOVS} \textit{(Modes of Operation Validation System for the Triple DES Algorithm)}, while others feature unusual capitalization or spelling, such as \textbf{ebXML} \textit{(Electronic Business using eXtensible Markup Language)} or \textbf{OECD} \textit{(Organization for Economic Co-operation and Development)}. These inconsistencies can lead to misinterpretation or confusion when definitions are inaccurately presented or inconsistently applied.

This list is designed to reduce such ambiguity by offering clear, standardized definitions of frequently encountered terms. It does not aim to be exhaustive, nor does it include every abbreviation or acronym found contained in system and network security literature.

The following conventions were used in preparing this list:

\begin{itemize}
    \item Abbreviations and acronyms generally appear in \textbf{uppercase letters}, though exceptions exist - e.g., \textbf{m} (meter) and \textbf{dBm} (decibels referenced to one milliwatt).
    \item If multiple meanings exist for the same abbreviation or acronym, the term is \textit{italicized and repeated} for each entry. Definitions are listed alphabetically.
\end{itemize}

    
\begin{description}[CABR]
\item[A]{Address Resource Record Type}
\item[A\&A]{Assessment and Authorization}
\item[A-GPS]{Assisted Global Positioning System}
\item[AA]{Attribute Authority}
\item[AAA]{Authentication, Authorization, Accountability}
\item[AAAA]{Authentication, Authorization, Accountability, Availability}
\item[AAAK]{Authentication, Authorization, and Accounting Key}
\item[AACS]{Authenticated Access Control Security}
\item[ABAC]{Attribute-based Access Control}
\item[ABM]{Asset Baseline Monitor | Management}
\item[ABR]{Area Border Router}
\item[ACAS]{Assured Compliance Assessment Solution}
\item[ACE]{Access Control Entry}
\item[ACK]{Acknowledgment}
\item[ACL]{Access Control List}
\item[ACM]{Association for Computing Machinery}
\item[ACO]{Authenticated Cipher Offset}
\item[ACS]{Access Control Security}
\item[AD]{Active Directory}
\item[AD CS]{Active Directory Certificate Services}
\item[AD DS]{Active Directory Domain Services}
\item[AD FS]{Active Directory Federated Services}
\item[ADUC]{Active Directory Users and Computers}
\item[ADP]{Automated Data Processing}
\item[ADS]{Alternate Data Stream}
\item[ADSL]{Asymmetric Digital Subscriber Line}
\item[AES]{Advanced Encryption Standard}
\item[AFL]{American Fuzzing Lop}
\item[AFH]{Adaptive Frequency Hopping}
\item[AFI]{Air Force Installation}
\item[AFOSI]{Air Force Office of Special Investigation}
\item[AFPD]{Air Force Policy Directive}
\item[AH]{Authentication Header}
\item[AIDC]{Automatic Identification and Data Capture}
\item[AIMS]{Automated Infrastructure Management System}
\item[AIS]{Automated Information System}
\item[AIT]{Automatic Identification Technology}
\item[AJAX]{Asynchronous JavaScript and XML}
\item[AK]{Authorization Key}
\item[AKID]{Authorization Key Identifier}
\item[AKM]{Authentication Key Management}
\item[ALG]{Application Layer Gateway}
\item[ALCON]{All Concerned}
\item[AMIDS]{Audit Monitoring and Intrusion Detection System}
\item[ANSI]{American National Standards Institute}
\item[AO]{Authorizing Official}
\item[AP]{Access Point}
\item[APC]{Angle Polished Connector}
\item[API]{Application Programming Interface}
\item[APIPA]{Automatic Private Internet Protocol Addressing}
\item[APK]{Android Package}
\item[APL]{Application Programming Language}
\item[APT]{Advanced Persistent Threat}
\item[APWG]{Anti-Phishing Working Group}
\item[ARIN]{American Registry for Internet Numbers}
\item[ARP]{Address Resolution Protocol}
\item[ARPA]{Advanced Research Projects Agency}
\item[ARPANet]{Advanced Research Projects Agency Network}
\item[ASC]{Anti-Spyware Coalition}
\item[ASC X9]{Accredited Standards Committee X9}
\item[AS]{Autonomous System}
\item[ASIC]{Application Specific Integrated Circuit}
\item[ASN]{Autonomous System Number}
\item[ASN.1]{Abstract Syntax Notation One}
\item[ASP]{Active Server Pages}
\item[ASIMS]{Automated Security Incident Measuring System}
\item[ASLR]{Address Space Layout Randomization}
\item[ATA]{Advanced Technology Attachment}
\item[ATC]{Authorization to Connect}
\item[ATD]{Authorization Termination Date}
\item[ATO]{Authorization to Operate}
\item[ATM][Asynchronous Transfer Mode]
\item[ATT\&CK]{Adversarial Tactics, Techniques \& Common Knowledge}
\item[AU]{Audit}
\item[AUI]{Attachment Unit Interface}
\item[AUP]{Acceptable Use Policy}
\item[AV]{Antivirus}
\item[AVIX]{Anti-Virus Information Exchange Network}
\item[AVP]{Attribute-Value Pair}
\item[AWS]{Amazon Web Services}
\item[AXFR]{Authoritative Zone Transfer}
\item[AZ] {Availability Zone}
\item[] % empty label
\vspace*{2em}
%\noindent\hspace*{-1.3em}\makebox[0pt][l]{\textbf{\LARGE B}}\null\\[1em]
\item[BCP] Business Continuity Plan
\item[BCS] Business Connectivity Services
\item[BDR]{Backup Designated Router}
\item[BERT]{Bit Error Rate Test}
\item[BGP]{Border Gateway Protocol}
\item[BLE]{Bluetooth Low Energy}
\item[BNC]{Bayonet Nut Connector} 
\item[BootP]{Boot Protocol}
\item[BPDU]{Bridge Protocol Data Unit}
\item[BRI]{Basic Rate Interface}
\item[BSSID]{Basic Service Set Identifier}
\item[BYOD]{Bring Your Own Device}

\item[C]
\item[CAM]{Channel Access Method}
    \item \textit{[CAM]{Content Addressable Memory}}
\item[CARP]{Common Address Redundancy Protocol}
\item[CAT]{Category (cable)}
\item[CCTV]{Closed Circuit Television}
\item[CDMA]{Code Division Multiple Access}
\item[CDMA/CD]{Carrier Sense Multiple Access / Collision Detection}
\item[CHAP]{Challenge Handshake Authentication Protocol}
\item[CIDR]{Classless Inter-Domain Routing}
\item[CIFS]{Common Internet File System | Services}
\item[CLI]{Command Line Interface}
\item[CNAME]{Canonical Name}
\item[COOP]{Continuity of Operations}
    \item \textit{[COOP]{Concurrent Object-Oriented Programming}}
\item[COS]{Class of Service}
\item[CPU]{Central Processing Unit}
\item[CRAM]{Challenge-Response Authentication Mechanism - Message Digest 5}
\item[CRC]{Cyclic Redundancy Check}
\item[CSMA/CA]{Carrier Sense Multiple Access/Collision Avoidance}
\item[CSU]{Channel Service Unit}
\item[CWDM]{Course Wave Division Multiplexing}

\item[D]
\item[dB]{Decibel}
\item[DCS]{Distributed Computer System}
\item[DDoS]{Distributed Denial of Service}
\item[DHCP]{Dynamic Host Configuration Protocol}
\



















\end{description}