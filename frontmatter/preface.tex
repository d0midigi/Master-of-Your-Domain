%%%%%%%%%%%%%%%%%%%%%%preface.tex%%%%%%%%%%%%%%%%%%%%%%%%%%%%%%%%%%%%%%%%%
% sample preface
%
% Use this file as a template for your own input.
%
%%%%%%%%%%%%%%%%%%%%%%%% Springer %%%%%%%%%%%%%%%%%%%%%%%%%%

\preface

%% Please write your preface here
\emph{Master of Your Domain: Hacking and Defending Active Directory (AD) for Ethical Hackers.tex} is a practical exploration of how today's modern Active Directory (AD) environments are breached, abused, and defended in the real-world. This book was written for aspiring and seasoned ethical hackers, red and blue teamers, penetration testers, defenders, and anyone who wants to understand the true dynamics of modern Windows network compromise - beyond theory, beyond checklists.
Active Directory remains one of the most targeted and misunderstood components in enterprise cyber security. It's vast, complex, and oftentimes misconfigured. For attackers, it's a playground - a goldmine. For defenders, it's a battlefield in the truest sense. This book aims to arm you with both perspectives of both offensive and defensive cyber security. You'll find detailed coverage of post-exploitation tactics, including certificate abuse (e.g., Golden Tickets), persistence techniques, lateral movement, privilege escalation, and the tools used and methods in which to attack and defend the Kerberos authentication protocol. But more than just technical content, this book emphasizes mindset: how an attacker thinks when they're inside, undetected, and methodically deciding their next move. Understanding that psychology is just as critical as knowing the tools.
Each chapter is based on real-world engagements and lessons learned the hard way. No two networks are the same - but patterns emerge, and opportunities repeat themselves. The goal of this book is to help you recognize those opportunities. Whether you are breaking in, or locking things down.
To those who supported the long hours, reviewed the raw material, and challenged my thinking - thank you. This book wouldn't exist without your input, insight, and encouragement.


A preface\index{preface} is a book's preliminary statement, usually written by the \textit{author or editor} of a work, which states its origin, scope, purpose, plan, and intended audience, and which sometimes includes afterthoughts and acknowledgments of assistance. 

When written by a person other than the author, it is called a foreword. The preface or foreword is distinct from the introduction, which deals with the subject of the work.

Customarily \textit{acknowledgments} are included as last part of the preface.
 

\vspace{\baselineskip}
\begin{flushright}\noindent
Place(s),\hfill {\it Firstname  Surname}\\
month year\hfill {\it Firstname  Surname}\\
\end{flushright}


